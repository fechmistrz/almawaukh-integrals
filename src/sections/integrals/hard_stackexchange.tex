%

\subsection{Znalezione na math.stackexchange.com}
\subsection{Znalezione na math.stackexchange.com} % SOLUTION
Wszystkie poniższe całki pojawiają się na stronie na math.stackexchange.com.

% https://math.stackexchange.com/questions/541751/how-prove-this-i-int-0-infty-frac1x-ln-left-frac1x1-x-right2/541861#541861
\begin{problem}[pytanie 541751]
    \label{stack_541751}%
    \begin{equation}
        I = \int_0^\infty \frac{1}{x} \log \left(\frac{1+x}{1-x}\right)^2 \,\mathrm{d}x = \pi^2
    \end{equation}
\end{problem}

\textbf{Problem \ref{stack_562694}} -- podstawiamy $y = (1+x) / (1-y)$ i mamy % SOLUTION
\begin{align} % SOLUTION
    I & = 2 \int_{-1}^1 \frac{\log y^2}{1-y^2} \,\mathrm{d}y \\ % SOLUTION
        & = 8 \int_0^1 \frac{\log y}{1-y^2} \, \mathrm{d}{y} \\ % SOLUTION
        & = 8 \sum_{k=0}^\infty \int_0^1 y^{2k} \log y \,\mathrm{d} y \\ % SOLUTION
        & = 8 \sum_{k=0}^\infty \frac{1}{(2k+1)^2} \\ % SOLUTION
        & = 8 \cdot \frac{\pi^2}{8} = 8. % SOLUTION
\end{align} % SOLUTION

% https://math.stackexchange.com/q/562694
\begin{problem}[pytanie 562694]
    \label{stack_562694}%
    \begin{equation}
        \int_{-1}^1 \frac{1}{x} \sqrt{\frac{1+x}{1-x}} \log \frac{2x^2+2x+1}{2x^2-2x+1} \,\mathrm{d}x = 4 \pi \operatorname{arccot} \sqrt{\phi}.
    \end{equation}
\end{problem}

\textbf{Problem \ref{stack_562694}} -- patrz przypis\footnote{\url{https://math.stackexchange.com/questions/562694/}}. % SOLUTION

% TODO: https://math.stackexchange.com/a/942440

%