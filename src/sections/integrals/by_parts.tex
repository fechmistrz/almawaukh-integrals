%

\section{Całkowanie przez części}

\begin{proposition}[wzór na całkowanie przez części]
\label{prp_int_by_parts}%
    Jeśli funkcje $f, g \colon I \to \R$ są różniczkowalne, to
    \begin{equation}
        \int f(x) g'(x) \,\mathrm{d}x = f(x) g(x) - \int f'(x) g(x) \,\mathrm{d} x.
    \end{equation}
\end{proposition}

\begin{proof}
    Całkujemy obie strony wzoru na pochodną iloczynu $(fg)' = fg' + f'g$, a następnie porządkujemy strony równości.
\end{proof}

% Banaś, Wędrychowicz: 12.1
\begin{problem_with_solution}
    \label{banas_12_1}%
    $\int x \sin x \,\mathrm{d} x$.
\end{problem_with_solution}

\textbf{Problem \ref{banas_12_1}}. % SOLUTION
Całkujemy przez części, $f(x) = x$, $g'(x) = \sin x$. % SOLUTION
\begin{align} % SOLUTION
    \int x \sin x \,\mathrm{d} x & = -x \cos x - \int - \cos x \, \mathrm{d}x \\ % SOLUTION
                                 & = -x \cos x + \sin x. % SOLUTION
\end{align} % SOLUTION

Analogicznie obliczamy całki:

% Banaś, Wędrychowicz: 12.2
\begin{problem}
    \label{banas_12_2}%
    $\int x \cos x \,\mathrm{d} x = x \sin x + \cos x$.
\end{problem}

% Banaś, Wędrychowicz: 12.3
% Banaś, Wędrychowicz: 12.4 - podobna
% Banaś, Wędrychowicz: 12.5 - podobna
\begin{problem}
    \label{banas_12_3}%
    $\int x \exp x \,\mathrm{d} x = (x-1) e^x$.
\end{problem}

% Banaś, Wędrychowicz: 12.6
\begin{problem_with_solution}
    \label{banas_12_6}%
    $\int x \arctan x \,\mathrm{d} x$.
\end{problem_with_solution}

\textbf{Problem \ref{banas_12_6}}. % SOLUTION
Całkujemy przez części, $f(x) = \arctan x$, $g'(x) = x$. % SOLUTION
\begin{align} % SOLUTION
    \int x \arctan x \, \mathrm{d} x % SOLUTION
    & = \frac 12 x^2 \arctan x - \int \frac{x^2 \,\mathrm{d}x}{2(x^2+1)} \\ % SOLUTION
    & = \frac 12 x^2 \arctan x - \frac 12 \left(\int 1 \,\mathrm{d}x - \int \frac{\mathrm{d}x}{x^2+1} \right) \\ % SOLUTION
    & = \frac 12 x^2 \arctan x - \frac 12 \left(x - \arctan x \right) \\ % SOLUTION
    & = \frac 12 \left((x^2+1)\arctan x - x \right). % SOLUTION
\end{align} % SOLUTION

% Banaś, Wędrychowicz: 12.7
\begin{problem_with_solution}
    \label{banas_12_7}%
    $\int x^n \log x \,\mathrm{d} x$, gdzie $n \in \N$.
\end{problem_with_solution}

\textbf{Problem \ref{banas_12_7}}. % SOLUTION
Całkujemy przez części, $f(x) = \log x$, $g'(x) = x^n$. % SOLUTION
\begin{align} % SOLUTION
    \int x^n \log x \, \mathrm{d} x & = \frac{x^{n+1} \log x}{n+1} - \int \frac{x^n \,\mathrm{d} x}{n+1} \\ % SOLUTION
                                    & = \frac{x^{n+1} \log x}{n+1} - \frac{x^{n+1}}{(n+1)^2}. % SOLUTION
\end{align} % SOLUTION

% Banaś, Wędrychowicz: 12.8
\begin{problem_with_solution}
    \label{banas_12_8}%
    $\int \arccos x \,\mathrm{d} x$.
\end{problem_with_solution}

\textbf{Problem \ref{banas_12_8}}. % SOLUTION
Całkujemy najpierw przez części, $f(x) = \arccos x$, $g'(x) = 1$, żeby następnie podstawić $u = 1 - x^2$, $\mathrm{d} u = -2x \mathrm{d}x$: % SOLUTION
\begin{align} % SOLUTION
    \int \arccos x \, \mathrm{d} x & = x \arccos x - \int  \frac{-x \,\mathrm{d}x}{\sqrt{1-x^2}} \\ % SOLUTION
    & = x \arccos x - \frac 12 \int \frac {\mathrm{d}u}{\sqrt{u}} \\ % SOLUTION
    & = x \arccos x - \sqrt{1 - x^2}. % SOLUTION
\end{align} % SOLUTION

% Banaś, Wędrychowicz: 12.9
\begin{problem}
    \label{banas_12_9}%
    $\int \arcsin x \,\mathrm{d} x = x \arcsin x + \sqrt{1-x^2}$.
\end{problem}

% Banaś, Wędrychowicz: 12.10
\begin{problem_with_solution}
    \label{banas_12_10}%
    $\int x (\tan x)^2 \,\mathrm{d} x$.
\end{problem_with_solution}

\textbf{Problem \ref{banas_12_10}}. % SOLUTION
Całkujemy przez części, $f(x) = x$, $g'(x) = (\tan x)^2$. % SOLUTION
\begin{align} % SOLUTION
    \int x (\tan x)^2 x \, \mathrm{d} x & = x (\tan x - x) - \int (\tan x - x) \,\mathrm{d}x \\ % SOLUTION
    & = x (\tan x - x) - \left(-\log(\cos(x)) - \frac{x^2}{2}\right). % SOLUTION
\end{align} % SOLUTION

% Banaś, Wędrychowicz: 12.11
\begin{problem_with_solution}
    \label{banas_12_11}%
    $\int x (\cos x)^2 \,\mathrm{d} x$.
\end{problem_with_solution}

\textbf{Problem \ref{banas_12_11}}. % SOLUTION
Ponieważ $\cos 2x = 2 \cos^2 x - 1$, potrzebujemy znaleźć prostszą całkę  % SOLUTION
\begin{align} % SOLUTION
    \int x \cos 2x \, \mathrm{d} x. % SOLUTION
\end{align} % SOLUTION
Całkujemy przez części: $f(x) = x$, $g'(x) = \cos 2x$, co prowadzi do jeszce prostszej całki funkcji $\sin 2x$. % SOLUTION
Ostatecznie % SOLUTION
\begin{align} % SOLUTION
    \int x \cos 2x \, \mathrm{d} x = \frac 1 8 \left(2x^2 + 2x \sin 2x + \cos 2x\right). % SOLUTION
\end{align} % SOLUTION

% Banaś, Wędrychowicz, 12.12 to całka z x log(x^2+1), ale tam wystarczy podstawić u = x^2 + 1, wtedy du = 2x dx.
% Banaś, Wędrychowicz, 12.16
% Banaś, Wędrychowicz, 12.17.
\begin{problem_with_solution}
    \label{banas_12_12}%
    $\int (\log x)^n \,\mathrm{d}x$.
\end{problem_with_solution}

\textbf{Problem \ref{banas_12_12}}. % SOLUTION
Całkujemy przez części, $f(x) = (\log x)^n$, $g'(x) = 1$. % SOLUTION
Dostajemy początek rekurencji: % SOLUTION
\begin{equation} % SOLUTION
    \int (\log x)^n \, \mathrm{d}x = x (\log x)^n - n \int (\log x)^{n-1} \,\mathrm{d} x. % SOLUTION
\end{equation} % SOLUTION
z warunkiem brzegowym: % SOLUTION
\begin{equation} % SOLUTION
    \int \log x\, \mathrm{d}x = x\log x - x. % SOLUTION
\end{equation} % SOLUTION

% Banaś, Wędrychowicz, 12.13.
\begin{problem_with_solution}
    \label{banas_12_13}%
    Niech $n$ będzie liczbą naturalną, wtedy
    \begin{equation}
        I_n = \int x^n e^x \,\mathrm{d} x = e^x \sum_{k=0}^n (-1)^{n-k} \frac{n!}{k!}x^k.
    \end{equation}
\end{problem_with_solution}

\textbf{Problem \ref{banas_12_13}}. % SOLUTION
Dowiedziemy tego indukcyjnie. % SOLUTION
Dla $n = 0$, całka jest elementarna. % SOLUTION
Jeżeli $n \ge 1$, to całkujemy przez części: $f(x) = x^n$, $g'(x) = e^x$ i dostajemy zależność rekurencyjną % SOLUTION
\begin{equation} % SOLUTION
    I_n = x^n e^x - nI_{n-1}. % SOLUTION
\end{equation} % SOLUTION

% Banaś, Wędrychowicz, 12.14.
% Banaś, Wędrychowicz, 12.15.
% Banaś, Wędrychowicz, 12.25.
\begin{problem_with_solution}
    \label{banas_12_14}%
    $\int x^3 \sin x \, \mathrm{d}x$.
\end{problem_with_solution}

\textbf{Problem \ref{banas_12_14}}. % SOLUTION
Całkujemy przez części, $f(x) = x^3$, $g'(x) = \sin x$. % SOLUTION
Dostajemy początek rekurencji: % SOLUTION
\begin{equation} % SOLUTION
    \int x^3 \sin x \, \mathrm{d}x = - x^3 \cos x - \int - 3x^2 \cos x \,\mathrm{d}x % SOLUTION
\end{equation} % SOLUTION
rozwiązaniem której jest $3 (x^2-2) \sin x + x (6-x^2) \cos x$. % SOLUTION

\begin{problem}
    % pomocnicza dla Banaś 12.19
    \label{banas_12_19_auxilia}%
    \begin{equation}
        \int e^x \sin x \,\mathrm{d}x = \frac {e^x} 2 (\sin x - \cos x).
    \end{equation}
\end{problem}

% Banaś, Wędrychowicz, 12.22.
% \begin{problem}
% Banaś-Wędrychowicz, 12.22. % x^2 e^x sin x
% \end{problem}

% Banaś, Wędrychowicz, 12.23.
% \begin{problem}
% Banaś-Wędrychowicz, 12.23. % x / (sin ^2 x)
% \end{problem}

% Banaś, Wędrychowicz, 12.24.
% \begin{problem}
% Banaś-Wędrychowicz, 12.24. % x arcsin x / (1 - x^2)
% \end{problem}

% Banaś, Wędrychowicz, 12.26.
\begin{problem_with_solution}
    \label{banas_12_26}%
    \begin{equation}
        \int \frac{x \log(\sqrt{x^2+1}+x)}{\sqrt{x^2+1}} \,\mathrm{d}x = \sqrt{x^2+1} \arsinh x - x.
    \end{equation}
\end{problem_with_solution}

\textbf{Problem \ref{banas_12_26}}. % SOLUTION
Zauważamy, że $\log(\sqrt{x^2+1} + x) = \arsinh x$, a następnie całkujemy przez części: $f(x) = \arsinh x$, $g'(x) = x / \sqrt{x^2+1}$, wtedy $f(x) = 1/\sqrt{x^2+1}$, $g(x) = \sqrt{x^2+1}$. % SOLUTION

% Banaś, Wędrychowicz, 12.27.
% \begin{problem}
% Banaś-Wędrychowicz, 12.27. % arc tg sqrt (x)
% \end{problem}

% Banaś, Wędrychowicz, 12.28.
% \begin{problem}
% Banaś-Wędrychowicz, 12.28. % x e ^ (arctg x) / (1+x^2)^1.5
% \end{problem}
%

\begin{problem_with_solution}
    \label{banas_12_21_auxilia}%
    $I = \int \sec^3 x \,\mathrm{d}x$.
\end{problem_with_solution}

\textbf{Problem \ref{banas_12_21_auxilia}}. % SOLUTION
Całkujemy najpierw przez części, $f(x) = \sec x$, $g'(x) = \sec^2 x$. % SOLUTION
Pamiętając, że $\tan^2 x = \sec^2 x - 1$, mamy % SOLUTION
\begin{align} % SOLUTION
    I & = \sec x \tan x - \int \sec x \tan^2 x \,\mathrm{d} x \\ % SOLUTION
        & = \sec x \tan x - I + \int \sec x \,\mathrm{d}x \\ % SOLUTION
    2I & = \sec x \tan x + \log |\tan x + \sec x|. % SOLUTION
\end{align} % SOLUTION
