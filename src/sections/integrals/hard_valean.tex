\subsection{Prawie niemożliwe całki}
% SOLUTION
\subsection{Prawie niemożliwe całki}
% SOLUTION
Wszystkie poniższe całki pojawiają się w książce Valeana \cite{valean19}.

\begin{problem_with_solution}
    \label{valean_grundpreis}%
    Niech $y \in (-1, 1)$.
    Wtedy
    \begin{equation}
        \int_0^1 \frac{\mathrm{d}x}{(1+yx) \sqrt{1-x^2}} = \frac{\arccos y}{\sqrt{1-y^2}}.
    \end{equation}
\end{problem_with_solution}

% SOLUTION
\textbf{Problem \ref{valean_grundpreis}} -- 
patrz \cite[s. 1]{valean19}.
% SOLUTION

\begin{problem_with_solution}
    \label{valean_zeugenstand}%
    Niech $m, n$ będą liczbami naturalnymi.
    Wtedy
    \begin{equation}
        \int_0^1 x^m \log^n x \,\mathrm{d} x = \frac{(-1)^n \cdot n!}{(m+1)^{n+1}}.
    \end{equation}
\end{problem_with_solution}

% SOLUTION
\begin{solution}[do problemu \ref{valean_zeugenstand}]
    Patrz \cite[s. 1]{valean19}.
\end{solution}
% SOLUTION


Niech $H_{n}^{(m)} = 1 + 1/2^m + \ldots + 1/n^m$ oznacza $n$-tą uogólnioną liczbę harmoniczną.

\begin{problem_with_solution}
    \label{valean_1_3}%
    Rozpatrujemy rodzinę całek
    \begin{equation}
        I_{k,n} := \int_0^1 x^{n-1} \log^k (1-x) \,\mathrm{d} x.
    \end{equation}
    Mamy:
    \begin{align}
        I_{1,n} & = - \frac{H_n}{n} \\
        I_{2,n} & = \frac{H_n^2 + H_n^{(2)}}{n} \\
        I_{3,n} & = - \frac{H_n^3 + 3H_nH_n^{(2)} + 2H_n^{(3)}}{n} \\
        I_{4,n} & = \frac{H_n^4 + 6H_n^2 H_n^{(2)} + 8H_nH_n^{(3)} + 3(H_n^{(2)})^2 + 6H_n^{(4)}}{n}.
    \end{align}
\end{problem_with_solution}

% (Valean nazywa to ,,four logarithmic integrals strongly connected with the league of harmonic series'').

% SOLUTION
\begin{solution}[do problemu \ref{valean_1_3}]
    Patrz \cite[s. 2]{valean19}.
\end{solution}
% SOLUTION

\begin{problem_with_solution}
    \label{valean_1_5}%
    Niech $s > 0$ będzie liczbą rzeczywistą, zaś $\psi$ oznacza funkcję digamma.
    Wtedy
    \begin{align}
        \int_0^1 \frac{x^{s-1}}{x+1} \,\mathrm{d} x & = \psi(s) - \psi\left(\frac s2\right) - \log 2 \\
        \int_0^\infty e^{-sx} \tanh x \,\mathrm{d} x & = \frac 1 2 \left[\psi\left(\frac{s+2}{4}\right) - \psi \left(\frac s4 \right) - \frac 2 s\right]. 
    \end{align}
\end{problem_with_solution}

% (Valean nazywa to ,,a couple of practical definite integrals expressed in terms of the digamma function'').

% SOLUTION
\begin{solution}[do problemu \ref{valean_1_5}]
    Patrz \cite[s. 3]{valean19}.
\end{solution}
% SOLUTION

\begin{problem_with_solution}
    \label{valean_1_7}%
    \begin{align}
        \int_0^1 \frac{1}{x} \log^2 (1+x) \,\mathrm{d}x & = \frac{1}{4} \zeta(3) \\
        \int_0^1 \frac{1}{x} \log (1+x) \log (1-x) \,\mathrm{d}x & = -\frac{5}{8} \zeta(3)
    \end{align}
\end{problem_with_solution}

% two little tricky classical logarithmic integrals

% SOLUTION
\begin{solution}[do problemu \ref{valean_1_7}]
    Patrz \cite[s. 4]{valean19}.
\end{solution}
% SOLUTION

\begin{problem_with_solution}[]
    \label{valean_1_8}%
    \begin{align}
        \int_0^1 [\log(1+x) \log(1-x)]^2 \,\mathrm{d} x & =
        24 - 8 \zeta(2)- 8 \zeta(3) - \zeta(4) \\
        & + 8 \log(2)\zeta(2) + 8 \log(2)\zeta(3) \\
        & - 4 \log^2(2)\zeta(2) \\
        & - 24 \log(2) + 12 \log^2(2)- 4 \log^3(2) + \log^4(2); 
    \end{align}
\end{problem_with_solution}

% a special trio of integrals

% SOLUTION
\begin{solution}[do problemu \ref{valean_1_8}]
    Patrz \cite[s. 4, 5]{valean19}.
\end{solution}
% SOLUTION

\begin{problem_with_solution}
    \label{valean_1_10}%
    Niech $n \ge 1$ będzie liczbą naturalną.
    Rozpatrujemy rodzinę całek z parametrem
    \begin{equation}
        I_n = \int_0^1 \frac 1 x \log(1-x) \log^{2n} x \log (1+x) \,\mathrm{d}x.
    \end{equation}
    Znaleźć $I_n$ lub, jeśli jest to za trudne, pokazać, że
    \begin{align}
        I_1 & = \frac 3 4 \zeta (2) \zeta (3) - \frac {27}{16} \zeta(5), \\
        I_2 & = \frac 9 4 \zeta (3) \zeta (4) + \frac{45}{4} \zeta(2) \zeta(5) - \frac{363}{16} \zeta (7), \\
        I_3 & = \frac{2835}{8} \zeta(2) \zeta (7) + \frac {135}{8} \zeta (3) \zeta (6) + \frac {675}{8} \zeta (4) \zeta (5) - \frac {22635}{32} \zeta (9).
    \end{align} 
\end{problem_with_solution}

% the evaluation of a class of logarithmic integrals using a slightly modified result from ,,Table of Integrals, Series and Products'' by I. S. Gradshteyn and I. M. Ryzhik together with a series result elementarily proved by Guy Bastien

% SOLUTION
\begin{solution}[do problemu \ref{valean_1_10}]
    Patrz \cite[s. 6, 7]{valean19}.
\end{solution}
% SOLUTION

\begin{problem_with_solution}
    \label{valean_1_13}%
    \begin{equation}
        \int_0^1 \frac{x \log (1 \pm x)}{1 + x^2} \, \mathrm{d} x = \frac 1 8 \left(\log^2 (2) + \frac{\pm 3 - 2}{2} \zeta(2)\right).
    \end{equation} 
\end{problem_with_solution}

% A Special Pair of Logarithmic Integrals with Connections in the Area of the Alternating Harmonic Series

% SOLUTION
\begin{solution}[do problemu \ref{valean_1_13}]
    Patrz \cite[s. 8]{valean19}.
\end{solution}
% SOLUTION

% Another Special Pair of Logarithmic Integrals with Connections in the Area of the Alternating Harmonic Series

\begin{problem_with_solution}
    \label{valean_1_14}%
    \begin{align}
        16 \int_0^1 \frac{x}{1+ x^2} \log (1 - x) \log x \,\mathrm{d}x & = \frac{41}{4} \zeta(3) - 9 \log(2) \zeta(2) \\
        16 \int_0^1 \frac{x}{1+ x^2} \log (1 + x) \log x \,\mathrm{d}x & = -\frac{15}{4} \zeta(3) + 3 \log(2) \zeta(2) \\
    \end{align} 
\end{problem_with_solution}

% A Special Pair of Logarithmic Integrals with Connections in the Area of the Alternating Harmonic Series

% SOLUTION
\begin{solution}[do problemu \ref{valean_1_14}]
    Patrz \cite[s. 8]{valean19}.
\end{solution}
% SOLUTION

\begin{problem_with_solution}
    \label{valean_1_17}%
    \begin{align}
        \int_0^1 \int_0^1 \frac{\log y - \log x}{\log (- \log x) - \log(- \log y)} \,\mathrm{d}x \,\mathrm{d}y = \frac{7 \zeta(3)}{6 \zeta (2)}.
    \end{align} 
\end{problem_with_solution}

% Let’s Take Two Double Logarithmic Integrals with Beautiful Values Expressed in Terms of the Riemann Zeta Function

% SOLUTION
\begin{solution}[do problemu \ref{valean_1_17}]
    Patrz \cite[s. 10]{valean19}.
\end{solution}
% SOLUTION

\begin{problem_with_solution}
    \label{valean_1_18}%
    Niech $G$ oznacza stałą Catalana.
    \begin{align}
        8 \int_0^1 \log (1 - x) \arctan x \,\mathrm{d}x & = 4 \log (2) - \log^2 (2) + \frac 5 2 \zeta(2) - 2 \pi + \pi \log 2 - 8 G \\
        8 \int_0^1 \log (1 + x) \arctan x \,\mathrm{d}x & = 4 \log (2) - \log^2 (2) - \frac 1 2 \zeta(2) - 2 \pi + 3 \pi \log 2.
    \end{align} 
\end{problem_with_solution}

% Interesting Integrals Containing the Inverse Tangent Function and the Logarithmic Function

% SOLUTION
\begin{solution}[do problemu \ref{valean_1_18}]
    Patrz \cite[s. 10, 11]{valean19}.
\end{solution}
% SOLUTION

\begin{problem_with_solution}
    \label{valean_1_20}%
    Niech $G$ oznacza stałą Catalana.
    \begin{align}
        8 \int_0^1 \frac{\log (1 - x) \arctan x}{1+x^2} \,\mathrm{d}x & = \frac 3 4 \log (2) \zeta(2) - \frac 7 8 \zeta(3) - \pi G, \\
        8 \int_0^1 \frac{\log (1 + x) \arctan x}{1+x^2} \,\mathrm{d}x & = \frac 3 4 \log (2) \zeta(2) + \frac {21} 8 \zeta(3) - \pi G,
    \end{align} 
\end{problem_with_solution}

% More Interesting Integrals Involving the Inverse Tangent Function and the Logarithmic Function: The First Part

% SOLUTION
\begin{solution}[do problemu \ref{valean_1_20}]
    Patrz \cite[s. 12]{valean19}.
\end{solution}
% SOLUTION

\begin{problem_with_solution}
    \label{valean_1_21}%
    Niech $G$ oznacza stałą Catalana.
    \begin{align}
        \int_0^1 \frac{\arctan^2 x \log (1 + x)}{1 + x^2} \,\mathrm{d} x = \log 2 \frac {\pi^3}{384} + \frac {21}{256} \pi \zeta(3) - \frac{3}{16} \zeta (2) G.
    \end{align} 
\end{problem_with_solution}

% More Interesting Integrals Involving the Inverse Tangent Function and the Logarithmic Function: The Second Part

% SOLUTION
\begin{solution}[do problemu \ref{valean_1_21}]
    Patrz \cite[s. 12]{valean19}.
\end{solution}
% SOLUTION

\begin{problem_with_solution}
    \label{valean_1_22}%
    Niech $G$ oznacza stałą Catalana.
    \begin{align}
        I & = \int_0^1 \arctan x \log x \left(\log (1-x) - \frac {x}{1-x}\right) \,\mathrm{d} x \\
        & = G - \frac{41}{64} \zeta (3) + \frac{9 \log 2 - 5}{96} \pi^2 + \frac{2 - \log 2}{8} \pi - \frac {\log 2}{2} + \frac{\log^2 (2)}{8}.
    \end{align} 
\end{problem_with_solution}

% Challenging Integrals Involving arctan(x), log(x), log(1−x)

% SOLUTION
\begin{solution}[do problemu \ref{valean_1_22}]
    Patrz \cite[s. 13]{valean19}.
\end{solution}
% SOLUTION


\begin{problem_with_solution}
    \label{valean_1_23}%
    \begin{align}
        I & = \int_0^1 \arctan x \log x \log (1 + x) \,\mathrm{d}x \\
        & = \frac{\log 2}{2} G - \frac{\pi^3}{64} + \frac{15}{64} \zeta(3) - \frac{\pi^2}{96} (3 \log 2 + 1) + \frac{\pi} {8} (4-3 \log 2) + \frac{\log^2(2)}{8} - \log 2.
    \end{align} 
\end{problem_with_solution}

% Challenging Integrals Involving arctan(x), log(x), log(1−x)

% SOLUTION
\begin{solution}[do problemu \ref{valean_1_23}]
    Patrz \cite[s. 13, 14]{valean19}.
\end{solution}
% SOLUTION



\begin{problem_with_solution}
    \label{valean_1_24}%
    Niech $n \ge 1$ będzie naturalne.
    Znaleźć wartość
    \begin{align}
        I_{2n} & := \int_0^1 \frac {\arctan x \log^{2n} (x)}{1 + x } \,\mathrm{d}x
    \end{align} 
    lub, jeśli jest to za trudne, pokazać, że 
    \begin{align}
        I_{1} = \frac{\log 2}{2}G - \frac{\pi^3}{64}.
    \end{align} 
\end{problem_with_solution}

% Challenging Integrals Involving arctan(x), log(x), log(1−x)

% SOLUTION
\begin{solution}[do problemu \ref{valean_1_24}]
    Patrz \cite[s. 14, 15]{valean19}.
\end{solution}
% SOLUTION


\begin{problem_with_solution}
    \label{valean_1_26}%
    \begin{align}
        \int_0^1 \frac{\arctan x}{x} \log \frac{1+x^2}{(1-x)^2} \,\mathrm{d}x = \frac{\pi^3}{16}.
    \end{align} 
\end{problem_with_solution}

% Challenging Integrals Involving arctan(x), log(x), log(1−x)

% SOLUTION
\begin{solution}[do problemu \ref{valean_1_26}]
    Patrz \cite[s. 17]{valean19}.
\end{solution}
% SOLUTION

\begin{problem_with_solution}
    \label{valean_1_32}%
    Znaleźć rekurencję, jaką spełnia
    \begin{align}
        I_n = \int_0^1 \frac{x^n}{(1+x)(1+x^2)^n} \,\mathrm{d}x.
    \end{align} 
\end{problem_with_solution}

% Challenging Integrals Involving arctan(x), log(x), log(1−x)

% SOLUTION
\begin{solution}[do problemu \ref{valean_1_32}]
    Patrz \cite[s. 21, 22]{valean19}.
\end{solution}
% SOLUTION


%