%

\section{Sztuczka Feynmana: różniczkowanie pod znakiem całki}
\section{Sztuczka Feynmana: różniczkowanie pod znakiem całki} % SOLUTION

% https://math.stackexchange.com/questions/942263/really-advanced-techniques-of-integration-definite-or-indefinite
\begin{problem}
    $\int_0^\infty \sin(x) / x \,\mathrm{d}x = \pi/2$.
\end{problem}

% TODO: przepisać całkę z s. 82, Nahin

\begin{problem}
    \label{nahin_holzweg}%
    Niech $a, b > 0$, wtedy
    \begin{equation}
        \int_{-\infty}^\infty \frac{\cos ax}{b^2 - x^4} \,\mathrm{d} x = \frac{\pi}{b} \sin (ab)
    \end{equation}
\end{problem}

\begin{solution}[do problemu \ref{nahin_holzweg}] % SOLUTION
    Nahin \cite[s. 115, 375, 376]{nahin15}. % SOLUTION
\end{solution} % SOLUTION

\begin{problem}
    \label{nahin_datenautobahn}%
    Niech $a > b$, wtedy
    \begin{equation}
        \int_{-\infty}^\infty \frac{\cos ax}{b^4 - x^4} \,\mathrm{d} x = \frac{\pi}{2b^3} [\sin (ab) + \exp (-ab)]
    \end{equation}
\end{problem}

\begin{solution}[do problemu \ref{nahin_datenautobahn}] % SOLUTION
    Nahin \cite[s. 115, 376]{nahin15}. % SOLUTION
\end{solution} % SOLUTION

% Nahin Inside interesting... page 83
\begin{problem}
    \begin{equation}
        \int_0^\infty \frac{\sin ax}{x e^{xy}} \,\mathrm{d}x = \pm \frac \pi 2 - \arctan \frac y a.
    \end{equation}
\end{problem}

\begin{problem}
    Niech $a > 0$.
    Wtedy
    \begin{equation}
        \int_0^\infty \frac{\sin ax}{x} \,\mathrm{d}x = \frac \pi 2.
    \end{equation}
\end{problem}

% Nahin Inside interesting... page 85
\begin{problem}[całka Frullaniego]
    \begin{equation}
        \int_0^\infty \frac{\arctan (ax) - \arctan (bx)}{x} \,\mathrm{d}x = \frac \pi 2 \log \frac a b.
    \end{equation}
\end{problem}

% Nahin Inside interesting... page 8x
\begin{problem}
    Niech $a, b > 0$.
    Wtedy
    \begin{equation}
        \int_0^\infty \frac{e^{-ax} - e^{-bx}}{x} \,\mathrm{d}x = \log \frac b a.
    \end{equation}
\end{problem}

% Nahin Inside interesting... page 89
\begin{problem}
    \begin{equation}
        \int_0^\infty \frac{\cos (ax) - \cos (bx)}{x^2} \,\mathrm{d}x = \frac \pi 2 (b - a).
    \end{equation}
\end{problem}

% Nahin Inside interesting... page 89
\begin{problem}
    \begin{equation}
        \int_0^\infty \frac{\cos (ax) - \cos (bx)}{x} \,\mathrm{d}x = \log \frac b a.
    \end{equation}
\end{problem}

% Nahin Inside interesting... page 89
\begin{problem}
    \label{nahin_kriegsrecht}
    \begin{equation}
        \int_0^\infty \frac{\log (a^2 x^2 + 1)}{x^2 + b^2} \,\mathrm{d}x = \frac \pi b \log (1 + ab)
    \end{equation}
\end{problem}

\begin{solution}[do problemu \ref{nahin_kriegsrecht}] % SOLUTION
    Nahin \cite[s. 67]{nahin15} w szczególnym przypadku $a = b = 1$. % SOLUTION
    Dla dowolnych wartości $a, b$ patrz \cite[s. 114, 375]{nahin15}. % SOLUTION
\end{solution} % SOLUTION

% Nahin Inside interesting... page 91
\begin{problem}
    Niech $a \ge 0$, wtedy
    \begin{equation}
        \int_0^1 \frac{x^a - 1}{\log x} \,\mathrm{d}x = \log(1+a).
    \end{equation}
\end{problem}

% Nahin Inside interesting... page 92
\begin{problem}
    Niech $a \ge 0$, wtedy
    \begin{equation}
        \int_0^1 \frac{x^a - x^b}{\log x} \,\mathrm{d}x = \log \frac{1+a}{1+b}.
    \end{equation}
\end{problem}

% Nahin Inside interesting... page 96
\begin{problem}
    Niech $a > b$, wtedy
    \begin{equation}
        \int_0^\pi \frac{\mathrm{d}x} {a + b \cos x} = \frac{\pi}{\sqrt{a^2 - b^2}}.
    \end{equation}
\end{problem}

Nahin pisze, że poniższą całkę wyznaczył jako pierwszy włoski matematyk Ulisse Dini w 1878 roku i że ma ważne zastosowania w fizyce i inżynierii.
\index[persons]{Dini, Ulisse}%

\begin{problem}
    Niech $a \ge 0$ będzie dowolną liczbą rzeczywistą.
    Wtedy
    \begin{equation}
        \int_0^\pi \log (1 - 2 a \cos x + a^2) \,\mathrm{d} x = \begin{cases}
            0, & \text{gdy } a^2 \le 1, \\
            2 \pi \log a & \textrm{w przeciwnym razie}.
        \end{cases}
    \end{equation}
\end{problem}

\begin{solution} % SOLUTION
    Nahin \cite[s. 109-112]{nahin15} % SOLUTION
\end{solution} % SOLUTION

%