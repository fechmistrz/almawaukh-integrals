%

\section{Całkowanie funkcji wymiernych}
% SOLUTION
\section{Całkowanie funkcji wymiernych}
% SOLUTION

\begin{problem_with_solution}
    \label{banas_12_100}%
\begin{equation}
    \int \frac{\mathrm{d}x}{x^3 + x}.
\end{equation}
\end{problem_with_solution}

% SOLUTION
\textbf{Problem \ref{banas_12_100}} -- wyłączyć z mianownika $x^3$, podstawić $u = 1 + x^{-2}$ i znaleźć całkę z funkcji $1/u$.
Alternatywnie, rozłożyć na ułamki proste i dopiero potem podstawiać.
Wynik to
\begin{equation}
    I_{B100} = \log x - \frac 12 \log (x^2 + 1).
\end{equation}
% SOLUTION

\begin{problem_with_solution}
    \label{banas_12_118}%
\begin{equation}
    \int \frac{3x^3 - 8x + 5}{\sqrt{x^2 - 4x - 7}} \,\mathrm{d}x.
\end{equation}
\end{problem_with_solution}

% SOLUTION
\textbf{Problem \ref{banas_12_118}} -- podstawić $u = x - 2$, a potem $w = \sqrt{11} \sec u$.
To prowadzi do całki, która jest kombinacją liniową potęg sekansa ($\sec, \sec^2, \sec^3, \sec^4$).
Wynik po odwróceniu podstawień to
\begin{equation}
    \sqrt{x^2 - 4x - 7} (x^2 + 5x + 36) + 112 \log \left(2 - x - \sqrt{x^2 - 4x-7}\right).
\end{equation}
% SOLUTION














\begin{problem}
\label{boros_4287}%
\begin{equation}
    \int_0^\infty \frac{x^n \,\mathrm{d}x}{(ax+b)^{m+1}}  = \frac{(-1)^{n+1} (-1-m)! \cdot n!}{a^{n+1} b^{m-n} (n-m)!}
\end{equation}
\end{problem}

% SOLUTION
\textbf{Problem \ref{boros_4287}} -- patrz \cite[s. 48-60]{boros04}
% SOLUTION

\begin{problem}
\label{frac_1_x3_1}%
\begin{equation}
    \int_0^\infty \frac{\mathrm{d}x}{x^3 - 1} = - \frac{\pi}{3\sqrt{3}}
\end{equation}
\end{problem}

% SOLUTION
\textbf{Problem \ref{frac_1_x3_1}} -- patrz \cite[s. 22]{nahin15}
% SOLUTION

\begin{problem}
    \label{experimental_mathematics_p258}%
\begin{equation}
    \int_0^\infty \frac{x^8-4x^6+9x^4-5x^2+1}{x^{12}-10x^{10}+37x^8-42x^6+26x^4-8x^2+1} \,\mathrm{d}x = \frac{\pi}{2}
\end{equation}
\end{problem}

% SOLUTION
\textbf{Problem \ref{experimental_mathematics_p258}} -- patrz \cite[s. 258]{bailey07}.
% SOLUTION

Rozdział 3 Borosa i Molla (TODO), rozdział 6. Całki postaci zostawia jako teren do eksploracji dla Czytelnika, a sami rozważają całki postaci
\begin{equation}
    \int_0^\infty \frac{P(x) \,\mathrm{d}x}{(q_4 x^4 + q_2 x^2 + q_0)^{m+1}},
\end{equation}
gdzie $m \in \mathbb N$, zaś $P$ jest wielomianem stopnia co najwyżej $4m + 2$.
Przykładowe zadania: \ref{boros_moll_7_1_1}, \ref{boros_moll_7_2_3}.

\subsection{Rozkład na ułamki proste}
% SOLUTION
\subsection{Rozkład na ułamki proste}
% SOLUTION

\begin{problem_with_solution}
    \label{boros_moll_7_1_1}
\begin{equation}
    \int_0^\infty \frac{3x^3 \,\mathrm{d}x}{(x^4 + 4x^2 + 1)^5} = \frac{5}{20736} \cdot \left(60 + 7 \sqrt{3} \log \left(7 - 4 \sqrt {3}\right)\right).
\end{equation}
\end{problem_with_solution}

% SOLUTION
\textbf{Problem \ref{boros_moll_7_1_1}} -- patrz \cite[s. 138]{boros04}.
% SOLUTION

%