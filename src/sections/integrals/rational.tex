%

\section{Całkowanie funkcji wymiernych}
% SOLUTION
\section{Całkowanie funkcji wymiernych}
% SOLUTION

\begin{problem}
\label{boros_4287}%
\begin{equation}
    \int_0^\infty \frac{x^n \,\mathrm{d}x}{(ax+b)^{m+1}}  = \frac{(-1)^{n+1} (-1-m)! \cdot n!}{a^{n+1} b^{m-n} (n-m)!}
\end{equation}
\end{problem}

% SOLUTION
\textbf{Problem \ref{boros_4287}} -- patrz \cite[s. 48-60]{boros04}
% SOLUTION

\begin{problem}
\label{frac_1_x3_1}%
\begin{equation}
    \int_0^\infty \frac{\mathrm{d}x}{x^3 - 1} = - \frac{\pi}{3\sqrt{3}}
\end{equation}
\end{problem}

% SOLUTION
\textbf{Problem \ref{frac_1_x3_1}} -- patrz \cite[s. 22]{nahin15}
% SOLUTION

\begin{problem}
    \label{experimental_mathematics_p258}%
\begin{equation}
    \int_0^\infty \frac{x^8-4x^6+9x^4-5x^2+1}{x^{12}-10x^{10}+37x^8-42x^6+26x^4-8x^2+1} \,\mathrm{d}x = \frac{\pi}{2}
\end{equation}
\end{problem}

% SOLUTION
\textbf{Problem \ref{experimental_mathematics_p258}} -- patrz \cite[s. 258]{bailey07}.
% SOLUTION

\subsection{Oszacowania liczby $\pi$}
\begin{problem}
\label{22_7_pi}%
\begin{equation}
    \int_0^1 \frac{x^4(1-x)^4}{1 + x^4} \,\mathrm{d}x = 22/7 - \pi.
\end{equation}
\end{problem}

Czasami używa się określenia całka Dalzella, ponieważ Donald Percy Dalzell \cite{dalzell44} jako pierwszy opublikował to cudo.
Ograniczając mianownik z dołu oraz góry przez $1$ oraz $2$ możemy dojść do wniosku, że
\begin{equation}
    \frac{22}{7} - \frac {1}{630} < \pi < \frac{22}{7} - \frac{1}{1260},
\end{equation}
a więc pomylić się o mniej niż $0.015\%$!

% TODO - na en.wiki jest łatwe w przepisaniu rozwiązanie
% SOLUTION
\textbf{Problem \ref{22_7_pi}} -- patrz \cite[s. 24]{nahin15}
% SOLUTION

% TODO: patrz też https://math.stackexchange.com/questions/1956/is-there-an-integral-that-proves-pi-333-106
Istnieje wiele uogólnień powyższego wyniku do innych przybliżeń liczby $\pi$, dobrym źródłem dalszych informacji, a może nawet inspiracji jest artykuł Lucasa \cite{lucas05}.
Na przykład:

\begin{problem}
\begin{equation}
    \int_0^1 \frac {x^8(1-x)^8 (25+816x^2)}{3164 (1+x^2)} \,\mathrm{d} x = \frac {355}{113} - \pi.
\end{equation}
\end{problem}

albo:

\begin{problem}
\begin{equation}
    \int_0^1 \frac{x^5 ( 1-x)^6 (197 + 462 x^2)}{530 (1+x^2)} \,\mathrm{d}x = \pi - \frac{333}{106}
\end{equation}
\end{problem}


%