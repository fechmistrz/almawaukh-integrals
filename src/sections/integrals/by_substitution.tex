\section{Całkowanie przez podstawianie}
% SOLUTION
\section{Całkowanie przez podstawianie}
% SOLUTION

% Banaś, Wędrychowicz, 12.18.
\begin{problem_with_solution}
    \label{banas_12_18}%
    \begin{equation}
        \int (\arcsin x)^2 \,\mathrm{d}x.
    \end{equation}
\end{problem_with_solution}

% SOLUTION
\textbf{Problem \ref{banas_12_18}}.
Podstawiamy $u = \arcsin x$ i dostajemy całkę z $u^2 \cos u$, którą rozwiązujemy przez części, tak jak w przykładzie \ref{banas_12_14}.
% SOLUTION

% Banaś, Wędrychowicz, 12.19.
\begin{problem_with_solution}
    \label{banas_12_19}%
    \begin{equation}
        \int \sin \log x \, \mathrm{d}x.
    \end{equation}
\end{problem_with_solution}

Analogicznie znajdujemy całkę funkcji $\cos \log x$.

% SOLUTION
\textbf{Problem \ref{banas_12_19}}.
Podstawiamy $u = \log x$, $\mathrm{d} u = \mathrm{d} x / x$, $x = \exp u$ i dostajemy całkę z $e^u \sin u$, którą rozwiązujemy przez części, tak jak w przykładzie \ref{banas_12_19_auxilia}.
% SOLUTION

% Banaś, Wędrychowicz, 12.20.
% \begin{problem}
  %   \label{banas_12_20}%
    % \begin{equation}
      %   \int \cos(\log x) \, \mathrm{d}x.
    % \end{equation}
% \end{problem}

% Banaś, Wędrychowicz, 12.21.
\begin{problem_with_solution}
    \label{banas_12_21}%
    \begin{equation}
        \int \sqrt{x^2 + 1} \, \mathrm{d}x.
    \end{equation}
\end{problem_with_solution}

% SOLUTION
\textbf{Problem \ref{banas_12_21}}.
Podstawiamy $x = \tan \theta$.
Na mocy tożsamości trygonometrycznej $\tan^2 \theta + 1 = \sec^2 \theta$ nasza całka zmienia się w $\int \sec^2 \theta \cdot \sec \theta \,\mathrm{d}\theta$, czyli całkę z problemu \ref{banas_12_21_auxilia}.
Zatem
\begin{align}
    \int \sqrt{x^2 + 1} \, \mathrm{d}x & = \frac 12 \sec \theta \tan \theta + \log |\tan \theta + \sec \theta| \\
    & = \frac 1 2 x \sqrt{x^2 + 1} + \log \left|x + \sqrt{x^2+1}\right|.
\end{align}
% SOLUTION

% Banaś, Wędrychowicz, 12.30.
\begin{problem_with_solution}
    \label{banas_12_30}%
    \begin{equation}
        \int x \sqrt{a^2 - x^2} \, \mathrm{d}x.
    \end{equation}
\end{problem_with_solution}

% SOLUTION
\textbf{Problem \ref{banas_12_30}}.
Podstawić $u = a^2 - x^2$.
% SOLUTION

% Banaś, Wędrychowicz, 12.31.
\begin{problem_with_solution}
    \label{banas_12_31}%
    \begin{equation}
        \int \exp \sqrt x \, \mathrm{d}x.
    \end{equation}
\end{problem_with_solution}

% SOLUTION
\textbf{Problem \ref{banas_12_31}}.
Podstawić $u = \sqrt x$, następnie scałkować przez części: $f(u) = u$, $g'(u) = \exp u$.
% SOLUTION

% Banaś, Wędrychowicz, 12.33.
\begin{problem_with_solution}
    \label{banas_12_33}%
    \begin{equation}
        \int \tan x \, \mathrm{d}x.
    \end{equation}
\end{problem_with_solution}

Analogicznie znajdujemy całkę funkcji $\cot x$.

% SOLUTION
\textbf{Problem \ref{banas_12_33}}.
Podstawić $u = \sin x$.
% SOLUTION

% Banaś, Wędrychowicz, 12.34.
\begin{problem_with_solution}
    \label{banas_12_34}%
    \begin{equation}
        \int \frac{\mathrm{d}x}{x \log x}.
    \end{equation}
\end{problem_with_solution}

% SOLUTION
\textbf{Problem \ref{banas_12_34}}.
Podstawić $u = \log x$.
% SOLUTION

% Banaś, Wędrychowicz, 12.40.
\begin{problem_with_solution}
    \label{banas_12_40}%
    \begin{equation}
        \int \frac{x^3}{1+x^8} \, \mathrm{d}x.
    \end{equation}
\end{problem_with_solution}

% SOLUTION
\textbf{Problem \ref{banas_12_40}}.
Podstawiamy $u = x^4$ i dostajemy całkę funkcji $(1+x^2)^{-1}$, czyli ...
% TODO: dodać link jak już będzie gdzieś ta całka policzona
% SOLUTION

% Banaś, Wędrychowicz, 12.XX.
\begin{problem_with_solution}
    \label{banas_12_XX}%
    \begin{equation}
        \int \ldots \, \mathrm{d}x.
    \end{equation}
\end{problem_with_solution}

% SOLUTION
\textbf{Problem \ref{banas_12_XX}}.
% SOLUTION

% Banaś, Wędrychowicz, 12.XX.
\begin{problem_with_solution}
    \label{banas_12_XX}%
    \begin{equation}
        \int \ldots \, \mathrm{d}x.
    \end{equation}
\end{problem_with_solution}

% SOLUTION
\textbf{Problem \ref{banas_12_XX}}.
% SOLUTION

% Banaś, Wędrychowicz, 12.XX.
\begin{problem_with_solution}
    \label{banas_12_XX}%
    \begin{equation}
        \int \ldots \, \mathrm{d}x.
    \end{equation}
\end{problem_with_solution}

% SOLUTION
\textbf{Problem \ref{banas_12_XX}}.
% SOLUTION

% Banaś, Wędrychowicz, 12.XX.
\begin{problem_with_solution}
    \label{banas_12_XX}%
    \begin{equation}
        \int \ldots \, \mathrm{d}x.
    \end{equation}
\end{problem_with_solution}

% SOLUTION
\textbf{Problem \ref{banas_12_XX}}.
% SOLUTION

% Banaś, Wędrychowicz, 12.XX.
\begin{problem_with_solution}
    \label{banas_12_XX}%
    \begin{equation}
        \int \ldots \, \mathrm{d}x.
    \end{equation}
\end{problem_with_solution}

% SOLUTION
\textbf{Problem \ref{banas_12_XX}}.
% SOLUTION

% Banaś, Wędrychowicz, 12.XX.
\begin{problem_with_solution}
    \label{banas_12_XX}%
    \begin{equation}
        \int \ldots \, \mathrm{d}x.
    \end{equation}
\end{problem_with_solution}

% SOLUTION
\textbf{Problem \ref{banas_12_XX}}.
% SOLUTION

% Banaś, Wędrychowicz, 12.XX.
\begin{problem_with_solution}
    \label{banas_12_XX}%
    \begin{equation}
        \int \ldots \, \mathrm{d}x.
    \end{equation}
\end{problem_with_solution}

% SOLUTION
\textbf{Problem \ref{banas_12_XX}}.
% SOLUTION

% Banaś, Wędrychowicz, 12.XX.
\begin{problem_with_solution}
    \label{banas_12_XX}%
    \begin{equation}
        \int \ldots \, \mathrm{d}x.
    \end{equation}
\end{problem_with_solution}

% SOLUTION
\textbf{Problem \ref{banas_12_XX}}.
% SOLUTION

% Banaś, Wędrychowicz, 12.XX.
\begin{problem_with_solution}
    \label{banas_12_XX}%
    \begin{equation}
        \int \ldots \, \mathrm{d}x.
    \end{equation}
\end{problem_with_solution}

% SOLUTION
\textbf{Problem \ref{banas_12_XX}}.
% SOLUTION

\begin{problem_with_solution}
    \label{nahin_1x_1x}%
    \begin{equation}
        \int_{-1}^1 \sqrt{\frac{1+x}{1-x}} \,\mathrm{d}x = \pi.
    \end{equation}
\end{problem_with_solution}

Znamy tę całkę z książki Nahina \cite{nahin15}.

% SOLUTION
\textbf{Problem \ref{nahin_1x_1x}} -- \cite[s. 115, 378]{nahin15}.
Wskazówka: podstawić $x = \cos 2 \varphi$.
% SOLUTION

\subsection{Podstawienia Eulera}

TODO: Banaś Wędrychowicz, 12.71 - 12.87

\begin{problem}
    Banaś-Wędrychowicz, 12.58.
\end{problem}

%