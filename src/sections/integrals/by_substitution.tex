\section{Całkowanie przez podstawianie}
% SOLUTION
\section{Całkowanie przez podstawianie}
% SOLUTION


% podstawienia trygonometryczne start

% Banaś, Wędrychowicz, 12.21.
\begin{problem_with_solution}
    \label{banas_12_21}%
    \begin{equation}
        \int \sqrt{x^2 + 1} \, \mathrm{d}x.
    \end{equation}
\end{problem_with_solution}

% SOLUTION
\textbf{Problem \ref{banas_12_21}}.
Podstawiamy $x = \tan \theta$.
Na mocy tożsamości trygonometrycznej $\tan^2 \theta + 1 = \sec^2 \theta$ nasza całka zmienia się w $\int \sec^2 \theta \cdot \sec \theta \,\mathrm{d}\theta$, czyli całkę z problemu \ref{banas_12_21_auxilia}.
Zatem
\begin{align}
    \int \sqrt{x^2 + 1} \, \mathrm{d}x & = \frac 12 \sec \theta \tan \theta + \log |\tan \theta + \sec \theta| \\
    & = \frac 1 2 x \sqrt{x^2 + 1} + \log \left|x + \sqrt{x^2+1}\right|.
\end{align}
% SOLUTION

% Banaś, Wędrychowicz, 12.33.
\begin{problem_with_solution}
    \label{banas_12_33}%
    \begin{equation}
        \int \tan x \, \mathrm{d}x.
    \end{equation}
\end{problem_with_solution}

% SOLUTION
\textbf{Problem \ref{banas_12_33}}.
Podstawiamy $u = \sin x$ i dostajemy całkę $- \int \mathrm{d}u / u$. 
% SOLUTION

Analogicznie:

% Banaś, Wędrychowicz, 12.32.
\begin{problem}
    \label{banas_12_32}%
    \begin{equation}
        \int \cot x \, \mathrm{d}x.
    \end{equation}
\end{problem}

% Banaś, Wędrychowicz, 12.36.
\begin{problem_with_solution}
    \label{banas_12_36}%
    \begin{equation}
        \int \frac{(\log x)^5}{x} \, \mathrm{d}x.
    \end{equation}
\end{problem_with_solution}

% SOLUTION
\textbf{Problem \ref{banas_12_36}}.
Podstawiamy $u = \log x$ i dostajemy całkę $\int u^5 \mathrm{d}u$, czyli $\frac 1 6 (\log x)^6$. 
% SOLUTION

% Banaś, Wędrychowicz, 12.40.
\begin{problem_with_solution}
    \label{banas_12_40}%
    \begin{equation}
        \int \frac{x^3}{1+x^8} \, \mathrm{d}x.
    \end{equation}
\end{problem_with_solution}

% SOLUTION
\textbf{Problem \ref{banas_12_40}}.
Podstawiamy $u = x^4$ i dostajemy całkę funkcji $(1+x^2)^{-1}$, czyli ...
% TODO: dodać link jak już będzie gdzieś ta całka policzona
% SOLUTION

\begin{problem_with_solution}
    \label{nahin_1x_1x}%
    \begin{equation}
        \int_{-1}^1 \sqrt{\frac{1+x}{1-x}} \,\mathrm{d}x = \pi.
    \end{equation}
\end{problem_with_solution}

Znamy tę całkę z książki Nahina \cite{nahin15}.

% SOLUTION
\textbf{Problem \ref{nahin_1x_1x}} -- \cite[s. 115, 378]{nahin15}.
Wskazówka: podstawić $x = \cos 2 \varphi$.
% SOLUTION

% podstawienia trygonometryczne koniec

% Banaś, Wędrychowicz, 12.18.
\begin{problem_with_solution}
    \label{banas_12_18}%
    \begin{equation}
        \int (\arcsin x)^2 \,\mathrm{d}x.
    \end{equation}
\end{problem_with_solution}

% SOLUTION
\textbf{Problem \ref{banas_12_18}}.
Podstawiamy $u = \arcsin x$ i dostajemy całkę z $u^2 \cos u$, którą rozwiązujemy przez części, tak jak w przykładzie \ref{banas_12_14}.
% SOLUTION

% Banaś, Wędrychowicz, 12.19.
\begin{problem_with_solution}
    \label{banas_12_19}%
    \begin{equation}
        \int \sin(\log x) \, \mathrm{d}x.
    \end{equation}
\end{problem_with_solution}

% SOLUTION
\textbf{Problem \ref{banas_12_19}}.
Podstawiamy $u = \log x$, $\mathrm{d} u = \mathrm{d} x / x$, $x = \exp u$ i dostajemy całkę z $e^u \sin u$, którą rozwiązujemy przez części, tak jak w przykładzie \ref{banas_12_19_auxilia}.
% SOLUTION

% Banaś, Wędrychowicz, 12.20.
\begin{problem}
    \label{banas_12_20}%
    \begin{equation}
        \int \cos(\log x) \, \mathrm{d}x.
    \end{equation}
\end{problem}    

% Banaś, Wędrychowicz: 12.29
\begin{problem}
    Banaś-Wędrychowicz, 12.29.
\end{problem}

% Banaś, Wędrychowicz: 12.30
\begin{problem}
    Banaś-Wędrychowicz, 12.30.
\end{problem}

% Banaś, Wędrychowicz: 12.31
\begin{problem}
    Banaś-Wędrychowicz, 12.31.
\end{problem}

% Banaś, Wędrychowicz: 12.32
\begin{problem}
    Banaś-Wędrychowicz, 12.32.
\end{problem}

% Banaś, Wędrychowicz: 12.33
\begin{problem}
    Banaś-Wędrychowicz, 12.33.
\end{problem}

% Banaś, Wędrychowicz: 12.34
\begin{problem}
    Banaś-Wędrychowicz, 12.34.
\end{problem}

% Banaś, Wędrychowicz: 12.35
\begin{problem}
    Banaś-Wędrychowicz, 12.35.
\end{problem}

% Banaś, Wędrychowicz: 12.36
\begin{problem}
    Banaś-Wędrychowicz, 12.36.
\end{problem}

% Banaś, Wędrychowicz: 12.37
\begin{problem}
    Banaś-Wędrychowicz, 12.37.
\end{problem}

% Banaś, Wędrychowicz: 12.38
\begin{problem}
    Banaś-Wędrychowicz, 12.38.
\end{problem}

% Banaś, Wędrychowicz: 12.39
\begin{problem}
    Banaś-Wędrychowicz, 12.39.
\end{problem}

% Banaś, Wędrychowicz: 12.40
\begin{problem}
    Banaś-Wędrychowicz, 12.40.
\end{problem}

% Banaś, Wędrychowicz: 12.41
\begin{problem}
    Banaś-Wędrychowicz, 12.41.
\end{problem}

% Banaś, Wędrychowicz: 12.42
\begin{problem}
    Banaś-Wędrychowicz, 12.42.
\end{problem}

% Banaś, Wędrychowicz: 12.43
\begin{problem}
    Banaś-Wędrychowicz, 12.43.
\end{problem}

% Banaś, Wędrychowicz: 12.44
\begin{problem}
    Banaś-Wędrychowicz, 12.44.
\end{problem}

% Banaś, Wędrychowicz: 12.45
\begin{problem}
    Banaś-Wędrychowicz, 12.45.
\end{problem}

% Banaś, Wędrychowicz: 12.46
\begin{problem}
    Banaś-Wędrychowicz, 12.46.
\end{problem}

% Banaś, Wędrychowicz: 12.47
\begin{problem}
    Banaś-Wędrychowicz, 12.47.
\end{problem}

% Banaś, Wędrychowicz: 12.48
\begin{problem}
    Banaś-Wędrychowicz, 12.48.
\end{problem}

% Banaś, Wędrychowicz: 12.49
\begin{problem}
    Banaś-Wędrychowicz, 12.49.
\end{problem}

% Banaś, Wędrychowicz: 12.50
\begin{problem}
    Banaś-Wędrychowicz, 12.50.
\end{problem}

% Banaś, Wędrychowicz: 12.51
\begin{problem}
    Banaś-Wędrychowicz, 12.51.
\end{problem}

% Banaś, Wędrychowicz: 12.52
\begin{problem}
    Banaś-Wędrychowicz, 12.52.
\end{problem}

% Banaś, Wędrychowicz: 12.53
\begin{problem}
    Banaś-Wędrychowicz, 12.53.
\end{problem}

% Banaś, Wędrychowicz: 12.54
\begin{problem}
    Banaś-Wędrychowicz, 12.54.
\end{problem}

% Banaś, Wędrychowicz: 12.55
\begin{problem}
    Banaś-Wędrychowicz, 12.55.
\end{problem}

% Banaś, Wędrychowicz: 12.56
\begin{problem}
    Banaś-Wędrychowicz, 12.56.
\end{problem}

% Banaś, Wędrychowicz: 12.57
\begin{problem}
    Banaś-Wędrychowicz, 12.57.
\end{problem}

% Banaś, Wędrychowicz: 12.58
\begin{problem}
    Banaś-Wędrychowicz, 12.58.
\end{problem}

\subsection{Podstawienia Eulera}

TODO: Banaś Wędrychowicz, 12.71 - 12.87

\begin{problem}
    Banaś-Wędrychowicz, 12.58.
\end{problem}

%