%

\begin{problem_with_solution}
    \label{reuleaux_tetrahedron}%
    Czworościan Reuleaux to bryła będąca częścią wspólną czterech kul, których środki leżą w wierzchołkach czworościanu foremnego, a promienie są tej samej długości, co krawędzie tego czworościanu.
    Znaleźć objętość tej bryły,
    \begin{equation}
        V = \int_0^1
        \frac{
            8\sqrt{3}
        }{
            1 + 3t^2
        } - \frac{
            16 \sqrt{2} (3t+1) (4t^2 +t+1)^{3/2}
        }{
            (3t^2+1)(11t^2 + 2t + 3)^2
        } - \frac{
            \sqrt{2} (249 t^2 + 54t + 65)
        }{
            (11t^2 + 2t +3)^2
        } \,\mathrm{d} t.
    \end{equation}
\end{problem_with_solution}

% SOLUTION
\textbf{Problem \ref{reuleaux_tetrahedron}} -- patrz \url{https://mathworld.wolfram.com/ReuleauxTetrahedron.html}.
% SOLUTION

% TODO: https://mathworld.wolfram.com/images/gifs/FoxTrotMathTest.jpg
% TODO https://mathworld.wolfram.com/DefiniteIntegral.html

%