
% https://math.stackexchange.com/a/9292/1298830
\begin{problem}
    $\int_{-\infty}^\infty \exp(-x^2) \,\mathrm{d}x = \sqrt x$.
\end{problem}

\begin{problem}
    \label{reuleaux_tetrahedron}%
    Czworościan Reuleaux to bryła będąca częścią wspólną czterech kul, których środki leżą w wierzchołkach czworościanu foremnego, a promienie są tej samej długości, co krawędzie tego czworościanu.
    Znaleźć objętość tej bryły,
    \begin{equation}
        V = \int_0^1
        \frac{
            8\sqrt{3}
        }{
            1 + 3t^2
        } - \frac{
            16 \sqrt{2} (3t+1) (4t^2 +t+1)^{3/2}
        }{
            (3t^2+1)(11t^2 + 2t + 3)^2
        } - \frac{
            \sqrt{2} (249 t^2 + 54t + 65)
        }{
            (11t^2 + 2t +3)^2
        } \,\mathrm{d} t.
    \end{equation}
\end{problem}

\begin{solution}[do problemu \ref{reuleaux_tetrahedron}] % SOLUTION
    Patrz przypis\footnote{\url{https://mathworld.wolfram.com/ReuleauxTetrahedron.html}}. % SOLUTION
\end{solution} % SOLUTION

% https://math.stackexchange.com/questions/580521/generalizing-int-01-frac-arctan-sqrtx2-2-sqrtx2-2
\begin{problem}[całka Ahmeda]
    \label{ahmed_integral}%
    \begin{equation}
        \int_0^1 \frac{\arctan \sqrt{x^2+2}}{(x^2+1) \sqrt{x^2+2}} \,\mathrm{d}x = \frac{5\pi^2}{96}.
    \end{equation}
\end{problem}

\begin{solution}[do problemu \ref{ahmed_integral}] % SOLUTION
    Patrz \cite{ahmed02}. % SOLUTION
\end{solution} % SOLUTION

% TODO: https://mathworld.wolfram.com/images/gifs/FoxTrotMathTest.jpg
% TODO https://mathworld.wolfram.com/DefiniteIntegral.html

%