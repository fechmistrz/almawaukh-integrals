
\section{Całki różne, różniste}
% SOLUTION
\section{Całki różne, różniste}
% SOLUTION

% TODO: https://math.stackexchange.com/questions/9286/evaluation-of-gaussian-integral-int-0-infty-mathrme-x2-dx
% https://math.stackexchange.com/a/9292/1298830
\begin{problem}[całka Gaußa, całka Eulera-Poissona]
    \label{gauss_euler_poisson}%
    \begin{equation}
        I = \int_{-\infty}^\infty \exp \left( -x^2 \right) \,\mathrm{d} x = \sqrt{\pi}.
    \end{equation}
\end{problem}

Pierwszą osobą, o której wiemy, że myślała o podobnych całkach był Abraham de Moivre (około 1733 roku), ale dopiero Gauß (1809) podał dokładny wynik i wspomniał, że to zasługa Laplace'a.
Całkę można wyznaczyć na wiele różnych sposobów, przy czym zazwyczaj zaczyna się od metody, która pochodzi jeszcze od Poissona: całkowana funkcja jest nieujemna, więc można policzyć kwadrat całki (i spierwiastkować wynik), przechodząc po drodze na współrzędne biegunowe.
(Co ciekawe, sposób ten działa zasadniczo z tylko jedną cąłką -- tą...).

% SOLUTION
\textbf{Problem \ref{gauss_euler_poisson}} -- zgodnie ze wskazówką, wyznaczamy kwadrat całki
\begin{align}
    I^2 & = \left(\int_{-\infty}^\infty \exp \left( -x^2 \right) \,\mathrm{d}x\right)\left(\int_{-\infty}^\infty \exp \left( -y^2 \right) \,\mathrm{d}y\right) \\
    & = \int_{-\infty}^\infty \int_{-\infty}^\infty \exp - \left(x^2+y^2\right) \,\mathrm{d}x \mathrm{d}y \\
    & = \int_0^{2\pi} \int_0^\infty r \exp (-r^2) \,\mathrm{d}r \mathrm{d}\theta \\
    & = 2\pi \int_0^\infty r \exp (-r^2) \,\mathrm{d} r \\
    & = 2\pi \int^0_{-\infty} \frac 1 2 \exp s \,\mathrm{d} s \\
    & = \pi.
\end{align}

% https://math.stackexchange.com/questions/34767/int-infty-infty-e-x2-dx-with-complex-analysis
Inne rozwiązanie zaczyna się od podstawienia $u = x^2$, $\mathrm{d} u = 2x \,\mathrm{d}x$.
Wtedy na mocy wzoru odbiciowego Eulera $\Gamma (z) \Gamma(1-z) = \pi/\sin \pi z$ mamy:
\begin{align}
	I = \int_0^\infty u^{-1/2} e^{-u}\,\mathrm{d}u = \Gamma \left(\frac 12\right) = \sqrt{\pi}.
\end{align}
% SOLUTION

\begin{problem}
    \begin{equation}
        \int \sec x \,\mathrm{d} x = \log| \sec x + \tan x|.
    \end{equation}
\end{problem}

% https://en.wikipedia.org/wiki/Integral_of_the_secant_function#History
W 1599 roku Edward Wright znalazł wartość tej całki metodami numerycznymi (było mu to potrzebne by dokładnie skonstruować odwzorowanie walcowe wiernokątne -- Mercatora).
\index{odwzorowanie Mercatora}%
\index[persons]{Wright, Edward}%
W latach czterdziestych XVI wieku Henry Bond porównał wyniki Wrighta z tablicami logarytmicznymi i postawił hipotezę co do zwartej postaci całki.
\index[persons]{Bond, Henry}%
Pierwsze rozwiązanie podał szkocki matematyk i astronom James Gregory w pracy \emph{,,Exercitationes Geometricae''} z 1668 roku, ale było tak trudne w zrozumieniu (chociaż poprawne; podstawił $u = \sec \theta + \tan \theta$), że angielski teolog i matematyk Isaac Barrow zaproponował w \emph{,,Lectiones Geometricae''} z 1670 roku zupełnie inne podejście.
\index[persons]{Gregory, James}%
\index[persons]{Barrow, Isaac}%
Dowód Barrowa jest znany przede wszystkim dlatego, że zawiera najstarszy znany rozkład na ułamki proste podczas całkowania (!).
\index{rozkład na ułamki proste}%

\begin{problem}[sinus całkowy]
    \begin{equation}
        \int_0^\infty \frac {\sin x}{x} \,\mathrm{d} x = \frac \pi 2.
    \end{equation}
\end{problem}

% TODO: https://math.stackexchange.com/questions/13344/proof-of-int-0-infty-left-frac-sin-xx-right2-mathrm-dx-frac-pi2
\begin{problem}
    \begin{equation}
        \int_0^\infty \left(\frac {\sin x}{x}\right)^2 \,\mathrm{d} x = \frac \pi 2.
    \end{equation}
\end{problem}