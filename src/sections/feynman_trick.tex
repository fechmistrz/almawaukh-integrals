%

\section{Różniczkowanie pod znakiem całki}

% Nahin Inside interesting... page 83
\begin{integral}
    \begin{equation}
        \int_0^\infty \frac{\sin ax}{x e^{xy}} \,\mathrm{d}x = \pm \frac \pi 2 - \arctan \frac y a.
    \end{equation}
\end{integral}

\begin{integral}
    Niech $a > 0$.
    Wtedy
    \begin{equation}
        \int_0^\infty \frac{\sin ax}{x} \,\mathrm{d}x = \frac \pi 2.
    \end{equation}
\end{integral}

% Nahin Inside interesting... page 85
\begin{integral}[całka Frullaniego]
    \begin{equation}
        \int_0^\infty \frac{\arctan (ax) - \arctan (bx)}{x} \,\mathrm{d}x = \frac \pi 2 \log \frac a b.
    \end{equation}
\end{integral}

% Nahin Inside interesting... page 8x
\begin{integral}
    Niech $a, b > 0$.
    Wtedy
    \begin{equation}
        \int_0^\infty \frac{e^{-ax} - e^{-bx}}{x} \,\mathrm{d}x = \log \frac b a.
    \end{equation}
\end{integral}

% Nahin Inside interesting... page 89
\begin{integral}
    \begin{equation}
        \int_0^\infty \frac{\cos (ax) - \cos (bx)}{x^2} \,\mathrm{d}x = \frac \pi 2 (b - a).
    \end{equation}
\end{integral}

% Nahin Inside interesting... page 89
\begin{integral}
    \begin{equation}
        \int_0^\infty \frac{\cos (ax) - \cos (bx)}{x} \,\mathrm{d}x = \log \frac b a.
    \end{equation}
\end{integral}

% Nahin Inside interesting... page 91
\begin{integral}
    Niech $a \ge 0$, wtedy
    \begin{equation}
        \int_0^1 \frac{x^a - 1}{\log x} \,\mathrm{d}x = \log(1+a).
    \end{equation}
\end{integral}

% Nahin Inside interesting... page 92
\begin{integral}
    Niech $a \ge 0$, wtedy
    \begin{equation}
        \int_0^1 \frac{x^a - x^b}{\log x} \,\mathrm{d}x = \log \frac{1+a}{1+b}.
    \end{equation}
\end{integral}

% Nahin Inside interesting... page 96
\begin{integral}
    Niech $a > b$, wtedy
    \begin{equation}
        \int_0^\pi \frac{\mathrm{d}x} {a + b \cos x} = \frac{\pi}{\sqrt{a^2 - b^2}}.
    \end{equation}
\end{integral}

%