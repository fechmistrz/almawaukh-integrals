%

\section{Całkowanie funkcji wymiernych}
Całkowanie funkcji wymiernych

\begin{integral}
\begin{equation}
    \int_0^\infty \frac{x^n \,\mathrm{d}x}{(ax+b)^{m+1}}  = \frac{(-1)^{n+1} (-1-m)! \cdot n!}{a^{n+1} b^{m-n} (n-m)!}
\end{equation}
\end{integral}

\begin{solution}
    Rozdział 3 Borosa, strony 48-60.
\end{solution}

\begin{integral}
\begin{equation}
    \int_0^\infty \frac{\mathrm{d}x}{x^3 - 1} = - \frac{\pi}{3\sqrt{3}}
\end{equation}
\end{integral}

\begin{solution}
    Strona 22 w Inside Interesting Integrals.
\end{solution}


Całka Donalda Percy'ego Dalzella:
\begin{integral}
\begin{equation}
    \int_0^1 \frac{x^4(1-x)^4}{1 + x^4} \,\mathrm{d}x = 22/7 - \pi
\end{equation}
\end{integral}

\begin{solution}
    Strona 24 w Inside Interesting Integrals.
\end{solution}


\begin{integral}
\begin{equation}
    \int_0^\infty \frac{x^8-4x^6+9x^4-5x^2+1}{x^{12}-10x^{10}+37x^8-42x^6+26x^4-8x^2+1} \,\mathrm{d}x = \frac{\pi}{2}
\end{equation}
\end{integral}

\begin{solution}
    Strona 258 w: Bailey, D. H.; Borwein, J. M.; Calkin, N. J.; Girgensohn, R.; Luke, D. R.; and Moll, V. H. Experimental Mathematics in Action. Wellesley, MA: A K Peters, 2007.
\end{solution}

%