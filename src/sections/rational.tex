%

\section{Całkowanie funkcji wymiernych}
Całkowanie funkcji wymiernych

\begin{problem}
\begin{equation}
    \int_0^\infty \frac{x^n \,\mathrm{d}x}{(ax+b)^{m+1}}  = \frac{(-1)^{n+1} (-1-m)! \cdot n!}{a^{n+1} b^{m-n} (n-m)!}
\end{equation}
\end{problem}

\begin{solution}
    Rozdział 3 Borosa, strony 48-60.
\end{solution}

\begin{problem}
\begin{equation}
    \int_0^\infty \frac{\mathrm{d}x}{x^3 - 1} = - \frac{\pi}{3\sqrt{3}}
\end{equation}
\end{problem}

\begin{solution}
    Strona 22 w Inside Interesting Integrals.
\end{solution}


Całka Donalda Percy'ego Dalzella:
% TODO: % The author of the 1944 paper that first published this gem was D. P. Dalzell, a Donald Percy Dalzell $$ \int x^4 (1-x)^4 / (1+x^2) = 22/7 - pi$$.  Dalzell, D. P. (1944), "On 22/7", Journal of the London Mathematical Society, 19 (75 Part 3): 133–134, doi:10.1112/jlms/19.75_part_3.133, MR 0013425, Zbl 0060.15306.

\begin{problem}
\begin{equation}
    \int_0^1 \frac{x^4(1-x)^4}{1 + x^4} \,\mathrm{d}x = 22/7 - \pi.
\end{equation}
\end{problem}

\begin{solution}
    Strona 24 w Inside Interesting Integrals.
\end{solution}

Ograniczając mianownik z dołu oraz góry przez $1$ oraz $2$ możemy dojść do wniosku, że $22/7 - 1/630 < \pi < 22/7 - 1/1260$, a więc pomylić się o mniej niż $0.015\%$!
% TODO: Lucas, Stephen (2005), "Integral proofs that 355/113 > π" (PDF), Australian Mathematical Society Gazette, 32 (4): 263–266, MR 2176249, Zbl 1181.11077
Od Stephena Lucasa wiemy, że istnieją podobne wzory dla innych przybliżeń liczby $\pi$.
Na przykład:
\begin{problem}
\begin{equation}
    \int_0^1 \frac {x^8(1-x)^8 (25+816x^2)}{3164 (1+x^2)} \,\mathrm{d} x = \frac {355}{113} - \pi.
\end{equation}
\end{problem}

albo:

\begin{problem}
\begin{equation}
    \int_0^1 \frac{x^5 ( 1-x)^6 (197 + 462 x^2)}{530 (1+x^2)} \,\mathrm{d}x = \pi - \frac{333}{106}
\end{equation}
\end{problem}

% TODO: patrz też https://math.stackexchange.com/questions/1956/is-there-an-integral-that-proves-pi-333-106

\begin{problem}
\begin{equation}
    \int_0^\infty \frac{x^8-4x^6+9x^4-5x^2+1}{x^{12}-10x^{10}+37x^8-42x^6+26x^4-8x^2+1} \,\mathrm{d}x = \frac{\pi}{2}
\end{equation}
\end{problem}

\begin{solution}
    Strona 258 w: Bailey, D. H.; Borwein, J. M.; Calkin, N. J.; Girgensohn, R.; Luke, D. R.; and Moll, V. H. Experimental Mathematics in Action. Wellesley, MA: A K Peters, 2007.
\end{solution}

%