%

\section{Całkowanie przez części}

\begin{proposition}[wzór na całkowanie przez części]
\label{prp_int_by_parts}%
    Jeśli funkcje $f, g \colon I \to \R$ są różniczkowalne, to
    \begin{equation}
        \int f(x) g'(x) \,\mathrm{d}x = f(x) g(x) - \int f'(x) g(x) \,\mathrm{d} x.
    \end{equation}
\end{proposition}

\begin{proof}
    Całkujemy obie strony wzoru na pochodną iloczynu $(fg)' = fg' + f'g$, a następnie porządkujemy strony równości.
\end{proof}

\begin{integral}
    $\int x \sin x \,\mathrm{d} x = \ldots$
\end{integral}

\begin{proof}
    Całkujemy przez części, $f(x) = x$, $g'(x) = \sin x$.
    \begin{align}
        \int x \sin x \,\mathrm{d} x & = -x \cos x - \int - \cos x \, \mathrm{d}x \\
                                     & = -x \cos x + \sin x.
    \end{align}
\end{proof}

Analogicznie obliczamy całki:

\begin{multicols}{2}
\begin{integral}
    $\int x \cos x \,\mathrm{d} x = \ldots$
\end{integral}

\begin{integral}
    $\int x \exp x \,\mathrm{d} x = \ldots$
\end{integral}
\end{multicols}

\begin{integral}
    $\int x \arctan x \,\mathrm{d} x = \ldots$
\end{integral}

\begin{proof}
    Całkujemy przez części, $f(x) = \arctan x$, $g'(x) = x$.
    \begin{align}
        \int x \arctan x \, \mathrm{d} x & = \frac 12 x^2 \arctan x - \int \frac{x^2 \,\mathrm{d}x}{2(x^2+1)} \\
                                         & = \frac 12 x^2 \arctan x - \frac 12 \left(\int 1 \,\mathrm{d}x - \int \frac{\mathrm{d}x}{x^2+1} \right) \\
                                         & = \frac 12 x^2 \arctan x - \frac 12 \left(x - \arctan x \right) \\
                                         & = \frac 12 \left((x^2+1)\arctan x - x \right).
    \end{align}
\end{proof}

\begin{integral}
    $\int x^n \log x \,\mathrm{d} x = \ldots$, gdzie $n \in \N$.
\end{integral}

\begin{proof}
    Całkujemy przez części, $f(x) = \log x$, $g'(x) = x^n$.
    \begin{align}
        \int x^n \log x \, \mathrm{d} x & = \frac{x^{n+1} \log x}{n+1} - \int \frac{x^n \,\mathrm{d} x}{n+1} \\
                                        & = \frac{x^{n+1} \log x}{n+1} - \frac{x^{n+1}}{(n+1)^2}.
    \end{align}
\end{proof}

%