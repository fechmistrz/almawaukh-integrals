\section{Całkowanie przez podstawianie}

% Banaś, Wędrychowicz, 12.18.
\begin{integral}
    $\int (\arcsin x)^2 \,\mathrm{d}x$.
\end{integral}

\begin{solution}
    Podstawiamy $u = \arcsin x$ i dostajemy całkę z $u^2 \cos u$, którą rozwiązujemy przez części, tak jak w przykładzie \ref{banas_12_14}.
\end{solution}

% Banaś, Wędrychowicz: 12.29
\begin{integral}
    Banaś-Wędrychowicz, 12.29.
\end{integral}

% Banaś, Wędrychowicz: 12.30
\begin{integral}
    Banaś-Wędrychowicz, 12.30.
\end{integral}

% Banaś, Wędrychowicz: 12.31
\begin{integral}
    Banaś-Wędrychowicz, 12.31.
\end{integral}

% Banaś, Wędrychowicz: 12.32
\begin{integral}
    Banaś-Wędrychowicz, 12.32.
\end{integral}

% Banaś, Wędrychowicz: 12.33
\begin{integral}
    Banaś-Wędrychowicz, 12.33.
\end{integral}

% Banaś, Wędrychowicz: 12.34
\begin{integral}
    Banaś-Wędrychowicz, 12.34.
\end{integral}

% Banaś, Wędrychowicz: 12.35
\begin{integral}
    Banaś-Wędrychowicz, 12.35.
\end{integral}

% Banaś, Wędrychowicz: 12.36
\begin{integral}
    Banaś-Wędrychowicz, 12.36.
\end{integral}

% Banaś, Wędrychowicz: 12.37
\begin{integral}
    Banaś-Wędrychowicz, 12.37.
\end{integral}

% Banaś, Wędrychowicz: 12.38
\begin{integral}
    Banaś-Wędrychowicz, 12.38.
\end{integral}

% Banaś, Wędrychowicz: 12.39
\begin{integral}
    Banaś-Wędrychowicz, 12.39.
\end{integral}

% Banaś, Wędrychowicz: 12.40
\begin{integral}
    Banaś-Wędrychowicz, 12.40.
\end{integral}

% Banaś, Wędrychowicz: 12.41
\begin{integral}
    Banaś-Wędrychowicz, 12.41.
\end{integral}

% Banaś, Wędrychowicz: 12.42
\begin{integral}
    Banaś-Wędrychowicz, 12.42.
\end{integral}

% Banaś, Wędrychowicz: 12.43
\begin{integral}
    Banaś-Wędrychowicz, 12.43.
\end{integral}

% Banaś, Wędrychowicz: 12.44
\begin{integral}
    Banaś-Wędrychowicz, 12.44.
\end{integral}

% Banaś, Wędrychowicz: 12.45
\begin{integral}
    Banaś-Wędrychowicz, 12.45.
\end{integral}

% Banaś, Wędrychowicz: 12.46
\begin{integral}
    Banaś-Wędrychowicz, 12.46.
\end{integral}

% Banaś, Wędrychowicz: 12.47
\begin{integral}
    Banaś-Wędrychowicz, 12.47.
\end{integral}

% Banaś, Wędrychowicz: 12.48
\begin{integral}
    Banaś-Wędrychowicz, 12.48.
\end{integral}

% Banaś, Wędrychowicz: 12.49
\begin{integral}
    Banaś-Wędrychowicz, 12.49.
\end{integral}

% Banaś, Wędrychowicz: 12.50
\begin{integral}
    Banaś-Wędrychowicz, 12.50.
\end{integral}

% Banaś, Wędrychowicz: 12.51
\begin{integral}
    Banaś-Wędrychowicz, 12.51.
\end{integral}

% Banaś, Wędrychowicz: 12.52
\begin{integral}
    Banaś-Wędrychowicz, 12.52.
\end{integral}

% Banaś, Wędrychowicz: 12.53
\begin{integral}
    Banaś-Wędrychowicz, 12.53.
\end{integral}

% Banaś, Wędrychowicz: 12.54
\begin{integral}
    Banaś-Wędrychowicz, 12.54.
\end{integral}

% Banaś, Wędrychowicz: 12.55
\begin{integral}
    Banaś-Wędrychowicz, 12.55.
\end{integral}

% Banaś, Wędrychowicz: 12.56
\begin{integral}
    Banaś-Wędrychowicz, 12.56.
\end{integral}

% Banaś, Wędrychowicz: 12.57
\begin{integral}
    Banaś-Wędrychowicz, 12.57.
\end{integral}

% Banaś, Wędrychowicz: 12.58
\begin{integral}
    Banaś-Wędrychowicz, 12.58.
\end{integral}

\subsection{Podstawienia Eulera}

TODO: Banaś Wędrychowicz, 12.71 - 12.87

\begin{integral}
    Banaś-Wędrychowicz, 12.58.
\end{integral}

% https://math.stackexchange.com/questions/541751/how-prove-this-i-int-0-infty-frac1x-ln-left-frac1x1-x-right2/541861#541861
\begin{integral}[pytanie 541751 na math.stackexchange.com]
    \begin{equation}
        I = \int_0^\infty \frac{1}{x} \log \left(\frac{1+x}{1-x}\right)^2 \,\mathrm{d}x = \pi^2
    \end{equation}
\end{integral}

\begin{proof}
    Podstawiamy $y = (1+x) / (1-y)$:
    \begin{align}
        I & = 2 \int_{-1}^1 \frac{\log y^2}{1-y^2} \,\mathrm{d}y \\
          & = 8 \int_0^1 \frac{\log y}{1-y^2} \, \mathrm{d}{y} \\
          & = 8 \sum_{k=0}^\infty \int_0^1 y^{2k} \log y \,\mathrm{d} y \\
          & = 8 \sum_{k=0}^\infty \frac{1}{(2k+1)^2} \\
          & = 8 \cdot \frac{\pi^2}{8} = 8.
    \end{align}
\end{proof}