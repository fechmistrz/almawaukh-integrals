\section{Trudne całki}

\begin{integralsolved}
    Niech $y \in (-1, 1)$.
    Wtedy
    \begin{equation}
        \int_0^1 \frac{\mathrm{d}x}{(1+yx) \sqrt{1-x^2}} = \frac{\arccos y}{\sqrt{1-y^2}}.
    \end{equation}
\end{integralsolved}

Powyższą całkę znaleźliśmy w \cite[s. 1]{valean}.


\begin{integralsolved}
    Niech $m, n$ będą liczbami naturalnymi.
    Wtedy
    \begin{equation}
        \int_0^1 x^m \log^n x \,\mathrm{d} x = \frac{(-1)^n \cdot n!}{(m+1)^{n+1}}.
    \end{equation}
\end{integralsolved}

Powyższą całkę znaleźliśmy w \cite[s. 1]{valean}.


% TODO: różniczkowanie pod znakiem całki
% https://math.stackexchange.com/questions/942263/really-advanced-techniques-of-integration-definite-or-indefinite
\begin{integral}
    $\int_0^\infty \sin(x) / x \,\mathrm{d}x = \pi/2$.
\end{integral}

% https://math.stackexchange.com/a/9292/1298830
\begin{integral}
    $\int_{-\infty}^\infty \exp(-x^2) \,\mathrm{d}x = \sqrt x$.
\end{integral}

% https://math.stackexchange.com/questions/580521/generalizing-int-01-frac-arctan-sqrtx2-2-sqrtx2-2
\begin{integral}[całka Ahmeda]
    \begin{equation}
        \int_0^1 \frac{\arctan \sqrt{x^2+2}}{(x^2+1) \sqrt{x^2+2}} \,\mathrm{d}x = \frac{5\pi^2}{96}.
    \end{equation}
\end{integral}

\begin{solution}
    Zafar Ahmed -- \emph{Definitely An Integral} w Amer. Math. Monthly 109, 670-671, 2002.
\end{solution}

% https://math.stackexchange.com/q/562694
\begin{integral}[pytanie 562694 na math.stackexchange.com]
    \begin{equation}
        \int_{-1}^1 \frac{1}{x} \sqrt{\frac{1+x}{1-x}} \log \frac{2x^2+2x+1}{2x^2-2x+1} \,\mathrm{d}x = 4 \pi \arccot \sqrt{\phi}.
    \end{equation}
\end{integral}

$$\int_0^\infty \log(x) / (1+x^2) dx = 0$$ (Euler?)

$$\int_0^\infty 1/(x^3 - 1) dx = -pi sqrt 3 / 9$$

% The author of the 1944 paper that first published this gem was D. P. Dalzell, a
% Donald Percy Dalzell
$$ \int x^4 (1-x)^4 / (1+x^2) = 22/7 - pi$$

% - Inside Interesting Integrals s 39 z klamrami


% https://math.stackexchange.com/questions/tagged/integration?tab=votes&page=2&pagesize=15

% https://math.stackexchange.com/questions/457231/how-to-prove-int-infty-infty-fxdx-int-infty-infty-f-left/457271#457271

Another challenging integral is that for the volume of the Reuleaux tetrahedron,
% V	=	int_0^1[(8sqrt(3))/(1+3t^2)-(16sqrt(2)(3t+1)(4t^2+t+1)^(3/2))/((3t^2+1)(11t^2+2t+3)^2)-(sqrt(2)(249t^2+54t+65))/((11t^2+2t+3)^2)]dt,	
% (35)	=	8/3pi-(27)/4cos^(-1)(1/3)+1/4sqrt(2)	(36)=	0.4221577...	 (37)  (OEIS A102888; Weisstein). 

% https://mathworld.wolfram.com/images/gifs/FoxTrotMathTest.jpg -> koniec https://mathworld.wolfram.com/DefiniteIntegral.html

% https://www.amazon.com/dp/303002461X/?coliid=I23YWXHH3UPCFX&colid=2V83EEIVCDHZL&psc=1&ref_=lv_ov_lig_dp_it



% https://www.amazon.com/dp/0867202939