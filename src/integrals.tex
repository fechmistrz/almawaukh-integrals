\documentclass{createspace}
\usepackage{xcolor}
\author{Imię Nazwisko}
\title{Tytuł książki}

\begin{document}

% strona pierwsza

\thispagestyle{empty}
{\noindent\fontsize{18pt}{18pt}\selectfont Księgozbiór matemagiczny, tom ??}

\noindent\makebox[\linewidth]{\rule{\textwidth}{1pt}}

\newpage

% koniec strony pierwszej



% strona druga

\thispagestyle{empty}
\phantom{nothing}
\newpage

% koniec strony drugiej



% strona trzecia

\thispagestyle{empty}
{\noindent\fontsize{18pt}{18pt}\selectfont Imię Nazwisko}

\noindent\makebox[\linewidth]{\rule{\textwidth}{1pt}}

\vspace{10mm}

{\noindent\fontsize{24pt}{24pt}\selectfont \textbf{Tytuł\\(takie tam)}}
\vspace{10mm}

{\noindent\fontsize{14pt}{14pt}\selectfont Wydanie pierwsze}

\newpage

% koniec strony trzeciej



% strona czwarta

\thispagestyle{empty}
\begin{figure}[H]
\begin{minipage}[b]{.48\linewidth}
{\noindent Prof. Imię Nazwisko\\
Gdzie\\
Gdzie dalej\\
Gdzie kraj}
\end{minipage}
\end{figure}

{\noindent \textbf{Tytuł oryginału}\\Tytuł oryginału}
\vspace{5mm}

{\noindent \textbf{Okładkę zaprojektował}\\Okładka}
\vspace{5mm}

{\noindent \textbf{Zredagował}\\Redakcja}
\vspace{5mm}

{\noindent \textbf{Zredagowała technicznie}\\Redakcja techniczna}
\vspace{5mm}

{\noindent \textbf{Złożyli i połamali}\\Skład, łamanie}
\vspace{5mm}

{\noindent \textbf{Korekty dokonali}\\Korekta}

\vfill

{\noindent Copyleft by Antykwariat Czarnoksięski, Gorzów Wielkopolski 2024.
Książka, a także każda jej część, mogą być przedrukowywane oraz w jakikolwiek inny sposób reprodukowane czy powielane mechanicznie, fotooptycznie, zapisywane elektronicznie lub magnetycznie, oraz odczytywane w środkach publicznego przekazu bez pisemnej zgody wydawcy.}

\vspace{5mm}

{\noindent Przygotowano w systemie \TeX, wydrukowano na siarczystym papierze.}

% koniec strony czwartej



% strona piąta

\tableofcontents

\chapter*{Przedmowa}
Sam nie potrafiłbym wymyślić wszystkich całek, jakie zostały zebrane w czytanym przez Ciebie opracowaniu.
Wiele z zaprezentowanych tu problemów, ich rozwiązań (lub jednego i drugiego) zostało zaczerpnięte z popularnych zbiorów, takich jak \cite{wedrychowicz12} Banasia, Wędrychowicza.
Czasami autorzy proszą o zastosowanie wzoru na całkowanie przez części, kiedy zmyślne podstawienie prowadzi do celu krótszą drogą.
Nie przejmowaliśmy się zbytnio poleceniami z ich książki, ale żeby się w tym wszystkim nie pogubić, zamieszczamy na ostatnich stronach ,,tłumaczenie'' naszej numeracji na numerację z rozdziałów 12, 13 \cite{wedrychowicz12}.
Niektóre zadania nie zasługują na szczególną uwagę i choć ich treść została pominięta, nie brakuje ich we wspomniamym przed chwilą ,,tłumaczeniu''.

Gdzie to możliwe, staram się też proponować rozwiązania alternatywne, aby ukazać różnorodność narzędzi, które mogą być użyteczne dla osób całkujących.

% koniec strony piątej


\tableofcontents

\chapter{Pochodne}


\begin{proposition}
    Pochodna jest operatorem liniowym:
    \begin{equation}
        \frac{\mathrm{d}}{\mathrm{d}x} [a f(x) + b g(x)] = a \frac{\mathrm{d}}{\mathrm{d}x} [f(x)] + b \frac{\mathrm{d}}{\mathrm{d}x} [g(x)]
    \end{equation}
\end{proposition}

\begin{proposition}[reguła Leibniza]
    \begin{equation}
        \frac{\mathrm{d}}{\mathrm{d}x} [f(x)g(x)] =  g(x) \frac{\mathrm{d}}{\mathrm{d}x} [f(x)] + f(x)\frac{\mathrm{d}}{\mathrm{d}x} [g(x)]
    \end{equation}
\end{proposition}

\begin{proof}
    Dla oszczędności miejsca, $x_h := x + h$.
\begin{align}
    \frac{\mathrm{d}}{\mathrm{d}x} [f(x)g(x)]
    & = \lim_{h \to 0} \frac{f(x_h)g(x_h) - f(x)g(x)}{h} \\
    & = \lim_{h \to 0} \frac{f(x_h)g(x_h) - f(x)g(x_h) + f(x)g(x_h) - f(x)g(x)}{h} \\
    & = \lim_{h \to 0} \frac{[f(x_h) - f(x)]g(x_h) + f(x)[g(x_h) - g(x)]}{h} \\
    & = \lim_{h \to 0} \frac{f(x_h) - f(x)}{h} \lim_{h\to 0} g(x_h) + 
        \lim_{h \to 0} \frac{g(x_h) - g(x)}{h} \lim_{h \to 0} f(x) \\
    & = g(x) \frac{\mathrm{d}}{\mathrm{d}x} [f(x)] + f(x)\frac{\mathrm{d}}{\mathrm{d}x} [g(x)],
\end{align}
    ponieważ funkcje różniczkowalne są też ciągłe.
\end{proof}

\begin{proposition}
    \label{prp:derivative_power}%
    \begin{equation}
        \frac{\mathrm{d}}{\mathrm{d}x} x^n = nx^{n-1}.
    \end{equation}
\end{proposition}

\begin{proof}
\begin{align}
    \frac{\mathrm{d}}{\mathrm{d}x} x^n
    & = \lim_{h \to 0} \frac{(x+h)^n - x^n}{h} \\
    & = \lim_{h \to 0} \frac{1}{h} \left(\sum_{k=0}^n {n \choose k} x^k h^{n-k} - x^n \right) \\
    & = \lim_{h \to 0} \frac{1}{h} \left(nx^{n-1}h + \sum_{k=0}^{n-2} {n \choose k} x^k h^{n-k}\right) \\
    & = nx^{n-1} + \lim_{h \to 0} \left(\sum_{k=0}^{n-2} {n \choose k} x^k h^{n-k-1}\right) \\
    & = nx^{n-1}.
\end{align}
\end{proof}

\chapter{Całki}

\begin{integral}
    Niech $n \neq -1$.
    Wtedy
\begin{equation}
    \int x^n \,\mathrm{d}x = \frac{x^{n+1}}{n+1}.
\end{equation}
\end{integral}

\begin{proof}
    Wprost z \ref{prp:derivative_power}.
\end{proof}

\end{document}