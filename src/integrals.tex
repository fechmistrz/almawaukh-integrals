\documentclass[9pt, twoside, a5paper, fleqn]{extbook}
\usepackage[
	margin=10mm,
    headsep=5mm,
	twoside,
    includehead,
    % showframe,
]{geometry}

\usepackage{fancyhdr}
\fancypagestyle{plain}{
    \fancyhf{}
    \fancyfoot[C]{}
    \renewcommand{\headrulewidth}{0pt}
    \renewcommand{\footrulewidth}{0pt}
}
\pagestyle{fancy}
\fancyhead{}
\fancyfoot{}
% 341      Rozdział 7. Kalifat algebry | 7.2 Pierścień adeli              342 |
\fancyhead[LE]{\thepage}
\fancyhead[RE]{\nouppercase{\leftmark}}
\fancyhead[LO]{\nouppercase{\rightmark}}
\fancyhead[RO]{\thepage}
\renewcommand{\headrulewidth}{0.4pt}
\renewcommand{\footrulewidth}{0.0pt}

\usepackage{parskip}
\usepackage{multicol}
\usepackage{float} % [H] for figure environment
\usepackage{hyperref} % for \url and \href

\usepackage{polski}
\usepackage[T1]{fontenc}

\usepackage{xcolor}

\usepackage{amsmath, amsfonts, amssymb, amsthm}
\DeclareMathOperator{\arsinh}{arsinh}
\newcommand{\N}{{\mathbb N}}
\newcommand{\R}{{\mathbb R}}

\newcounter{counter}
% \numberwithin{counter}{chapter}
\newtheorem{proposition}[counter]{Fakt}
\newtheorem{problem}[counter]{Problem}
\newtheorem{problem_with_solution}[counter]{Problem z rozwiązaniem}
\newtheorem{exercise}{Exercise}[chapter]

\theoremstyle{remark}
\newtheorem*{solution}{Rozwiązanie}

\author{Imię Nazwisko}
\title{Tytuł książki}

\usepackage{imakeidx}
\usepackage{makeidx}
\usepackage{etoolbox} % for patchcmd
\patchcmd{\theindex}{\MakeUppercase\indexname}{\sffamily\normalsize\bfseries\indexname}{}{}
\makeindex[title=Skorowidz]
\makeindex[name=persons,title=Indeks osób]

\usepackage{Alegreya}

\begin{document}
\input{00_intro/head_1}
\input{00_intro/head_2}
\input{00_intro/head_3}
\input{00_intro/head_4}
\input{00_intro/head_5}

\raggedbottom

\chapter{Pochodne}
\input{sections/derivatives}

\chapter{Całki}
\input{sections/by_derivatives}
\section{Całkowanie przez podstawianie}
\section{Całkowanie przez podstawianie} % SOLUTION

% podstawienia trygonometryczne start

\begin{multicols}{2}
% Banaś, Wędrychowicz, 12.21.
\begin{problem}
    \label{banas_12_21}%
    \begin{equation}
        \int \sqrt{x^2 + 1} \, \mathrm{d}x.
    \end{equation}
\end{problem}

\textbf{Problem \ref{banas_12_21}}. % SOLUTION
Podstawiamy $x = \tan \theta$. % SOLUTION
Na mocy tożsamości trygonometrycznej $\tan^2 \theta + 1 = \sec^2 \theta$ nasza całka zmienia się w $\int \sec^2 \theta \cdot \sec \theta \,\mathrm{d}\theta$, czyli całkę z problemu \ref{banas_12_21_auxilia}. % SOLUTION
Zatem % SOLUTION
\begin{align} % SOLUTION
    \int \sqrt{x^2 + 1} \, \mathrm{d}x & = \frac 12 \sec \theta \tan \theta + \log |\tan \theta + \sec \theta| \\ % SOLUTION
    & = \frac 1 2 x \sqrt{x^2 + 1} + \log \left|x + \sqrt{x^2+1}\right|. % SOLUTION
\end{align} % SOLUTION

\begin{problem}
    \label{nahin_1x_1x}%
    \begin{equation}
        \int_{-1}^1 \sqrt{\frac{1+x}{1-x}} \,\mathrm{d}x = \pi.
    \end{equation}
\end{problem}

\textbf{Problem \ref{nahin_1x_1x}} -- \cite[s. 115, 378]{nahin15}. % SOLUTION
Wskazówka: podstawić $x = \cos 2 \varphi$. % SOLUTION

% podstawienia trygonometryczne koniec

% Banaś, Wędrychowicz, 12.18.
\begin{problem}
    \label{banas_12_18}%
    \begin{equation}
        \int (\arcsin x)^2 \,\mathrm{d}x.
    \end{equation}
\end{problem}

\textbf{Problem \ref{banas_12_18}}. % SOLUTION
Podstawiamy $u = \arcsin x$ i dostajemy całkę z $u^2 \cos u$, którą rozwiązujemy przez części, tak jak w przykładzie \ref{banas_12_14}. % SOLUTION

% Banaś, Wędrychowicz, 12.19.
\begin{problem}
    \label{banas_12_19}%
    \begin{equation}
        \int \sin(\log x) \, \mathrm{d}x.
    \end{equation}
\end{problem}

\textbf{Problem \ref{banas_12_19}}. % SOLUTION
Podstawiamy $u = \log x$, $\mathrm{d} u = \mathrm{d} x / x$, $x = \exp u$ i dostajemy całkę z $e^u \sin u$, którą rozwiązujemy przez części, tak jak w przykładzie \ref{banas_12_19_auxilia}. % SOLUTION

% Banaś, Wędrychowicz, 12.20.
\begin{problem}
    \label{banas_12_20}%
    \begin{equation}
        \int \cos(\log x) \, \mathrm{d}x.
    \end{equation}
\end{problem}    

% Banaś, Wędrychowicz: 12.29
\begin{problem}
    Banaś-Wędrychowicz, 12.29.
\end{problem}

% Banaś, Wędrychowicz: 12.30
\begin{problem}
    Banaś-Wędrychowicz, 12.30.
\end{problem}

% Banaś, Wędrychowicz: 12.31
\begin{problem}
    Banaś-Wędrychowicz, 12.31.
\end{problem}

% Banaś, Wędrychowicz: 12.32
\begin{problem}
    Banaś-Wędrychowicz, 12.32.
\end{problem}

% Banaś, Wędrychowicz: 12.33
\begin{problem}
    Banaś-Wędrychowicz, 12.33.
\end{problem}

% Banaś, Wędrychowicz: 12.34
\begin{problem}
    Banaś-Wędrychowicz, 12.34.
\end{problem}

% Banaś, Wędrychowicz: 12.35
\begin{problem}
    Banaś-Wędrychowicz, 12.35.
\end{problem}

% Banaś, Wędrychowicz: 12.36
\begin{problem}
    Banaś-Wędrychowicz, 12.36.
\end{problem}

% Banaś, Wędrychowicz: 12.37
\begin{problem}
    Banaś-Wędrychowicz, 12.37.
\end{problem}

% Banaś, Wędrychowicz: 12.38
\begin{problem}
    Banaś-Wędrychowicz, 12.38.
\end{problem}

% Banaś, Wędrychowicz: 12.39
\begin{problem}
    Banaś-Wędrychowicz, 12.39.
\end{problem}

% Banaś, Wędrychowicz: 12.40
\begin{problem}
    Banaś-Wędrychowicz, 12.40.
\end{problem}

% Banaś, Wędrychowicz: 12.41
\begin{problem}
    Banaś-Wędrychowicz, 12.41.
\end{problem}

% Banaś, Wędrychowicz: 12.42
\begin{problem}
    Banaś-Wędrychowicz, 12.42.
\end{problem}

% Banaś, Wędrychowicz: 12.43
\begin{problem}
    Banaś-Wędrychowicz, 12.43.
\end{problem}

% Banaś, Wędrychowicz: 12.44
\begin{problem}
    Banaś-Wędrychowicz, 12.44.
\end{problem}

% Banaś, Wędrychowicz: 12.45
\begin{problem}
    Banaś-Wędrychowicz, 12.45.
\end{problem}

% Banaś, Wędrychowicz: 12.46
\begin{problem}
    Banaś-Wędrychowicz, 12.46.
\end{problem}

% Banaś, Wędrychowicz: 12.47
\begin{problem}
    Banaś-Wędrychowicz, 12.47.
\end{problem}

% Banaś, Wędrychowicz: 12.48
\begin{problem}
    Banaś-Wędrychowicz, 12.48.
\end{problem}

% Banaś, Wędrychowicz: 12.49
\begin{problem}
    Banaś-Wędrychowicz, 12.49.
\end{problem}

% Banaś, Wędrychowicz: 12.50
\begin{problem}
    Banaś-Wędrychowicz, 12.50.
\end{problem}

% Banaś, Wędrychowicz: 12.51
\begin{problem}
    Banaś-Wędrychowicz, 12.51.
\end{problem}

% Banaś, Wędrychowicz: 12.52
\begin{problem}
    Banaś-Wędrychowicz, 12.52.
\end{problem}

% Banaś, Wędrychowicz: 12.53
\begin{problem}
    Banaś-Wędrychowicz, 12.53.
\end{problem}

% Banaś, Wędrychowicz: 12.54
\begin{problem}
    Banaś-Wędrychowicz, 12.54.
\end{problem}

% Banaś, Wędrychowicz: 12.55
\begin{problem}
    Banaś-Wędrychowicz, 12.55.
\end{problem}

% Banaś, Wędrychowicz: 12.56
\begin{problem}
    Banaś-Wędrychowicz, 12.56.
\end{problem}

% Banaś, Wędrychowicz: 12.57
\begin{problem}
    Banaś-Wędrychowicz, 12.57.
\end{problem}

% Banaś, Wędrychowicz: 12.58
\begin{problem}
    Banaś-Wędrychowicz, 12.58.
\end{problem}

\end{multicols}

\subsection{Podstawienia Eulera}

TODO: Banaś Wędrychowicz, 12.71 - 12.87

\begin{problem}
    Banaś-Wędrychowicz, 12.58.
\end{problem}

%
%

\section{Całkowanie przez części}

\begin{proposition}[wzór na całkowanie przez części]
\label{prp_int_by_parts}%
    Jeśli funkcje $f, g \colon I \to \R$ są różniczkowalne, to
    \begin{equation}
        \int f(x) g'(x) \,\mathrm{d}x = f(x) g(x) - \int f'(x) g(x) \,\mathrm{d} x.
    \end{equation}
\end{proposition}

\begin{proof}
    Całkujemy obie strony wzoru na pochodną iloczynu $(fg)' = fg' + f'g$, a następnie porządkujemy strony równości.
\end{proof}

% Banaś, Wędrychowicz: 12.1
\begin{problem}
    \label{banas_12_1}%
    $\int x \sin x \,\mathrm{d} x$.
\end{problem}

\textbf{Problem \ref{banas_12_1}}. % SOLUTION
Całkujemy przez części, $f(x) = x$, $g'(x) = \sin x$. % SOLUTION
\begin{align} % SOLUTION
    \int x \sin x \,\mathrm{d} x & = -x \cos x - \int - \cos x \, \mathrm{d}x \\ % SOLUTION
                                 & = -x \cos x + \sin x. % SOLUTION
\end{align} % SOLUTION

Analogicznie obliczamy całki:

% Banaś, Wędrychowicz: 12.2
\begin{problem}
    \label{banas_12_2}%
    $\int x \cos x \,\mathrm{d} x = x \sin x + \cos x$.
\end{problem}

% Banaś, Wędrychowicz: 12.3
% Banaś, Wędrychowicz: 12.4 - podobna
% Banaś, Wędrychowicz: 12.5 - podobna
\begin{problem}
    \label{banas_12_3}%
    $\int x \exp x \,\mathrm{d} x = (x-1) e^x$.
\end{problem}

% Banaś, Wędrychowicz: 12.6
\begin{problem}
    \label{banas_12_6}%
    $\int x \arctan x \,\mathrm{d} x$.
\end{problem}

\textbf{Problem \ref{banas_12_6}}. % SOLUTION
Całkujemy przez części, $f(x) = \arctan x$, $g'(x) = x$. % SOLUTION
\begin{align} % SOLUTION
    \int x \arctan x \, \mathrm{d} x % SOLUTION
    & = \frac 12 x^2 \arctan x - \int \frac{x^2 \,\mathrm{d}x}{2(x^2+1)} \\ % SOLUTION
    & = \frac 12 x^2 \arctan x - \frac 12 \left(\int 1 \,\mathrm{d}x - \int \frac{\mathrm{d}x}{x^2+1} \right) \\ % SOLUTION
    & = \frac 12 x^2 \arctan x - \frac 12 \left(x - \arctan x \right) \\ % SOLUTION
    & = \frac 12 \left((x^2+1)\arctan x - x \right). % SOLUTION
\end{align} % SOLUTION

% Banaś, Wędrychowicz: 12.7
\begin{problem}
    \label{banas_12_7}%
    $\int x^n \log x \,\mathrm{d} x$, gdzie $n \in \N$.
\end{problem}

\textbf{Problem \ref{banas_12_7}}. % SOLUTION
Całkujemy przez części, $f(x) = \log x$, $g'(x) = x^n$. % SOLUTION
\begin{align} % SOLUTION
    \int x^n \log x \, \mathrm{d} x & = \frac{x^{n+1} \log x}{n+1} - \int \frac{x^n \,\mathrm{d} x}{n+1} \\ % SOLUTION
                                    & = \frac{x^{n+1} \log x}{n+1} - \frac{x^{n+1}}{(n+1)^2}. % SOLUTION
\end{align} % SOLUTION

% Banaś, Wędrychowicz: 12.8
\begin{problem}
    \label{banas_12_8}%
    $\int \arccos x \,\mathrm{d} x$.
\end{problem}

\textbf{Problem \ref{banas_12_8}}. % SOLUTION
Całkujemy najpierw przez części, $f(x) = \arccos x$, $g'(x) = 1$, żeby następnie podstawić $u = 1 - x^2$, $\mathrm{d} u = -2x \mathrm{d}x$: % SOLUTION
\begin{align} % SOLUTION
    \int \arccos x \, \mathrm{d} x & = x \arccos x - \int  \frac{-x \,\mathrm{d}x}{\sqrt{1-x^2}} \\ % SOLUTION
    & = x \arccos x - \frac 12 \int \frac {\mathrm{d}u}{\sqrt{u}} \\ % SOLUTION
    & = x \arccos x - \sqrt{1 - x^2}. % SOLUTION
\end{align} % SOLUTION

% Banaś, Wędrychowicz: 12.9
\begin{problem}
    \label{banas_12_9}%
    $\int \arcsin x \,\mathrm{d} x = x \arcsin x + \sqrt{1-x^2}$.
\end{problem}

% Banaś, Wędrychowicz: 12.10
\begin{problem}
    \label{banas_12_10}%
    $\int x (\tan x)^2 \,\mathrm{d} x$.
\end{problem}

\textbf{Problem \ref{banas_12_10}}. % SOLUTION
Całkujemy przez części, $f(x) = x$, $g'(x) = (\tan x)^2$. % SOLUTION
\begin{align} % SOLUTION
    \int x (\tan x)^2 x \, \mathrm{d} x & = x (\tan x - x) - \int (\tan x - x) \,\mathrm{d}x \\ % SOLUTION
    & = x (\tan x - x) - \left(-\log(\cos(x)) - \frac{x^2}{2}\right). % SOLUTION
\end{align} % SOLUTION

% Banaś, Wędrychowicz: 12.11
\begin{problem}
    \label{banas_12_11}%
    $\int x (\cos x)^2 \,\mathrm{d} x$.
\end{problem}

\textbf{Problem \ref{banas_12_11}}. % SOLUTION
Ponieważ $\cos 2x = 2 \cos^2 x - 1$, potrzebujemy znaleźć prostszą całkę  % SOLUTION
\begin{align} % SOLUTION
    \int x \cos 2x \, \mathrm{d} x. % SOLUTION
\end{align} % SOLUTION
Całkujemy przez części: $f(x) = x$, $g'(x) = \cos 2x$, co prowadzi do jeszce prostszej całki funkcji $\sin 2x$. % SOLUTION
Ostatecznie % SOLUTION
\begin{align} % SOLUTION
    \int x \cos 2x \, \mathrm{d} x = \frac 1 8 \left(2x^2 + 2x \sin 2x + \cos 2x\right). % SOLUTION
\end{align} % SOLUTION

% Banaś, Wędrychowicz, 12.12 to całka z x log(x^2+1), ale tam wystarczy podstawić u = x^2 + 1, wtedy du = 2x dx.
% Banaś, Wędrychowicz, 12.16
% Banaś, Wędrychowicz, 12.17.
\begin{problem}
    \label{banas_12_12}%
    $\int (\log x)^n \,\mathrm{d}x$.
\end{problem}

\textbf{Problem \ref{banas_12_12}}. % SOLUTION
Całkujemy przez części, $f(x) = (\log x)^n$, $g'(x) = 1$. % SOLUTION
Dostajemy początek rekurencji: % SOLUTION
\begin{equation} % SOLUTION
    \int (\log x)^n \, \mathrm{d}x = x (\log x)^n - n \int (\log x)^{n-1} \,\mathrm{d} x. % SOLUTION
\end{equation} % SOLUTION
z warunkiem brzegowym: % SOLUTION
\begin{equation} % SOLUTION
    \int \log x\, \mathrm{d}x = x\log x - x. % SOLUTION
\end{equation} % SOLUTION

% Banaś, Wędrychowicz, 12.13.
\begin{problem}
    \label{banas_12_13}%
    Niech $n$ będzie liczbą naturalną, wtedy
    \begin{equation}
        I_n = \int x^n e^x \,\mathrm{d} x = e^x \sum_{k=0}^n (-1)^{n-k} \frac{n!}{k!}x^k.
    \end{equation}
\end{problem}

\textbf{Problem \ref{banas_12_13}}. % SOLUTION
Dowiedziemy tego indukcyjnie. % SOLUTION
Dla $n = 0$, całka jest elementarna. % SOLUTION
Jeżeli $n \ge 1$, to całkujemy przez części: $f(x) = x^n$, $g'(x) = e^x$ i dostajemy zależność rekurencyjną % SOLUTION
\begin{equation} % SOLUTION
    I_n = x^n e^x - nI_{n-1}. % SOLUTION
\end{equation} % SOLUTION

% Banaś, Wędrychowicz, 12.14.
% Banaś, Wędrychowicz, 12.15.
% Banaś, Wędrychowicz, 12.25.
\begin{problem}
    \label{banas_12_14}%
    $\int x^3 \sin x \, \mathrm{d}x$.
\end{problem}

\textbf{Problem \ref{banas_12_14}}. % SOLUTION
Całkujemy przez części, $f(x) = x^3$, $g'(x) = \sin x$. % SOLUTION
Dostajemy początek rekurencji: % SOLUTION
\begin{equation} % SOLUTION
    \int x^3 \sin x \, \mathrm{d}x = - x^3 \cos x - \int - 3x^2 \cos x \,\mathrm{d}x % SOLUTION
\end{equation} % SOLUTION
rozwiązaniem której jest $3 (x^2-2) \sin x + x (6-x^2) \cos x$. % SOLUTION

\begin{problem}
    % pomocnicza dla Banaś 12.19
    \label{banas_12_19_auxilia}%
    \begin{equation}
        \int e^x \sin x \,\mathrm{d}x = \frac {e^x} 2 (\sin x - \cos x).
    \end{equation}
\end{problem}

% Banaś, Wędrychowicz, 12.22.
% \begin{problem}
% Banaś-Wędrychowicz, 12.22. % x^2 e^x sin x
% \end{problem}

% Banaś, Wędrychowicz, 12.23.
% \begin{problem}
% Banaś-Wędrychowicz, 12.23. % x / (sin ^2 x)
% \end{problem}

% Banaś, Wędrychowicz, 12.24.
% \begin{problem}
% Banaś-Wędrychowicz, 12.24. % x arcsin x / (1 - x^2)
% \end{problem}

% Banaś, Wędrychowicz, 12.26.
\begin{problem}
    \label{banas_12_26}%
    \begin{equation}
        \int \frac{x \log(\sqrt{x^2+1}+x)}{\sqrt{x^2+1}} \,\mathrm{d}x = \sqrt{x^2+1} \arsinh x - x.
    \end{equation}
\end{problem}

\textbf{Problem \ref{banas_12_26}}. % SOLUTION
Zauważamy, że $\log(\sqrt{x^2+1} + x) = \arsinh x$, a następnie całkujemy przez części: $f(x) = \arsinh x$, $g'(x) = x / \sqrt{x^2+1}$, wtedy $f(x) = 1/\sqrt{x^2+1}$, $g(x) = \sqrt{x^2+1}$. % SOLUTION

% Banaś, Wędrychowicz, 12.27.
% \begin{problem}
% Banaś-Wędrychowicz, 12.27. % arc tg sqrt (x)
% \end{problem}

% Banaś, Wędrychowicz, 12.28.
% \begin{problem}
% Banaś-Wędrychowicz, 12.28. % x e ^ (arctg x) / (1+x^2)^1.5
% \end{problem}
%

\begin{problem}
    \label{banas_12_21_auxilia}%
    $I = \int \sec^3 x \,\mathrm{d}x$.
\end{problem}

\textbf{Problem \ref{banas_12_21_auxilia}}. % SOLUTION
Całkujemy najpierw przez części, $f(x) = \sec x$, $g'(x) = \sec^2 x$. % SOLUTION
Pamiętając, że $\tan^2 x = \sec^2 x - 1$, mamy % SOLUTION
\begin{align} % SOLUTION
    I & = \sec x \tan x - \int \sec x \tan^2 x \,\mathrm{d} x \\ % SOLUTION
        & = \sec x \tan x - I + \int \sec x \,\mathrm{d}x \\ % SOLUTION
    2I & = \sec x \tan x + \log |\tan x + \sec x|. % SOLUTION
\end{align} % SOLUTION

%

\section{Całkowanie funkcji wymiernych}
% SOLUTION
\section{Całkowanie funkcji wymiernych}
% SOLUTION

\begin{problem}
\label{boros_4287}%
\begin{equation}
    \int_0^\infty \frac{x^n \,\mathrm{d}x}{(ax+b)^{m+1}}  = \frac{(-1)^{n+1} (-1-m)! \cdot n!}{a^{n+1} b^{m-n} (n-m)!}
\end{equation}
\end{problem}

% SOLUTION
\textbf{Problem \ref{boros_4287}} -- patrz \cite[s. 48-60]{boros04}
% SOLUTION

\begin{problem}
\label{frac_1_x3_1}%
\begin{equation}
    \int_0^\infty \frac{\mathrm{d}x}{x^3 - 1} = - \frac{\pi}{3\sqrt{3}}
\end{equation}
\end{problem}

% SOLUTION
\textbf{Problem \ref{frac_1_x3_1}} -- patrz \cite[s. 22]{nahin15}
% SOLUTION

\begin{problem}
    \label{experimental_mathematics_p258}%
\begin{equation}
    \int_0^\infty \frac{x^8-4x^6+9x^4-5x^2+1}{x^{12}-10x^{10}+37x^8-42x^6+26x^4-8x^2+1} \,\mathrm{d}x = \frac{\pi}{2}
\end{equation}
\end{problem}

% SOLUTION
\textbf{Problem \ref{experimental_mathematics_p258}} -- patrz \cite[s. 258]{bailey07}.
% SOLUTION

\subsection{Oszacowania liczby $\pi$}
\begin{problem}
\label{22_7_pi}%
\begin{equation}
    \int_0^1 \frac{x^4(1-x)^4}{1 + x^4} \,\mathrm{d}x = 22/7 - \pi.
\end{equation}
\end{problem}

Czasami używa się określenia całka Dalzella, ponieważ Donald Percy Dalzell \cite{dalzell44} jako pierwszy opublikował to cudo.
Ograniczając mianownik z dołu oraz góry przez $1$ oraz $2$ możemy dojść do wniosku, że
\begin{equation}
    \frac{22}{7} - \frac {1}{630} < \pi < \frac{22}{7} - \frac{1}{1260},
\end{equation}
a więc pomylić się o mniej niż $0.015\%$!

% TODO - na en.wiki jest łatwe w przepisaniu rozwiązanie
% SOLUTION
\textbf{Problem \ref{22_7_pi}} -- patrz \cite[s. 24]{nahin15}
% SOLUTION

% TODO: patrz też https://math.stackexchange.com/questions/1956/is-there-an-integral-that-proves-pi-333-106
Istnieje wiele uogólnień powyższego wyniku do innych przybliżeń liczby $\pi$, dobrym źródłem dalszych informacji, a może nawet inspiracji jest artykuł Lucasa \cite{lucas05}.
Na przykład:

\begin{problem}
\begin{equation}
    \int_0^1 \frac {x^8(1-x)^8 (25+816x^2)}{3164 (1+x^2)} \,\mathrm{d} x = \frac {355}{113} - \pi.
\end{equation}
\end{problem}

albo:

\begin{problem}
\begin{equation}
    \int_0^1 \frac{x^5 ( 1-x)^6 (197 + 462 x^2)}{530 (1+x^2)} \,\mathrm{d}x = \pi - \frac{333}{106}
\end{equation}
\end{problem}


%

\begin{problem}
    \begin{equation}
        \int \sec x \,\mathrm{d} x = \log| \sec x + \tan x|.
    \end{equation}
\end{problem}

% https://en.wikipedia.org/wiki/Integral_of_the_secant_function#History
W 1599 roku Edward Wright znalazł wartość tej całki metodami numerycznymi (było mu to potrzebne by dokładnie skonstruować odwzorowanie walcowe wiernokątne -- Mercatora).
\index{odwzorowanie Mercatora}%
\index[persons]{Wright, Edward}%
W latach czterdziestych XVI wieku Henry Bond porównał wyniki Wrighta z tablicami logarytmicznymi i postawił hipotezę co do zwartej postaci całki.
\index[persons]{Bond, Henry}%
Pierwsze rozwiązanie podał szkocki matematyk i astronom James Gregory w pracy \emph{,,Exercitationes Geometricae''} z 1668 roku, ale było tak trudne w zrozumieniu (chociaż poprawne; podstawił $u = \sec \theta + \tan \theta$), że angielski teolog i matematyk Isaac Barrow zaproponował w \emph{,,Lectiones Geometricae''} z 1670 roku zupełnie inne podejście.
\index[persons]{Gregory, James}%
\index[persons]{Barrow, Isaac}%
Dowód Barrowa jest znany przede wszystkim dlatego, że zawiera najstarszy znany rozkład na ułamki proste podczas całkowania (!).
\index{rozkład na ułamki proste}%

\begin{problem}[sinus całkowy]
    \begin{equation}
        \int_0^\infty \frac {\sin x}{x} \,\mathrm{d} x = \frac \pi 2.
    \end{equation}
\end{problem}

% TODO: https://math.stackexchange.com/questions/13344/proof-of-int-0-infty-left-frac-sin-xx-right2-mathrm-dx-frac-pi2
\begin{problem}
    \begin{equation}
        \int_0^\infty \left(\frac {\sin x}{x}\right)^2 \,\mathrm{d} x = \frac \pi 2.
    \end{equation}
\end{problem}

% TODO: https://math.stackexchange.com/questions/9286/evaluation-of-gaussian-integral-int-0-infty-mathrme-x2-dx
\begin{problem}
    \begin{equation}
        I = \int_{-\infty}^\infty \exp \left( -x^2 \right) \,\mathrm{d} x = \sqrt{\pi}.
    \end{equation}
\end{problem}

% https://math.stackexchange.com/questions/34767/int-infty-infty-e-x2-dx-with-complex-analysis
Podstawiamy $u = x^2$, $\mathrm{d} u = 2x \,\mathrm{d}x$.
Wtedy
\begin{align}
	I = \int_0^\infty u^{-1/2} e^{-u}\,\mathrm{d}u = \Gamma \left(\frac 12\right) = \sqrt{\pi}
\end{align}
na mocy wzoru $\Gamma (z) \Gamma(1-z) = \pi/\sin \pi z$.

\input{sections/differential_binomial}
% \section{Całkowanie funkcji trygonometrycznych}
% Całkowanie funkcji trygonometrycznych
	%

\section{Sztuczka Feynmana: różniczkowanie pod znakiem całki}
\section{Sztuczka Feynmana: różniczkowanie pod znakiem całki} % SOLUTION

% https://math.stackexchange.com/questions/942263/really-advanced-techniques-of-integration-definite-or-indefinite
\begin{problem}
    $\int_0^\infty \sin(x) / x \,\mathrm{d}x = \pi/2$.
\end{problem}

% TODO: przepisać całkę z s. 82, Nahin

\begin{problem}
    \label{nahin_holzweg}%
    Niech $a, b > 0$, wtedy
    \begin{equation}
        \int_{-\infty}^\infty \frac{\cos ax}{b^2 - x^4} \,\mathrm{d} x = \frac{\pi}{b} \sin (ab)
    \end{equation}
\end{problem}

\textbf{Problem \ref{nahin_holzweg}} -- \cite[s. 115, 375, 376]{nahin15}. % SOLUTION

\begin{problem}
    \label{nahin_datenautobahn}%
    Niech $a > b$, wtedy
    \begin{equation}
        \int_{-\infty}^\infty \frac{\cos ax}{b^4 - x^4} \,\mathrm{d} x = \frac{\pi}{2b^3} [\sin (ab) + \exp (-ab)]
    \end{equation}
\end{problem}

\textbf{Problem \ref{nahin_datenautobahn}} -- \cite[s. 115, 376]{nahin15}. % SOLUTION

% Nahin Inside interesting... page 83
\begin{problem}
    \begin{equation}
        \int_0^\infty \frac{\sin ax}{x e^{xy}} \,\mathrm{d}x = \pm \frac \pi 2 - \arctan \frac y a.
    \end{equation}
\end{problem}

\begin{problem}
    Niech $a > 0$.
    Wtedy
    \begin{equation}
        \int_0^\infty \frac{\sin ax}{x} \,\mathrm{d}x = \frac \pi 2.
    \end{equation}
\end{problem}

% Nahin Inside interesting... page 85
\begin{problem}[całka Frullaniego]
    \begin{equation}
        \int_0^\infty \frac{\arctan (ax) - \arctan (bx)}{x} \,\mathrm{d}x = \frac \pi 2 \log \frac a b.
    \end{equation}
\end{problem}

% Nahin Inside interesting... page 8x
\begin{problem}
    Niech $a, b > 0$.
    Wtedy
    \begin{equation}
        \int_0^\infty \frac{e^{-ax} - e^{-bx}}{x} \,\mathrm{d}x = \log \frac b a.
    \end{equation}
\end{problem}

% Nahin Inside interesting... page 89
\begin{problem}
    \begin{equation}
        \int_0^\infty \frac{\cos (ax) - \cos (bx)}{x^2} \,\mathrm{d}x = \frac \pi 2 (b - a).
    \end{equation}
\end{problem}

% Nahin Inside interesting... page 89
\begin{problem}
    \begin{equation}
        \int_0^\infty \frac{\cos (ax) - \cos (bx)}{x} \,\mathrm{d}x = \log \frac b a.
    \end{equation}
\end{problem}

% Nahin Inside interesting... page 89
\begin{problem}
    \label{nahin_kriegsrecht}
    \begin{equation}
        \int_0^\infty \frac{\log (a^2 x^2 + 1)}{x^2 + b^2} \,\mathrm{d}x = \frac \pi b \log (1 + ab)
    \end{equation}
\end{problem}

\textbf{Problem \ref{nahin_kriegsrecht}} -- \cite[s. 67]{nahin15} w szczególnym przypadku $a = b = 1$; \cite[s. 114, 375]{nahin15} w ogólności. % SOLUTION

% Nahin Inside interesting... page 91
\begin{problem}
    Niech $a \ge 0$, wtedy
    \begin{equation}
        \int_0^1 \frac{x^a - 1}{\log x} \,\mathrm{d}x = \log(1+a).
    \end{equation}
\end{problem}

% Nahin Inside interesting... page 92
\begin{problem}
    Niech $a \ge 0$, wtedy
    \begin{equation}
        \int_0^1 \frac{x^a - x^b}{\log x} \,\mathrm{d}x = \log \frac{1+a}{1+b}.
    \end{equation}
\end{problem}

% Nahin Inside interesting... page 96
\begin{problem}
    Niech $a > b$, wtedy
    \begin{equation}
        \int_0^\pi \frac{\mathrm{d}x} {a + b \cos x} = \frac{\pi}{\sqrt{a^2 - b^2}}.
    \end{equation}
\end{problem}

Nahin pisze, że poniższą całkę wyznaczył jako pierwszy włoski matematyk Ulisse Dini w 1878 roku i że ma ważne zastosowania w fizyce i inżynierii.
\index[persons]{Dini, Ulisse}%

\begin{problem}
    \label{nahin_dini}%
    Niech $a \ge 0$ będzie dowolną liczbą rzeczywistą.
    Wtedy
    \begin{equation}
        \int_0^\pi \log (1 - 2 a \cos x + a^2) \,\mathrm{d} x = \begin{cases}
            0, & \text{gdy } a^2 \le 1, \\
            2 \pi \log a & \textrm{w przeciwnym razie}.
        \end{cases}
    \end{equation}
\end{problem}

\textbf{Problem \ref{nahin_dini}} -- \cite[s. 109-112]{nahin15} % SOLUTION

%

\section{Zadania z turniejów całkowania}
	%

\subsection{MIT Integration BEE 2024} % SOLUTION

\begin{problem}[ćwierćfinał 2, problem 2]
    \begin{equation}
        I = \int_0^1 \frac 1 x \log (1 + x^2 + x^3 + x^4 + x^5 + x^6 + x^7 + x^9) \,\mathrm{d}x
    \end{equation}
\end{problem}

% https://math.stackexchange.com/questions/1617081/proving-an-integration-equality % SOLUTION
\begin{solution} % SOLUTION
    Łatwo widać, że szukana całka jest równa $I_2 + I_3 + I_4$, gdzie % SOLUTION
    \begin{align} % SOLUTION
        I_n & = \int_0^1 \frac {\log (1 + x^n)}{x} \,\mathrm{d}x \\ % SOLUTION
            & = \frac 1 n \int_0^1 \frac {\log (1 + x)}{x} \,\mathrm{d}x \\ % SOLUTION
            & = \frac 1 n \int_0^1 \frac 1 x \sum_{k=1}^\infty \frac{(-1)^{k-1}x^k}{k} \,\mathrm{d}x \\ % SOLUTION
            & = \frac 1 n \sum_{k=1}^\infty \frac{(-1)^{k-1}}{k^2} \\ % SOLUTION
            & = \frac {\pi^2}{12n}. % SOLUTION
    \end{align} % SOLUTION
\end{solution} % SOLUTION


%

\input{sections/makefile-problems/theoretic}

\section{Trudne całki}
	%

\begin{problem_with_solution}
    \label{reuleaux_tetrahedron}%
    Czworościan Reuleaux to bryła będąca częścią wspólną czterech kul, których środki leżą w wierzchołkach czworościanu foremnego, a promienie są tej samej długości, co krawędzie tego czworościanu.
    Znaleźć objętość tej bryły,
    \begin{equation}
        V = \int_0^1
        \frac{
            8\sqrt{3}
        }{
            1 + 3t^2
        } - \frac{
            16 \sqrt{2} (3t+1) (4t^2 +t+1)^{3/2}
        }{
            (3t^2+1)(11t^2 + 2t + 3)^2
        } - \frac{
            \sqrt{2} (249 t^2 + 54t + 65)
        }{
            (11t^2 + 2t +3)^2
        } \,\mathrm{d} t.
    \end{equation}
\end{problem_with_solution}

% SOLUTION
\textbf{Problem \ref{reuleaux_tetrahedron}} -- patrz \url{https://mathworld.wolfram.com/ReuleauxTetrahedron.html}.
% SOLUTION

% https://math.stackexchange.com/questions/580521/generalizing-int-01-frac-arctan-sqrtx2-2-sqrtx2-2
\begin{problem_with_solution}[całka Ahmeda]
    \label{ahmed_integral}%
    \begin{equation}
        \int_0^1 \frac{\arctan \sqrt{x^2+2}}{(x^2+1) \sqrt{x^2+2}} \,\mathrm{d}x = \frac{5\pi^2}{96}.
    \end{equation}
\end{problem_with_solution}

% SOLUTION
\textbf{Problem \ref{ahmed_integral}} -- patrz praca Ahmeda Zafara \cite{ahmed02}.
\index[persons]{Zafar, Ahmed}
% SOLUTION

% TODO: https://mathworld.wolfram.com/images/gifs/FoxTrotMathTest.jpg
% TODO https://mathworld.wolfram.com/DefiniteIntegral.html

%
	\subsection{Prawie niemożliwe całki}
\subsection{Prawie niemożliwe całki} % SOLUTION
Wszystkie poniższe całki pojawiają się w książce Valeana \cite{valean19}.

\begin{problem_with_solution}
    \label{valean_grundpreis}%
    Niech $y \in (-1, 1)$.
    Wtedy
    \begin{equation}
        \int_0^1 \frac{\mathrm{d}x}{(1+yx) \sqrt{1-x^2}} = \frac{\arccos y}{\sqrt{1-y^2}}.
    \end{equation}
\end{problem_with_solution}

\begin{solution}[do problemu \ref{valean_grundpreis}] % SOLUTION
    Patrz \cite[s. 1]{valean19}. % SOLUTION
\end{solution} % SOLUTION

\begin{problem_with_solution}
    \label{valean_zeugenstand}%
    Niech $m, n$ będą liczbami naturalnymi.
    Wtedy
    \begin{equation}
        \int_0^1 x^m \log^n x \,\mathrm{d} x = \frac{(-1)^n \cdot n!}{(m+1)^{n+1}}.
    \end{equation}
\end{problem_with_solution}

\begin{solution}[do problemu \ref{valean_zeugenstand}] % SOLUTION
    Patrz \cite[s. 1]{valean19}. % SOLUTION
\end{solution} % SOLUTION

Niech $H_{n}^{(m)} = 1 + 1/2^m + \ldots + 1/n^m$ oznacza $n$-tą uogólnioną liczbę harmoniczną.

\begin{problem_with_solution}
    \label{valean_1_3}%
    Rozpatrujemy rodzinę całek
    \begin{equation}
        I_{k,n} := \int_0^1 x^{n-1} \log^k (1-x) \,\mathrm{d} x.
    \end{equation}
    Mamy:
    \begin{align}
        I_{1,n} & = - \frac{H_n}{n} \\
        I_{2,n} & = \frac{H_n^2 + H_n^{(2)}}{n} \\
        I_{3,n} & = - \frac{H_n^3 + 3H_nH_n^{(2)} + 2H_n^{(3)}}{n} \\
        I_{4,n} & = \frac{H_n^4 + 6H_n^2 H_n^{(2)} + 8H_nH_n^{(3)} + 3(H_n^{(2)})^2 + 6H_n^{(4)}}{n}.
    \end{align}
\end{problem_with_solution}

% (Valean nazywa to ,,four logarithmic integrals strongly connected with the league of harmonic series'').

\begin{solution}[do problemu \ref{valean_1_3}] % SOLUTION
    Patrz \cite[s. 2]{valean19}. % SOLUTION
\end{solution} % SOLUTION

\begin{problem_with_solution}
    \label{valean_1_5}%
    Niech $s > 0$ będzie liczbą rzeczywistą, zaś $\psi$ oznacza funkcję digamma.
    Wtedy
    \begin{align}
        \int_0^1 \frac{x^{s-1}}{x+1} \,\mathrm{d} x & = \psi(s) - \psi\left(\frac s2\right) - \log 2 \\
        \int_0^\infty e^{-sx} \tanh x \,\mathrm{d} x & = \frac 1 2 \left[\psi\left(\frac{s+2}{4}\right) - \psi \left(\frac s4 \right) - \frac 2 s\right]. 
    \end{align}
\end{problem_with_solution}

% (Valean nazywa to ,,a couple of practical definite integrals expressed in terms of the digamma function'').

\begin{solution}[do problemu \ref{valean_1_5}] % SOLUTION
    Patrz \cite[s. 3]{valean19}. % SOLUTION
\end{solution} % SOLUTION

\begin{problem_with_solution}
    \label{valean_1_7}%
    \begin{align}
        \int_0^1 \frac{1}{x} \log^2 (1+x) \,\mathrm{d}x & = \frac{1}{4} \zeta(3) \\
        \int_0^1 \frac{1}{x} \log (1+x) \log (1-x) \,\mathrm{d}x & = -\frac{5}{8} \zeta(3)
    \end{align}
\end{problem_with_solution}

% two little tricky classical logarithmic integrals

\begin{solution}[do problemu \ref{valean_1_7}] % SOLUTION
    Patrz \cite[s. 4]{valean19}. % SOLUTION
\end{solution} % SOLUTION

\begin{problem_with_solution}[]
    \label{valean_1_8}%
    \begin{align}
        \int_0^1 [\log(1+x) \log(1-x)]^2 \,\mathrm{d} x & =
        24 - 8 \zeta(2)- 8 \zeta(3) - \zeta(4) \\
        & + 8 \log(2)\zeta(2) + 8 \log(2)\zeta(3) \\
        & - 4 \log^2(2)\zeta(2) \\
        & - 24 \log(2) + 12 \log^2(2)- 4 \log^3(2) + \log^4(2); 
    \end{align}
\end{problem_with_solution}

% a special trio of integrals

\begin{solution}[do problemu \ref{valean_1_8}] % SOLUTION
    Patrz \cite[s. 4, 5]{valean19}. % SOLUTION
\end{solution} % SOLUTION

\begin{problem_with_solution}
    \label{valean_1_10}%
    Niech $n \ge 1$ będzie liczbą naturalną.
    Rozpatrujemy rodzinę całek z parametrem
    \begin{equation}
        I_n = \int_0^1 \frac 1 x \log(1-x) \log^{2n} x \log (1+x) \,\mathrm{d}x.
    \end{equation}
    Znaleźć $I_n$ lub, jeśli jest to za trudne, pokazać, że
    \begin{align}
        I_1 & = \frac 3 4 \zeta (2) \zeta (3) - \frac {27}{16} \zeta(5), \\
        I_2 & = \frac 9 4 \zeta (3) \zeta (4) + \frac{45}{4} \zeta(2) \zeta(5) - \frac{363}{16} \zeta (7), \\
        I_3 & = \frac{2835}{8} \zeta(2) \zeta (7) + \frac {135}{8} \zeta (3) \zeta (6) + \frac {675}{8} \zeta (4) \zeta (5) - \frac {22635}{32} \zeta (9).
    \end{align} 
\end{problem_with_solution}

% the evaluation of a class of logarithmic integrals using a slightly modified result from ,,Table of Integrals, Series and Products'' by I. S. Gradshteyn and I. M. Ryzhik together with a series result elementarily proved by Guy Bastien

\begin{solution}[do problemu \ref{valean_1_10}] % SOLUTION
    Patrz \cite[s. 6, 7]{valean19}. % SOLUTION
\end{solution} % SOLUTION

%
	%

\subsection{Znalezione na math.stackexchange.com}
\subsection{Znalezione na math.stackexchange.com} % SOLUTION
Wszystkie poniższe całki pojawiają się na stronie na math.stackexchange.com.

% https://math.stackexchange.com/q/562694
\begin{problem}[pytanie 562694]
    \label{stack_562694}%
    \begin{equation}
        \int_{-1}^1 \frac{1}{x} \sqrt{\frac{1+x}{1-x}} \log \frac{2x^2+2x+1}{2x^2-2x+1} \,\mathrm{d}x = 4 \pi \operatorname{arccot} \sqrt{\phi}.
    \end{equation}
\end{problem}

\begin{solution}[do problemu \ref{stack_562694}] % SOLUTION
    Patrz przypis\footnote{\url{https://math.stackexchange.com/questions/562694/}}. % SOLUTION
\end{solution} % SOLUTION

% TODO: https://math.stackexchange.com/a/942440

%

\chapter{Rozwiązania}
% \input{sections/by_derivatives}
\section{Całkowanie przez podstawianie}
\section{Całkowanie przez podstawianie} % SOLUTION

% podstawienia trygonometryczne start

\begin{multicols}{2}
% Banaś, Wędrychowicz, 12.21.
\begin{problem}
    \label{banas_12_21}%
    \begin{equation}
        \int \sqrt{x^2 + 1} \, \mathrm{d}x.
    \end{equation}
\end{problem}

\textbf{Problem \ref{banas_12_21}}. % SOLUTION
Podstawiamy $x = \tan \theta$. % SOLUTION
Na mocy tożsamości trygonometrycznej $\tan^2 \theta + 1 = \sec^2 \theta$ nasza całka zmienia się w $\int \sec^2 \theta \cdot \sec \theta \,\mathrm{d}\theta$, czyli całkę z problemu \ref{banas_12_21_auxilia}. % SOLUTION
Zatem % SOLUTION
\begin{align} % SOLUTION
    \int \sqrt{x^2 + 1} \, \mathrm{d}x & = \frac 12 \sec \theta \tan \theta + \log |\tan \theta + \sec \theta| \\ % SOLUTION
    & = \frac 1 2 x \sqrt{x^2 + 1} + \log \left|x + \sqrt{x^2+1}\right|. % SOLUTION
\end{align} % SOLUTION

\begin{problem}
    \label{nahin_1x_1x}%
    \begin{equation}
        \int_{-1}^1 \sqrt{\frac{1+x}{1-x}} \,\mathrm{d}x = \pi.
    \end{equation}
\end{problem}

\textbf{Problem \ref{nahin_1x_1x}} -- \cite[s. 115, 378]{nahin15}. % SOLUTION
Wskazówka: podstawić $x = \cos 2 \varphi$. % SOLUTION

% podstawienia trygonometryczne koniec

% Banaś, Wędrychowicz, 12.18.
\begin{problem}
    \label{banas_12_18}%
    \begin{equation}
        \int (\arcsin x)^2 \,\mathrm{d}x.
    \end{equation}
\end{problem}

\textbf{Problem \ref{banas_12_18}}. % SOLUTION
Podstawiamy $u = \arcsin x$ i dostajemy całkę z $u^2 \cos u$, którą rozwiązujemy przez części, tak jak w przykładzie \ref{banas_12_14}. % SOLUTION

% Banaś, Wędrychowicz, 12.19.
\begin{problem}
    \label{banas_12_19}%
    \begin{equation}
        \int \sin(\log x) \, \mathrm{d}x.
    \end{equation}
\end{problem}

\textbf{Problem \ref{banas_12_19}}. % SOLUTION
Podstawiamy $u = \log x$, $\mathrm{d} u = \mathrm{d} x / x$, $x = \exp u$ i dostajemy całkę z $e^u \sin u$, którą rozwiązujemy przez części, tak jak w przykładzie \ref{banas_12_19_auxilia}. % SOLUTION

% Banaś, Wędrychowicz, 12.20.
\begin{problem}
    \label{banas_12_20}%
    \begin{equation}
        \int \cos(\log x) \, \mathrm{d}x.
    \end{equation}
\end{problem}    

% Banaś, Wędrychowicz: 12.29
\begin{problem}
    Banaś-Wędrychowicz, 12.29.
\end{problem}

% Banaś, Wędrychowicz: 12.30
\begin{problem}
    Banaś-Wędrychowicz, 12.30.
\end{problem}

% Banaś, Wędrychowicz: 12.31
\begin{problem}
    Banaś-Wędrychowicz, 12.31.
\end{problem}

% Banaś, Wędrychowicz: 12.32
\begin{problem}
    Banaś-Wędrychowicz, 12.32.
\end{problem}

% Banaś, Wędrychowicz: 12.33
\begin{problem}
    Banaś-Wędrychowicz, 12.33.
\end{problem}

% Banaś, Wędrychowicz: 12.34
\begin{problem}
    Banaś-Wędrychowicz, 12.34.
\end{problem}

% Banaś, Wędrychowicz: 12.35
\begin{problem}
    Banaś-Wędrychowicz, 12.35.
\end{problem}

% Banaś, Wędrychowicz: 12.36
\begin{problem}
    Banaś-Wędrychowicz, 12.36.
\end{problem}

% Banaś, Wędrychowicz: 12.37
\begin{problem}
    Banaś-Wędrychowicz, 12.37.
\end{problem}

% Banaś, Wędrychowicz: 12.38
\begin{problem}
    Banaś-Wędrychowicz, 12.38.
\end{problem}

% Banaś, Wędrychowicz: 12.39
\begin{problem}
    Banaś-Wędrychowicz, 12.39.
\end{problem}

% Banaś, Wędrychowicz: 12.40
\begin{problem}
    Banaś-Wędrychowicz, 12.40.
\end{problem}

% Banaś, Wędrychowicz: 12.41
\begin{problem}
    Banaś-Wędrychowicz, 12.41.
\end{problem}

% Banaś, Wędrychowicz: 12.42
\begin{problem}
    Banaś-Wędrychowicz, 12.42.
\end{problem}

% Banaś, Wędrychowicz: 12.43
\begin{problem}
    Banaś-Wędrychowicz, 12.43.
\end{problem}

% Banaś, Wędrychowicz: 12.44
\begin{problem}
    Banaś-Wędrychowicz, 12.44.
\end{problem}

% Banaś, Wędrychowicz: 12.45
\begin{problem}
    Banaś-Wędrychowicz, 12.45.
\end{problem}

% Banaś, Wędrychowicz: 12.46
\begin{problem}
    Banaś-Wędrychowicz, 12.46.
\end{problem}

% Banaś, Wędrychowicz: 12.47
\begin{problem}
    Banaś-Wędrychowicz, 12.47.
\end{problem}

% Banaś, Wędrychowicz: 12.48
\begin{problem}
    Banaś-Wędrychowicz, 12.48.
\end{problem}

% Banaś, Wędrychowicz: 12.49
\begin{problem}
    Banaś-Wędrychowicz, 12.49.
\end{problem}

% Banaś, Wędrychowicz: 12.50
\begin{problem}
    Banaś-Wędrychowicz, 12.50.
\end{problem}

% Banaś, Wędrychowicz: 12.51
\begin{problem}
    Banaś-Wędrychowicz, 12.51.
\end{problem}

% Banaś, Wędrychowicz: 12.52
\begin{problem}
    Banaś-Wędrychowicz, 12.52.
\end{problem}

% Banaś, Wędrychowicz: 12.53
\begin{problem}
    Banaś-Wędrychowicz, 12.53.
\end{problem}

% Banaś, Wędrychowicz: 12.54
\begin{problem}
    Banaś-Wędrychowicz, 12.54.
\end{problem}

% Banaś, Wędrychowicz: 12.55
\begin{problem}
    Banaś-Wędrychowicz, 12.55.
\end{problem}

% Banaś, Wędrychowicz: 12.56
\begin{problem}
    Banaś-Wędrychowicz, 12.56.
\end{problem}

% Banaś, Wędrychowicz: 12.57
\begin{problem}
    Banaś-Wędrychowicz, 12.57.
\end{problem}

% Banaś, Wędrychowicz: 12.58
\begin{problem}
    Banaś-Wędrychowicz, 12.58.
\end{problem}

\end{multicols}

\subsection{Podstawienia Eulera}

TODO: Banaś Wędrychowicz, 12.71 - 12.87

\begin{problem}
    Banaś-Wędrychowicz, 12.58.
\end{problem}

%
%

\section{Całkowanie przez części}

\begin{proposition}[wzór na całkowanie przez części]
\label{prp_int_by_parts}%
    Jeśli funkcje $f, g \colon I \to \R$ są różniczkowalne, to
    \begin{equation}
        \int f(x) g'(x) \,\mathrm{d}x = f(x) g(x) - \int f'(x) g(x) \,\mathrm{d} x.
    \end{equation}
\end{proposition}

\begin{proof}
    Całkujemy obie strony wzoru na pochodną iloczynu $(fg)' = fg' + f'g$, a następnie porządkujemy strony równości.
\end{proof}

% Banaś, Wędrychowicz: 12.1
\begin{problem}
    \label{banas_12_1}%
    $\int x \sin x \,\mathrm{d} x$.
\end{problem}

\textbf{Problem \ref{banas_12_1}}. % SOLUTION
Całkujemy przez części, $f(x) = x$, $g'(x) = \sin x$. % SOLUTION
\begin{align} % SOLUTION
    \int x \sin x \,\mathrm{d} x & = -x \cos x - \int - \cos x \, \mathrm{d}x \\ % SOLUTION
                                 & = -x \cos x + \sin x. % SOLUTION
\end{align} % SOLUTION

Analogicznie obliczamy całki:

% Banaś, Wędrychowicz: 12.2
\begin{problem}
    \label{banas_12_2}%
    $\int x \cos x \,\mathrm{d} x = x \sin x + \cos x$.
\end{problem}

% Banaś, Wędrychowicz: 12.3
% Banaś, Wędrychowicz: 12.4 - podobna
% Banaś, Wędrychowicz: 12.5 - podobna
\begin{problem}
    \label{banas_12_3}%
    $\int x \exp x \,\mathrm{d} x = (x-1) e^x$.
\end{problem}

% Banaś, Wędrychowicz: 12.6
\begin{problem}
    \label{banas_12_6}%
    $\int x \arctan x \,\mathrm{d} x$.
\end{problem}

\textbf{Problem \ref{banas_12_6}}. % SOLUTION
Całkujemy przez części, $f(x) = \arctan x$, $g'(x) = x$. % SOLUTION
\begin{align} % SOLUTION
    \int x \arctan x \, \mathrm{d} x % SOLUTION
    & = \frac 12 x^2 \arctan x - \int \frac{x^2 \,\mathrm{d}x}{2(x^2+1)} \\ % SOLUTION
    & = \frac 12 x^2 \arctan x - \frac 12 \left(\int 1 \,\mathrm{d}x - \int \frac{\mathrm{d}x}{x^2+1} \right) \\ % SOLUTION
    & = \frac 12 x^2 \arctan x - \frac 12 \left(x - \arctan x \right) \\ % SOLUTION
    & = \frac 12 \left((x^2+1)\arctan x - x \right). % SOLUTION
\end{align} % SOLUTION

% Banaś, Wędrychowicz: 12.7
\begin{problem}
    \label{banas_12_7}%
    $\int x^n \log x \,\mathrm{d} x$, gdzie $n \in \N$.
\end{problem}

\textbf{Problem \ref{banas_12_7}}. % SOLUTION
Całkujemy przez części, $f(x) = \log x$, $g'(x) = x^n$. % SOLUTION
\begin{align} % SOLUTION
    \int x^n \log x \, \mathrm{d} x & = \frac{x^{n+1} \log x}{n+1} - \int \frac{x^n \,\mathrm{d} x}{n+1} \\ % SOLUTION
                                    & = \frac{x^{n+1} \log x}{n+1} - \frac{x^{n+1}}{(n+1)^2}. % SOLUTION
\end{align} % SOLUTION

% Banaś, Wędrychowicz: 12.8
\begin{problem}
    \label{banas_12_8}%
    $\int \arccos x \,\mathrm{d} x$.
\end{problem}

\textbf{Problem \ref{banas_12_8}}. % SOLUTION
Całkujemy najpierw przez części, $f(x) = \arccos x$, $g'(x) = 1$, żeby następnie podstawić $u = 1 - x^2$, $\mathrm{d} u = -2x \mathrm{d}x$: % SOLUTION
\begin{align} % SOLUTION
    \int \arccos x \, \mathrm{d} x & = x \arccos x - \int  \frac{-x \,\mathrm{d}x}{\sqrt{1-x^2}} \\ % SOLUTION
    & = x \arccos x - \frac 12 \int \frac {\mathrm{d}u}{\sqrt{u}} \\ % SOLUTION
    & = x \arccos x - \sqrt{1 - x^2}. % SOLUTION
\end{align} % SOLUTION

% Banaś, Wędrychowicz: 12.9
\begin{problem}
    \label{banas_12_9}%
    $\int \arcsin x \,\mathrm{d} x = x \arcsin x + \sqrt{1-x^2}$.
\end{problem}

% Banaś, Wędrychowicz: 12.10
\begin{problem}
    \label{banas_12_10}%
    $\int x (\tan x)^2 \,\mathrm{d} x$.
\end{problem}

\textbf{Problem \ref{banas_12_10}}. % SOLUTION
Całkujemy przez części, $f(x) = x$, $g'(x) = (\tan x)^2$. % SOLUTION
\begin{align} % SOLUTION
    \int x (\tan x)^2 x \, \mathrm{d} x & = x (\tan x - x) - \int (\tan x - x) \,\mathrm{d}x \\ % SOLUTION
    & = x (\tan x - x) - \left(-\log(\cos(x)) - \frac{x^2}{2}\right). % SOLUTION
\end{align} % SOLUTION

% Banaś, Wędrychowicz: 12.11
\begin{problem}
    \label{banas_12_11}%
    $\int x (\cos x)^2 \,\mathrm{d} x$.
\end{problem}

\textbf{Problem \ref{banas_12_11}}. % SOLUTION
Ponieważ $\cos 2x = 2 \cos^2 x - 1$, potrzebujemy znaleźć prostszą całkę  % SOLUTION
\begin{align} % SOLUTION
    \int x \cos 2x \, \mathrm{d} x. % SOLUTION
\end{align} % SOLUTION
Całkujemy przez części: $f(x) = x$, $g'(x) = \cos 2x$, co prowadzi do jeszce prostszej całki funkcji $\sin 2x$. % SOLUTION
Ostatecznie % SOLUTION
\begin{align} % SOLUTION
    \int x \cos 2x \, \mathrm{d} x = \frac 1 8 \left(2x^2 + 2x \sin 2x + \cos 2x\right). % SOLUTION
\end{align} % SOLUTION

% Banaś, Wędrychowicz, 12.12 to całka z x log(x^2+1), ale tam wystarczy podstawić u = x^2 + 1, wtedy du = 2x dx.
% Banaś, Wędrychowicz, 12.16
% Banaś, Wędrychowicz, 12.17.
\begin{problem}
    \label{banas_12_12}%
    $\int (\log x)^n \,\mathrm{d}x$.
\end{problem}

\textbf{Problem \ref{banas_12_12}}. % SOLUTION
Całkujemy przez części, $f(x) = (\log x)^n$, $g'(x) = 1$. % SOLUTION
Dostajemy początek rekurencji: % SOLUTION
\begin{equation} % SOLUTION
    \int (\log x)^n \, \mathrm{d}x = x (\log x)^n - n \int (\log x)^{n-1} \,\mathrm{d} x. % SOLUTION
\end{equation} % SOLUTION
z warunkiem brzegowym: % SOLUTION
\begin{equation} % SOLUTION
    \int \log x\, \mathrm{d}x = x\log x - x. % SOLUTION
\end{equation} % SOLUTION

% Banaś, Wędrychowicz, 12.13.
\begin{problem}
    \label{banas_12_13}%
    Niech $n$ będzie liczbą naturalną, wtedy
    \begin{equation}
        I_n = \int x^n e^x \,\mathrm{d} x = e^x \sum_{k=0}^n (-1)^{n-k} \frac{n!}{k!}x^k.
    \end{equation}
\end{problem}

\textbf{Problem \ref{banas_12_13}}. % SOLUTION
Dowiedziemy tego indukcyjnie. % SOLUTION
Dla $n = 0$, całka jest elementarna. % SOLUTION
Jeżeli $n \ge 1$, to całkujemy przez części: $f(x) = x^n$, $g'(x) = e^x$ i dostajemy zależność rekurencyjną % SOLUTION
\begin{equation} % SOLUTION
    I_n = x^n e^x - nI_{n-1}. % SOLUTION
\end{equation} % SOLUTION

% Banaś, Wędrychowicz, 12.14.
% Banaś, Wędrychowicz, 12.15.
% Banaś, Wędrychowicz, 12.25.
\begin{problem}
    \label{banas_12_14}%
    $\int x^3 \sin x \, \mathrm{d}x$.
\end{problem}

\textbf{Problem \ref{banas_12_14}}. % SOLUTION
Całkujemy przez części, $f(x) = x^3$, $g'(x) = \sin x$. % SOLUTION
Dostajemy początek rekurencji: % SOLUTION
\begin{equation} % SOLUTION
    \int x^3 \sin x \, \mathrm{d}x = - x^3 \cos x - \int - 3x^2 \cos x \,\mathrm{d}x % SOLUTION
\end{equation} % SOLUTION
rozwiązaniem której jest $3 (x^2-2) \sin x + x (6-x^2) \cos x$. % SOLUTION

\begin{problem}
    % pomocnicza dla Banaś 12.19
    \label{banas_12_19_auxilia}%
    \begin{equation}
        \int e^x \sin x \,\mathrm{d}x = \frac {e^x} 2 (\sin x - \cos x).
    \end{equation}
\end{problem}

% Banaś, Wędrychowicz, 12.22.
% \begin{problem}
% Banaś-Wędrychowicz, 12.22. % x^2 e^x sin x
% \end{problem}

% Banaś, Wędrychowicz, 12.23.
% \begin{problem}
% Banaś-Wędrychowicz, 12.23. % x / (sin ^2 x)
% \end{problem}

% Banaś, Wędrychowicz, 12.24.
% \begin{problem}
% Banaś-Wędrychowicz, 12.24. % x arcsin x / (1 - x^2)
% \end{problem}

% Banaś, Wędrychowicz, 12.26.
\begin{problem}
    \label{banas_12_26}%
    \begin{equation}
        \int \frac{x \log(\sqrt{x^2+1}+x)}{\sqrt{x^2+1}} \,\mathrm{d}x = \sqrt{x^2+1} \arsinh x - x.
    \end{equation}
\end{problem}

\textbf{Problem \ref{banas_12_26}}. % SOLUTION
Zauważamy, że $\log(\sqrt{x^2+1} + x) = \arsinh x$, a następnie całkujemy przez części: $f(x) = \arsinh x$, $g'(x) = x / \sqrt{x^2+1}$, wtedy $f(x) = 1/\sqrt{x^2+1}$, $g(x) = \sqrt{x^2+1}$. % SOLUTION

% Banaś, Wędrychowicz, 12.27.
% \begin{problem}
% Banaś-Wędrychowicz, 12.27. % arc tg sqrt (x)
% \end{problem}

% Banaś, Wędrychowicz, 12.28.
% \begin{problem}
% Banaś-Wędrychowicz, 12.28. % x e ^ (arctg x) / (1+x^2)^1.5
% \end{problem}
%

\begin{problem}
    \label{banas_12_21_auxilia}%
    $I = \int \sec^3 x \,\mathrm{d}x$.
\end{problem}

\textbf{Problem \ref{banas_12_21_auxilia}}. % SOLUTION
Całkujemy najpierw przez części, $f(x) = \sec x$, $g'(x) = \sec^2 x$. % SOLUTION
Pamiętając, że $\tan^2 x = \sec^2 x - 1$, mamy % SOLUTION
\begin{align} % SOLUTION
    I & = \sec x \tan x - \int \sec x \tan^2 x \,\mathrm{d} x \\ % SOLUTION
        & = \sec x \tan x - I + \int \sec x \,\mathrm{d}x \\ % SOLUTION
    2I & = \sec x \tan x + \log |\tan x + \sec x|. % SOLUTION
\end{align} % SOLUTION

% %

\section{Całkowanie funkcji wymiernych}
% SOLUTION
\section{Całkowanie funkcji wymiernych}
% SOLUTION

\begin{problem}
\label{boros_4287}%
\begin{equation}
    \int_0^\infty \frac{x^n \,\mathrm{d}x}{(ax+b)^{m+1}}  = \frac{(-1)^{n+1} (-1-m)! \cdot n!}{a^{n+1} b^{m-n} (n-m)!}
\end{equation}
\end{problem}

% SOLUTION
\textbf{Problem \ref{boros_4287}} -- patrz \cite[s. 48-60]{boros04}
% SOLUTION

\begin{problem}
\label{frac_1_x3_1}%
\begin{equation}
    \int_0^\infty \frac{\mathrm{d}x}{x^3 - 1} = - \frac{\pi}{3\sqrt{3}}
\end{equation}
\end{problem}

% SOLUTION
\textbf{Problem \ref{frac_1_x3_1}} -- patrz \cite[s. 22]{nahin15}
% SOLUTION

\begin{problem}
    \label{experimental_mathematics_p258}%
\begin{equation}
    \int_0^\infty \frac{x^8-4x^6+9x^4-5x^2+1}{x^{12}-10x^{10}+37x^8-42x^6+26x^4-8x^2+1} \,\mathrm{d}x = \frac{\pi}{2}
\end{equation}
\end{problem}

% SOLUTION
\textbf{Problem \ref{experimental_mathematics_p258}} -- patrz \cite[s. 258]{bailey07}.
% SOLUTION

\subsection{Oszacowania liczby $\pi$}
\begin{problem}
\label{22_7_pi}%
\begin{equation}
    \int_0^1 \frac{x^4(1-x)^4}{1 + x^4} \,\mathrm{d}x = 22/7 - \pi.
\end{equation}
\end{problem}

Czasami używa się określenia całka Dalzella, ponieważ Donald Percy Dalzell \cite{dalzell44} jako pierwszy opublikował to cudo.
Ograniczając mianownik z dołu oraz góry przez $1$ oraz $2$ możemy dojść do wniosku, że
\begin{equation}
    \frac{22}{7} - \frac {1}{630} < \pi < \frac{22}{7} - \frac{1}{1260},
\end{equation}
a więc pomylić się o mniej niż $0.015\%$!

% TODO - na en.wiki jest łatwe w przepisaniu rozwiązanie
% SOLUTION
\textbf{Problem \ref{22_7_pi}} -- patrz \cite[s. 24]{nahin15}
% SOLUTION

% TODO: patrz też https://math.stackexchange.com/questions/1956/is-there-an-integral-that-proves-pi-333-106
Istnieje wiele uogólnień powyższego wyniku do innych przybliżeń liczby $\pi$, dobrym źródłem dalszych informacji, a może nawet inspiracji jest artykuł Lucasa \cite{lucas05}.
Na przykład:

\begin{problem}
\begin{equation}
    \int_0^1 \frac {x^8(1-x)^8 (25+816x^2)}{3164 (1+x^2)} \,\mathrm{d} x = \frac {355}{113} - \pi.
\end{equation}
\end{problem}

albo:

\begin{problem}
\begin{equation}
    \int_0^1 \frac{x^5 ( 1-x)^6 (197 + 462 x^2)}{530 (1+x^2)} \,\mathrm{d}x = \pi - \frac{333}{106}
\end{equation}
\end{problem}


%
% \input{sections/differential_binomial}
% \section{Całkowanie funkcji trygonometrycznych}
% Całkowanie funkcji trygonometrycznych
%

\section{Sztuczka Feynmana: różniczkowanie pod znakiem całki}
\section{Sztuczka Feynmana: różniczkowanie pod znakiem całki} % SOLUTION

% https://math.stackexchange.com/questions/942263/really-advanced-techniques-of-integration-definite-or-indefinite
\begin{problem}
    $\int_0^\infty \sin(x) / x \,\mathrm{d}x = \pi/2$.
\end{problem}

% TODO: przepisać całkę z s. 82, Nahin

\begin{problem}
    \label{nahin_holzweg}%
    Niech $a, b > 0$, wtedy
    \begin{equation}
        \int_{-\infty}^\infty \frac{\cos ax}{b^2 - x^4} \,\mathrm{d} x = \frac{\pi}{b} \sin (ab)
    \end{equation}
\end{problem}

\textbf{Problem \ref{nahin_holzweg}} -- \cite[s. 115, 375, 376]{nahin15}. % SOLUTION

\begin{problem}
    \label{nahin_datenautobahn}%
    Niech $a > b$, wtedy
    \begin{equation}
        \int_{-\infty}^\infty \frac{\cos ax}{b^4 - x^4} \,\mathrm{d} x = \frac{\pi}{2b^3} [\sin (ab) + \exp (-ab)]
    \end{equation}
\end{problem}

\textbf{Problem \ref{nahin_datenautobahn}} -- \cite[s. 115, 376]{nahin15}. % SOLUTION

% Nahin Inside interesting... page 83
\begin{problem}
    \begin{equation}
        \int_0^\infty \frac{\sin ax}{x e^{xy}} \,\mathrm{d}x = \pm \frac \pi 2 - \arctan \frac y a.
    \end{equation}
\end{problem}

\begin{problem}
    Niech $a > 0$.
    Wtedy
    \begin{equation}
        \int_0^\infty \frac{\sin ax}{x} \,\mathrm{d}x = \frac \pi 2.
    \end{equation}
\end{problem}

% Nahin Inside interesting... page 85
\begin{problem}[całka Frullaniego]
    \begin{equation}
        \int_0^\infty \frac{\arctan (ax) - \arctan (bx)}{x} \,\mathrm{d}x = \frac \pi 2 \log \frac a b.
    \end{equation}
\end{problem}

% Nahin Inside interesting... page 8x
\begin{problem}
    Niech $a, b > 0$.
    Wtedy
    \begin{equation}
        \int_0^\infty \frac{e^{-ax} - e^{-bx}}{x} \,\mathrm{d}x = \log \frac b a.
    \end{equation}
\end{problem}

% Nahin Inside interesting... page 89
\begin{problem}
    \begin{equation}
        \int_0^\infty \frac{\cos (ax) - \cos (bx)}{x^2} \,\mathrm{d}x = \frac \pi 2 (b - a).
    \end{equation}
\end{problem}

% Nahin Inside interesting... page 89
\begin{problem}
    \begin{equation}
        \int_0^\infty \frac{\cos (ax) - \cos (bx)}{x} \,\mathrm{d}x = \log \frac b a.
    \end{equation}
\end{problem}

% Nahin Inside interesting... page 89
\begin{problem}
    \label{nahin_kriegsrecht}
    \begin{equation}
        \int_0^\infty \frac{\log (a^2 x^2 + 1)}{x^2 + b^2} \,\mathrm{d}x = \frac \pi b \log (1 + ab)
    \end{equation}
\end{problem}

\textbf{Problem \ref{nahin_kriegsrecht}} -- \cite[s. 67]{nahin15} w szczególnym przypadku $a = b = 1$; \cite[s. 114, 375]{nahin15} w ogólności. % SOLUTION

% Nahin Inside interesting... page 91
\begin{problem}
    Niech $a \ge 0$, wtedy
    \begin{equation}
        \int_0^1 \frac{x^a - 1}{\log x} \,\mathrm{d}x = \log(1+a).
    \end{equation}
\end{problem}

% Nahin Inside interesting... page 92
\begin{problem}
    Niech $a \ge 0$, wtedy
    \begin{equation}
        \int_0^1 \frac{x^a - x^b}{\log x} \,\mathrm{d}x = \log \frac{1+a}{1+b}.
    \end{equation}
\end{problem}

% Nahin Inside interesting... page 96
\begin{problem}
    Niech $a > b$, wtedy
    \begin{equation}
        \int_0^\pi \frac{\mathrm{d}x} {a + b \cos x} = \frac{\pi}{\sqrt{a^2 - b^2}}.
    \end{equation}
\end{problem}

Nahin pisze, że poniższą całkę wyznaczył jako pierwszy włoski matematyk Ulisse Dini w 1878 roku i że ma ważne zastosowania w fizyce i inżynierii.
\index[persons]{Dini, Ulisse}%

\begin{problem}
    \label{nahin_dini}%
    Niech $a \ge 0$ będzie dowolną liczbą rzeczywistą.
    Wtedy
    \begin{equation}
        \int_0^\pi \log (1 - 2 a \cos x + a^2) \,\mathrm{d} x = \begin{cases}
            0, & \text{gdy } a^2 \le 1, \\
            2 \pi \log a & \textrm{w przeciwnym razie}.
        \end{cases}
    \end{equation}
\end{problem}

\textbf{Problem \ref{nahin_dini}} -- \cite[s. 109-112]{nahin15} % SOLUTION

%

\section{Zadania z turniejów całkowania}
	%

\subsection{MIT Integration BEE 2024} % SOLUTION

\begin{problem}[ćwierćfinał 2, problem 2]
    \begin{equation}
        I = \int_0^1 \frac 1 x \log (1 + x^2 + x^3 + x^4 + x^5 + x^6 + x^7 + x^9) \,\mathrm{d}x
    \end{equation}
\end{problem}

% https://math.stackexchange.com/questions/1617081/proving-an-integration-equality % SOLUTION
\begin{solution} % SOLUTION
    Łatwo widać, że szukana całka jest równa $I_2 + I_3 + I_4$, gdzie % SOLUTION
    \begin{align} % SOLUTION
        I_n & = \int_0^1 \frac {\log (1 + x^n)}{x} \,\mathrm{d}x \\ % SOLUTION
            & = \frac 1 n \int_0^1 \frac {\log (1 + x)}{x} \,\mathrm{d}x \\ % SOLUTION
            & = \frac 1 n \int_0^1 \frac 1 x \sum_{k=1}^\infty \frac{(-1)^{k-1}x^k}{k} \,\mathrm{d}x \\ % SOLUTION
            & = \frac 1 n \sum_{k=1}^\infty \frac{(-1)^{k-1}}{k^2} \\ % SOLUTION
            & = \frac {\pi^2}{12n}. % SOLUTION
    \end{align} % SOLUTION
\end{solution} % SOLUTION


%

\input{sections/makefile-solutions/theoretic}

\section{Trudne całki}
	%

\begin{problem_with_solution}
    \label{reuleaux_tetrahedron}%
    Czworościan Reuleaux to bryła będąca częścią wspólną czterech kul, których środki leżą w wierzchołkach czworościanu foremnego, a promienie są tej samej długości, co krawędzie tego czworościanu.
    Znaleźć objętość tej bryły,
    \begin{equation}
        V = \int_0^1
        \frac{
            8\sqrt{3}
        }{
            1 + 3t^2
        } - \frac{
            16 \sqrt{2} (3t+1) (4t^2 +t+1)^{3/2}
        }{
            (3t^2+1)(11t^2 + 2t + 3)^2
        } - \frac{
            \sqrt{2} (249 t^2 + 54t + 65)
        }{
            (11t^2 + 2t +3)^2
        } \,\mathrm{d} t.
    \end{equation}
\end{problem_with_solution}

% SOLUTION
\textbf{Problem \ref{reuleaux_tetrahedron}} -- patrz \url{https://mathworld.wolfram.com/ReuleauxTetrahedron.html}.
% SOLUTION

% https://math.stackexchange.com/questions/580521/generalizing-int-01-frac-arctan-sqrtx2-2-sqrtx2-2
\begin{problem_with_solution}[całka Ahmeda]
    \label{ahmed_integral}%
    \begin{equation}
        \int_0^1 \frac{\arctan \sqrt{x^2+2}}{(x^2+1) \sqrt{x^2+2}} \,\mathrm{d}x = \frac{5\pi^2}{96}.
    \end{equation}
\end{problem_with_solution}

% SOLUTION
\textbf{Problem \ref{ahmed_integral}} -- patrz praca Ahmeda Zafara \cite{ahmed02}.
\index[persons]{Zafar, Ahmed}
% SOLUTION

% TODO: https://mathworld.wolfram.com/images/gifs/FoxTrotMathTest.jpg
% TODO https://mathworld.wolfram.com/DefiniteIntegral.html

%
	\subsection{Prawie niemożliwe całki}
\subsection{Prawie niemożliwe całki} % SOLUTION
Wszystkie poniższe całki pojawiają się w książce Valeana \cite{valean19}.

\begin{problem_with_solution}
    \label{valean_grundpreis}%
    Niech $y \in (-1, 1)$.
    Wtedy
    \begin{equation}
        \int_0^1 \frac{\mathrm{d}x}{(1+yx) \sqrt{1-x^2}} = \frac{\arccos y}{\sqrt{1-y^2}}.
    \end{equation}
\end{problem_with_solution}

\begin{solution}[do problemu \ref{valean_grundpreis}] % SOLUTION
    Patrz \cite[s. 1]{valean19}. % SOLUTION
\end{solution} % SOLUTION

\begin{problem_with_solution}
    \label{valean_zeugenstand}%
    Niech $m, n$ będą liczbami naturalnymi.
    Wtedy
    \begin{equation}
        \int_0^1 x^m \log^n x \,\mathrm{d} x = \frac{(-1)^n \cdot n!}{(m+1)^{n+1}}.
    \end{equation}
\end{problem_with_solution}

\begin{solution}[do problemu \ref{valean_zeugenstand}] % SOLUTION
    Patrz \cite[s. 1]{valean19}. % SOLUTION
\end{solution} % SOLUTION

Niech $H_{n}^{(m)} = 1 + 1/2^m + \ldots + 1/n^m$ oznacza $n$-tą uogólnioną liczbę harmoniczną.

\begin{problem_with_solution}
    \label{valean_1_3}%
    Rozpatrujemy rodzinę całek
    \begin{equation}
        I_{k,n} := \int_0^1 x^{n-1} \log^k (1-x) \,\mathrm{d} x.
    \end{equation}
    Mamy:
    \begin{align}
        I_{1,n} & = - \frac{H_n}{n} \\
        I_{2,n} & = \frac{H_n^2 + H_n^{(2)}}{n} \\
        I_{3,n} & = - \frac{H_n^3 + 3H_nH_n^{(2)} + 2H_n^{(3)}}{n} \\
        I_{4,n} & = \frac{H_n^4 + 6H_n^2 H_n^{(2)} + 8H_nH_n^{(3)} + 3(H_n^{(2)})^2 + 6H_n^{(4)}}{n}.
    \end{align}
\end{problem_with_solution}

% (Valean nazywa to ,,four logarithmic integrals strongly connected with the league of harmonic series'').

\begin{solution}[do problemu \ref{valean_1_3}] % SOLUTION
    Patrz \cite[s. 2]{valean19}. % SOLUTION
\end{solution} % SOLUTION

\begin{problem_with_solution}
    \label{valean_1_5}%
    Niech $s > 0$ będzie liczbą rzeczywistą, zaś $\psi$ oznacza funkcję digamma.
    Wtedy
    \begin{align}
        \int_0^1 \frac{x^{s-1}}{x+1} \,\mathrm{d} x & = \psi(s) - \psi\left(\frac s2\right) - \log 2 \\
        \int_0^\infty e^{-sx} \tanh x \,\mathrm{d} x & = \frac 1 2 \left[\psi\left(\frac{s+2}{4}\right) - \psi \left(\frac s4 \right) - \frac 2 s\right]. 
    \end{align}
\end{problem_with_solution}

% (Valean nazywa to ,,a couple of practical definite integrals expressed in terms of the digamma function'').

\begin{solution}[do problemu \ref{valean_1_5}] % SOLUTION
    Patrz \cite[s. 3]{valean19}. % SOLUTION
\end{solution} % SOLUTION

\begin{problem_with_solution}
    \label{valean_1_7}%
    \begin{align}
        \int_0^1 \frac{1}{x} \log^2 (1+x) \,\mathrm{d}x & = \frac{1}{4} \zeta(3) \\
        \int_0^1 \frac{1}{x} \log (1+x) \log (1-x) \,\mathrm{d}x & = -\frac{5}{8} \zeta(3)
    \end{align}
\end{problem_with_solution}

% two little tricky classical logarithmic integrals

\begin{solution}[do problemu \ref{valean_1_7}] % SOLUTION
    Patrz \cite[s. 4]{valean19}. % SOLUTION
\end{solution} % SOLUTION

\begin{problem_with_solution}[]
    \label{valean_1_8}%
    \begin{align}
        \int_0^1 [\log(1+x) \log(1-x)]^2 \,\mathrm{d} x & =
        24 - 8 \zeta(2)- 8 \zeta(3) - \zeta(4) \\
        & + 8 \log(2)\zeta(2) + 8 \log(2)\zeta(3) \\
        & - 4 \log^2(2)\zeta(2) \\
        & - 24 \log(2) + 12 \log^2(2)- 4 \log^3(2) + \log^4(2); 
    \end{align}
\end{problem_with_solution}

% a special trio of integrals

\begin{solution}[do problemu \ref{valean_1_8}] % SOLUTION
    Patrz \cite[s. 4, 5]{valean19}. % SOLUTION
\end{solution} % SOLUTION

\begin{problem_with_solution}
    \label{valean_1_10}%
    Niech $n \ge 1$ będzie liczbą naturalną.
    Rozpatrujemy rodzinę całek z parametrem
    \begin{equation}
        I_n = \int_0^1 \frac 1 x \log(1-x) \log^{2n} x \log (1+x) \,\mathrm{d}x.
    \end{equation}
    Znaleźć $I_n$ lub, jeśli jest to za trudne, pokazać, że
    \begin{align}
        I_1 & = \frac 3 4 \zeta (2) \zeta (3) - \frac {27}{16} \zeta(5), \\
        I_2 & = \frac 9 4 \zeta (3) \zeta (4) + \frac{45}{4} \zeta(2) \zeta(5) - \frac{363}{16} \zeta (7), \\
        I_3 & = \frac{2835}{8} \zeta(2) \zeta (7) + \frac {135}{8} \zeta (3) \zeta (6) + \frac {675}{8} \zeta (4) \zeta (5) - \frac {22635}{32} \zeta (9).
    \end{align} 
\end{problem_with_solution}

% the evaluation of a class of logarithmic integrals using a slightly modified result from ,,Table of Integrals, Series and Products'' by I. S. Gradshteyn and I. M. Ryzhik together with a series result elementarily proved by Guy Bastien

\begin{solution}[do problemu \ref{valean_1_10}] % SOLUTION
    Patrz \cite[s. 6, 7]{valean19}. % SOLUTION
\end{solution} % SOLUTION

%
	%

\subsection{Znalezione na math.stackexchange.com}
\subsection{Znalezione na math.stackexchange.com} % SOLUTION
Wszystkie poniższe całki pojawiają się na stronie na math.stackexchange.com.

% https://math.stackexchange.com/q/562694
\begin{problem}[pytanie 562694]
    \label{stack_562694}%
    \begin{equation}
        \int_{-1}^1 \frac{1}{x} \sqrt{\frac{1+x}{1-x}} \log \frac{2x^2+2x+1}{2x^2-2x+1} \,\mathrm{d}x = 4 \pi \operatorname{arccot} \sqrt{\phi}.
    \end{equation}
\end{problem}

\begin{solution}[do problemu \ref{stack_562694}] % SOLUTION
    Patrz przypis\footnote{\url{https://math.stackexchange.com/questions/562694/}}. % SOLUTION
\end{solution} % SOLUTION

% TODO: https://math.stackexchange.com/a/942440

%

\chapter{Tłumaczenie numeracji}
Po lewej stronie numer zadania w \cite{banas_wedrychowicz}, po prawej numer tego samego zadaina u nas.

\begin{multicols}{3}
\begin{itemize}
    \item 12.1 -- \ref{banas_12_1}
    \item 12.2 -- \ref{banas_12_2}
    \item 12.3 -- \ref{banas_12_3}
    \item 12.4 -- patrz 12.3
    \item 12.5 -- patrz 12.3
    \item 12.6 -- \ref{banas_12_6}
    \item 12.7 -- \ref{banas_12_7}
    \item 12.8 -- \ref{banas_12_8}
    \item 12.9 -- \ref{banas_12_9}
    \item 12.10 -- \ref{banas_12_10}
    \item 12.11 -- \ref{banas_12_11}
    \item 12.12 -- \ref{banas_12_12}
    \item 12.13 -- \ref{banas_12_13}
    \item 12.14 -- \ref{banas_12_14}
    \item 12.15 -- patrz 12.14
    \item 12.16 -- patrz 12.12
    \item 12.17 -- patrz 12.12
    \item 12.18 -- \ref{banas_12_18}
    \item 12.19 -- \ref{banas_12_19}
    \item 12.20 -- \ref{banas_12_20}
    \item 12.21 -- \ref{banas_12_21}
    \item 12.22 -- pominięta
    \item 12.23 -- pominięta
    \item 12.24 -- pominięta
    \item 12.25 -- patrz 12.14
    \item 12.26 -- \ref{banas_12_26}
    % \item 12.27 -- \ref{banas_12_27}
    % \item 12.28 -- \ref{banas_12_28}
    % \item 12.29 -- \ref{banas_12_29}
    % \item 12.30 -- \ref{banas_12_30}
    % \item 12.31 -- \ref{banas_12_31}
    % \item 12.32 -- \ref{banas_12_32}
    % \item 12.33 -- \ref{banas_12_33}
    % \item 12.34 -- \ref{banas_12_34}
    % \item 12.35 -- \ref{banas_12_35}
    % \item 12.36 -- \ref{banas_12_36}
    % \item 12.37 -- \ref{banas_12_37}
    % \item 12.38 -- \ref{banas_12_38}
    % \item 12.39 -- \ref{banas_12_39}
    % \item 12.40 -- \ref{banas_12_40}
    % \item 12.41 -- \ref{banas_12_41}
    % \item 12.42 -- \ref{banas_12_42}
    % \item 12.43 -- \ref{banas_12_43}
    % \item 12.44 -- \ref{banas_12_44}
    % \item 12.45 -- \ref{banas_12_45}
    % \item 12.46 -- \ref{banas_12_46}
    % \item 12.47 -- \ref{banas_12_47}
    % \item 12.48 -- \ref{banas_12_48}
    % \item 12.49 -- \ref{banas_12_49}
    % \item 12.50 -- \ref{banas_12_50}
    % \item 12.51 -- \ref{banas_12_51}
    % \item 12.52 -- \ref{banas_12_52}
    % \item 12.53 -- \ref{banas_12_53}
    % \item 12.54 -- \ref{banas_12_54}
    % \item 12.55 -- \ref{banas_12_55}
    % \item 12.56 -- \ref{banas_12_56}
    % \item 12.57 -- \ref{banas_12_57}
    % \item 12.58 -- \ref{banas_12_58}
    % \item 12.59 -- \ref{banas_12_59}
    % \item 12.60 -- \ref{banas_12_60}
    % \item 12.61 -- \ref{banas_12_61}
    % \item 12.62 -- \ref{banas_12_62}
    % \item 12.63 -- \ref{banas_12_63}
    % \item 12.64 -- \ref{banas_12_64}
    % \item 12.65 -- \ref{banas_12_65}
    % \item 12.66 -- \ref{banas_12_66}
    % \item 12.67 -- \ref{banas_12_67}
    % \item 12.68 -- \ref{banas_12_68}
    % \item 12.69 -- \ref{banas_12_69}
    % \item 12.70 -- \ref{banas_12_70}
    % \item 12.71 -- \ref{banas_12_71}
    % \item 12.72 -- \ref{banas_12_72}
    % \item 12.73 -- \ref{banas_12_73}
    % \item 12.74 -- \ref{banas_12_74}
    % \item 12.75 -- \ref{banas_12_75}
    % \item 12.76 -- \ref{banas_12_76}
    % \item 12.77 -- \ref{banas_12_77}
    % \item 12.78 -- \ref{banas_12_78}
    % \item 12.79 -- \ref{banas_12_79}
    % \item 12.80 -- \ref{banas_12_80}
    % \item 12.81 -- \ref{banas_12_81}
    % \item 12.82 -- \ref{banas_12_82}
    % \item 12.83 -- \ref{banas_12_83}
    % \item 12.84 -- \ref{banas_12_84}
    % \item 12.85 -- \ref{banas_12_85}
    % \item 12.86 -- \ref{banas_12_86}
    % \item 12.87 -- \ref{banas_12_87}
    % \item 12.88 -- \ref{banas_12_88}
    % \item 12.89 -- \ref{banas_12_89}
    % \item 12.90 -- \ref{banas_12_90}
    % \item 12.91 -- \ref{banas_12_91}
    % \item 12.92 -- \ref{banas_12_92}
    % \item 12.93 -- \ref{banas_12_93}
    % \item 12.94 -- \ref{banas_12_94}
    % \item 12.95 -- \ref{banas_12_95}
    % \item 12.96 -- \ref{banas_12_96}
    % \item 12.97 -- \ref{banas_12_97}
    % \item 12.98 -- \ref{banas_12_98}
    % \item 12.99 -- \ref{banas_12_99}
    % \item 12.100 -- \ref{banas_12_100}
    % \item 12.101 -- \ref{banas_12_101}
    % \item 12.102 -- \ref{banas_12_102}
    % \item 12.103 -- \ref{banas_12_103}
    % \item 12.104 -- \ref{banas_12_104}
    % \item 12.105 -- \ref{banas_12_105}
    % \item 12.106 -- \ref{banas_12_106}
    % \item 12.107 -- \ref{banas_12_107}
    % \item 12.108 -- \ref{banas_12_108}
    % \item 12.109 -- \ref{banas_12_109}
    % \item 12.110 -- \ref{banas_12_110}
    % \item 12.111 -- \ref{banas_12_111}
    % \item 12.112 -- \ref{banas_12_112}
    % \item 12.113 -- \ref{banas_12_113}
    % \item 12.114 -- \ref{banas_12_114}
    % \item 12.115 -- \ref{banas_12_115}
    % \item 12.116 -- \ref{banas_12_116}
    % \item 12.117 -- \ref{banas_12_117}
    % \item 12.118 -- \ref{banas_12_118}
    % \item 12.119 -- \ref{banas_12_119}
    % \item 12.120 -- \ref{banas_12_120}
    % \item 12.121 -- \ref{banas_12_121}
    % \item 12.122 -- \ref{banas_12_122}
    % \item 12.123 -- \ref{banas_12_123}
    % \item 12.124 -- \ref{banas_12_124}
    % \item 12.125 -- \ref{banas_12_125}
    % \item 12.126 -- \ref{banas_12_126}
    % \item 12.127 -- \ref{banas_12_127}
    % \item 12.128 -- \ref{banas_12_128}
    % \item 12.129 -- \ref{banas_12_129}
    % \item 12.130 -- \ref{banas_12_130}
    % \item 12.131 -- \ref{banas_12_131}
    % \item 12.132 -- \ref{banas_12_132}
    % \item 12.133 -- \ref{banas_12_133}
    % \item 12.134 -- \ref{banas_12_134}
    % \item 12.135 -- \ref{banas_12_135}
    % \item 12.136 -- \ref{banas_12_136}
    % \item 12.137 -- \ref{banas_12_137}
    % \item 12.138 -- \ref{banas_12_138}
    % \item 12.139 -- \ref{banas_12_139}
    % \item 12.140 -- \ref{banas_12_140}
    % \item 12.141 -- \ref{banas_12_141}
    % \item 12.142 -- \ref{banas_12_142}
    % \item 12.143 -- \ref{banas_12_143}
    % \item 12.144 -- \ref{banas_12_144}
    % \item 12.145 -- \ref{banas_12_145}
    % \item 12.146 -- \ref{banas_12_146}
    % \item 12.147 -- \ref{banas_12_147}
    % \item 12.148 -- \ref{banas_12_148}
    % \item 12.149 -- \ref{banas_12_149}
    % \item 12.150 -- \ref{banas_12_150}
    % \item 12.151 -- \ref{banas_12_151}
    % \item 12.152 -- \ref{banas_12_152}
    % \item 12.153 -- \ref{banas_12_153}
    % \item 12.154 -- \ref{banas_12_154}
    % \item 12.155 -- \ref{banas_12_155}
    % \item 12.156 -- \ref{banas_12_156}
    % \item 12.157 -- \ref{banas_12_157}
    % \item 12.158 -- \ref{banas_12_158}
    % \item 12.159 -- \ref{banas_12_159}
    % \item 12.160 -- \ref{banas_12_160}
    % \item 12.161 -- \ref{banas_12_161}
    % \item 12.162 -- \ref{banas_12_162}
    % \item 12.163 -- \ref{banas_12_163}
    % \item 12.164 -- \ref{banas_12_164}
    % \item 12.165 -- \ref{banas_12_165}
    % \item 12.166 -- \ref{banas_12_166}
    % \item 12.167 -- \ref{banas_12_167}
    % \item 12.168 -- \ref{banas_12_168}
    % \item 12.169 -- \ref{banas_12_169}
    % \item 12.170 -- \ref{banas_12_170}
    % \item 12.171 -- \ref{banas_12_171}
    % \item 12.172 -- \ref{banas_12_172}
    % \item 12.173 -- \ref{banas_12_173}
    % \item 12.174 -- \ref{banas_12_174}
    % \item 12.175 -- \ref{banas_12_175}
    % \item 12.176 -- \ref{banas_12_176}
    % \item 12.177 -- \ref{banas_12_177}
    % \item 12.178 -- \ref{banas_12_178}
    % \item 12.179 -- \ref{banas_12_179}
    % \item 12.180 -- \ref{banas_12_180}
    % \item 12.181 -- \ref{banas_12_181}
    % \item 12.182 -- \ref{banas_12_182}
    % \item 12.183 -- \ref{banas_12_183}
    % \item 12.184 -- \ref{banas_12_184}
    % \item 12.185 -- \ref{banas_12_185}
    % \item 12.186 -- \ref{banas_12_186}
    % \item 12.187 -- \ref{banas_12_187}
    % \item 12.188 -- \ref{banas_12_188}
    % \item 12.189 -- \ref{banas_12_189}
    % \item 12.190 -- \ref{banas_12_190}
    % \item 12.191 -- \ref{banas_12_191}
    % \item 12.192 -- \ref{banas_12_192}
    % \item 12.193 -- \ref{banas_12_193}
    % \item 12.194 -- \ref{banas_12_194}
    % \item 12.195 -- \ref{banas_12_195}
    % \item 12.196 -- \ref{banas_12_196}
    % \item 12.197 -- \ref{banas_12_197}
    % \item 12.198 -- \ref{banas_12_198}
    % \item 12.199 -- \ref{banas_12_199}
\end{itemize}
\end{multicols}

%

\indexprologue{\small Tekst prologu I.}
\printindex

\indexprologue{\small Tekst prologu II.}
\printindex[persons]

\raggedright
\bibliography{integrals}{}
\bibliographystyle{plain}

\end{document}

% TODO: https://www.amazon.com/dp/0867202939
% TODO: $$\int_0^\infty \log(x) / (1+x^2) dx = 0$$ (Euler?)
% TODO: $$\int_0^\infty 1/(x^3 - 1) dx = -pi sqrt 3 / 9$$

% TODO: https://math.stackexchange.com/questions/tagged/integration?tab=votes&page=2&pagesize=15
