\documentclass[9pt, twoside, a5paper, fleqn]{extbook}
\usepackage[
	margin=10mm,
    headsep=5mm,
	twoside,
    includehead,
    % showframe,
]{geometry}

\usepackage{fancyhdr}
\fancypagestyle{plain}{
    \fancyhf{}
    \fancyfoot[C]{}
    \renewcommand{\headrulewidth}{0pt}
    \renewcommand{\footrulewidth}{0pt}
}
\pagestyle{fancy}
\fancyhead{}
\fancyfoot{}
% 341      Rozdział 7. Kalifat algebry | 7.2 Pierścień adeli              342 |
\fancyhead[LE]{\thepage}
\fancyhead[RE]{\nouppercase{\leftmark}}
\fancyhead[LO]{\nouppercase{\rightmark}}
\fancyhead[RO]{\thepage}
\renewcommand{\headrulewidth}{0.4pt}
\renewcommand{\footrulewidth}{0.0pt}

\usepackage{parskip}
\usepackage{multicol}
\usepackage{float} % [H] for figure environment
\usepackage{hyperref} % for \url and \href

\usepackage{polski}
\usepackage[T1]{fontenc}

\usepackage{xcolor}

\usepackage{amsmath, amsfonts, amssymb, amsthm}
\DeclareMathOperator{\arsinh}{arsinh}
\newcommand{\N}{{\mathbb N}}
\newcommand{\R}{{\mathbb R}}

\newcounter{counter}
% \numberwithin{counter}{chapter}
\newtheorem{proposition}[counter]{Fakt}
\newtheorem{problem}[counter]{Problem}
\newtheorem{exercise}{Exercise}[chapter]

\theoremstyle{remark}
\newtheorem*{solution}{Rozwiązanie}

\author{Imię Nazwisko}
\title{Tytuł książki}

\usepackage{imakeidx}
\usepackage{makeidx}
\usepackage{etoolbox} % for patchcmd
\patchcmd{\theindex}{\MakeUppercase\indexname}{\sffamily\normalsize\bfseries\indexname}{}{}
\makeindex[title=Skorowidz]
\makeindex[name=persons,title=Indeks osób]

\usepackage{Alegreya}

\begin{document}

% strona pierwsza

\thispagestyle{empty}
{\noindent\fontsize{18pt}{18pt}\selectfont Księgozbiór matemagiczny, tom ??}

\noindent\makebox[\linewidth]{\rule{\textwidth}{1pt}}

\newpage

% koniec strony pierwszej



% strona druga

\thispagestyle{empty}
\phantom{nothing}
\newpage

% koniec strony drugiej



% strona trzecia

\thispagestyle{empty}
{\noindent\fontsize{18pt}{18pt}\selectfont Imię Nazwisko}

\noindent\makebox[\linewidth]{\rule{\textwidth}{1pt}}

\vspace{10mm}

{\noindent\fontsize{24pt}{24pt}\selectfont \textbf{Tytuł\\(takie tam)}}
\vspace{10mm}

{\noindent\fontsize{14pt}{14pt}\selectfont Wydanie pierwsze}

\newpage

% koniec strony trzeciej



% strona czwarta

\thispagestyle{empty}
\begin{figure}[H]
\begin{minipage}[b]{.48\linewidth}
{\noindent Prof. Imię Nazwisko\\
Gdzie\\
Gdzie dalej\\
Gdzie kraj}
\end{minipage}
\end{figure}

{\noindent \textbf{Tytuł oryginału}\\Tytuł oryginału}
\vspace{5mm}

{\noindent \textbf{Okładkę zaprojektował}\\Okładka}
\vspace{5mm}

{\noindent \textbf{Zredagował}\\Redakcja}
\vspace{5mm}

{\noindent \textbf{Zredagowała technicznie}\\Redakcja techniczna}
\vspace{5mm}

{\noindent \textbf{Złożyli i połamali}\\Skład, łamanie}
\vspace{5mm}

{\noindent \textbf{Korekty dokonali}\\Korekta}

\vfill

{\noindent Copyleft by Antykwariat Czarnoksięski, Gorzów Wielkopolski 2024.
Książka, a także każda jej część, mogą być przedrukowywane oraz w jakikolwiek inny sposób reprodukowane czy powielane mechanicznie, fotooptycznie, zapisywane elektronicznie lub magnetycznie, oraz odczytywane w środkach publicznego przekazu bez pisemnej zgody wydawcy.}

\vspace{5mm}

{\noindent Przygotowano w systemie \TeX, wydrukowano na siarczystym papierze.}

% koniec strony czwartej



% strona piąta

\tableofcontents

\chapter*{Przedmowa}
Sam nie potrafiłbym wymyślić wszystkich całek, jakie zostały zebrane w czytanym przez Ciebie opracowaniu.
Wiele z zaprezentowanych tu problemów, ich rozwiązań (lub jednego i drugiego) zostało zaczerpnięte z popularnych zbiorów, takich jak \cite{wedrychowicz12} Banasia, Wędrychowicza.
Czasami autorzy proszą o zastosowanie wzoru na całkowanie przez części, kiedy zmyślne podstawienie prowadzi do celu krótszą drogą.
Nie przejmowaliśmy się zbytnio poleceniami z ich książki, ale żeby się w tym wszystkim nie pogubić, zamieszczamy na ostatnich stronach ,,tłumaczenie'' naszej numeracji na numerację z rozdziałów 12, 13 \cite{wedrychowicz12}.
Niektóre zadania nie zasługują na szczególną uwagę i choć ich treść została pominięta, nie brakuje ich we wspomniamym przed chwilą ,,tłumaczeniu''.

Gdzie to możliwe, staram się też proponować rozwiązania alternatywne, aby ukazać różnorodność narzędzi, które mogą być użyteczne dla osób całkujących.

% koniec strony piątej



\raggedbottom

\chapter{Pochodne}
%

\begin{proposition}
    Pochodna jest operatorem liniowym:
    \begin{equation}
        \frac{\mathrm{d}}{\mathrm{d}x} [a f(x) + b g(x)] = a \frac{\mathrm{d}}{\mathrm{d}x} [f(x)] + b \frac{\mathrm{d}}{\mathrm{d}x} [g(x)].
    \end{equation}
\end{proposition}

\begin{proposition}[reguła Leibniza]
    \begin{equation}
        \frac{\mathrm{d}}{\mathrm{d}x} [f(x)g(x)] =  g(x) \frac{\mathrm{d}}{\mathrm{d}x} [f(x)] + f(x)\frac{\mathrm{d}}{\mathrm{d}x} [g(x)].
    \end{equation}
\end{proposition}

\begin{proof}
    Dla oszczędności miejsca, $x_h := x + h$.
\begin{align}
    \frac{\mathrm{d}}{\mathrm{d}x} [f(x)g(x)]
    & = \lim_{h \to 0} \frac{f(x_h)g(x_h) - f(x)g(x)}{h} \\
    & = \lim_{h \to 0} \frac{f(x_h)g(x_h) - f(x)g(x_h) + f(x)g(x_h) - f(x)g(x)}{h} \\
    & = \lim_{h \to 0} \frac{[f(x_h) - f(x)]g(x_h) + f(x)[g(x_h) - g(x)]}{h} \\
    & = \lim_{h \to 0} \frac{f(x_h) - f(x)}{h} \lim_{h\to 0} g(x_h) + 
        \lim_{h \to 0} \frac{g(x_h) - g(x)}{h} \lim_{h \to 0} f(x) \\
    & = g(x) \frac{\mathrm{d}}{\mathrm{d}x} [f(x)] + f(x)\frac{\mathrm{d}}{\mathrm{d}x} [g(x)],
\end{align}
    ponieważ funkcje różniczkowalne są też ciągłe.
\end{proof}

\begin{proposition}
    \label{prp:derivative_power}%
    \begin{equation}
        \frac{\mathrm{d}}{\mathrm{d}x} x^n = nx^{n-1}.
    \end{equation}
\end{proposition}

\begin{proof}
\begin{align}
    \frac{\mathrm{d}}{\mathrm{d}x} x^n
    & = \lim_{h \to 0} \frac{(x+h)^n - x^n}{h} \\
    & = \lim_{h \to 0} \frac{1}{h} \left(\sum_{k=0}^n {n \choose k} x^k h^{n-k} - x^n \right) \\
    & = \lim_{h \to 0} \frac{1}{h} \left(nx^{n-1}h + \sum_{k=0}^{n-2} {n \choose k} x^k h^{n-k}\right) \\
    & = nx^{n-1} + \lim_{h \to 0} \left(\sum_{k=0}^{n-2} {n \choose k} x^k h^{n-k-1}\right) \\
    & = nx^{n-1}.
\end{align}
\end{proof}

%

\chapter{Całki}
%

\section{Całkowanie przez zgadnięcie pochodnej}

Czasami wystarczy zgadnąć wynik (albo znaleźć go w tablicy pochodnych) i sprawdzić, że pasuje przez zróżniczkowanie.

\begin{integral}
    Niech $n \neq -1$.
    Wtedy
    \begin{equation}
        \int x^n \,\mathrm{d}x = \frac{x^{n+1}}{n+1}.
    \end{equation}
\end{integral}

\begin{proof}
    Wprost z \ref{prp:derivative_power}.
\end{proof}

%
\section{Całkowanie przez podstawianie}

% Banaś, Wędrychowicz, 12.18.
\begin{integral}
    $\int (\arcsin x)^2 \,\mathrm{d}x$.
\end{integral}

\begin{solution}
    Podstawiamy $u = \arcsin x$ i dostajemy całkę z $u^2 \cos u$, którą rozwiązujemy przez części, tak jak w przykładzie \ref{banas_12_14}.
\end{solution}

% Banaś, Wędrychowicz, 12.19.
\begin{integral}
    $\int \sin(\log x) \, \mathrm{d}x$.
\end{integral}

\begin{solution}
    Podstawiamy $u = \log x$, $\mathrm{d} u = \mathrm{d} x / x$, $x = \exp u$ i dostajemy całkę z $e^u \sin u$, którą rozwiązujemy przez części, tak jak w przykładzie \ref{banas_12_19_auxilia}.
\end{solution}

% Banaś, Wędrychowicz, 12.20.
\begin{integral}
    $\int \cos(\log x) \, \mathrm{d}x$.
\end{integral}    

% Banaś, Wędrychowicz: 12.29
\begin{integral}
    Banaś-Wędrychowicz, 12.29.
\end{integral}

% Banaś, Wędrychowicz: 12.30
\begin{integral}
    Banaś-Wędrychowicz, 12.30.
\end{integral}

% Banaś, Wędrychowicz: 12.31
\begin{integral}
    Banaś-Wędrychowicz, 12.31.
\end{integral}

% Banaś, Wędrychowicz: 12.32
\begin{integral}
    Banaś-Wędrychowicz, 12.32.
\end{integral}

% Banaś, Wędrychowicz: 12.33
\begin{integral}
    Banaś-Wędrychowicz, 12.33.
\end{integral}

% Banaś, Wędrychowicz: 12.34
\begin{integral}
    Banaś-Wędrychowicz, 12.34.
\end{integral}

% Banaś, Wędrychowicz: 12.35
\begin{integral}
    Banaś-Wędrychowicz, 12.35.
\end{integral}

% Banaś, Wędrychowicz: 12.36
\begin{integral}
    Banaś-Wędrychowicz, 12.36.
\end{integral}

% Banaś, Wędrychowicz: 12.37
\begin{integral}
    Banaś-Wędrychowicz, 12.37.
\end{integral}

% Banaś, Wędrychowicz: 12.38
\begin{integral}
    Banaś-Wędrychowicz, 12.38.
\end{integral}

% Banaś, Wędrychowicz: 12.39
\begin{integral}
    Banaś-Wędrychowicz, 12.39.
\end{integral}

% Banaś, Wędrychowicz: 12.40
\begin{integral}
    Banaś-Wędrychowicz, 12.40.
\end{integral}

% Banaś, Wędrychowicz: 12.41
\begin{integral}
    Banaś-Wędrychowicz, 12.41.
\end{integral}

% Banaś, Wędrychowicz: 12.42
\begin{integral}
    Banaś-Wędrychowicz, 12.42.
\end{integral}

% Banaś, Wędrychowicz: 12.43
\begin{integral}
    Banaś-Wędrychowicz, 12.43.
\end{integral}

% Banaś, Wędrychowicz: 12.44
\begin{integral}
    Banaś-Wędrychowicz, 12.44.
\end{integral}

% Banaś, Wędrychowicz: 12.45
\begin{integral}
    Banaś-Wędrychowicz, 12.45.
\end{integral}

% Banaś, Wędrychowicz: 12.46
\begin{integral}
    Banaś-Wędrychowicz, 12.46.
\end{integral}

% Banaś, Wędrychowicz: 12.47
\begin{integral}
    Banaś-Wędrychowicz, 12.47.
\end{integral}

% Banaś, Wędrychowicz: 12.48
\begin{integral}
    Banaś-Wędrychowicz, 12.48.
\end{integral}

% Banaś, Wędrychowicz: 12.49
\begin{integral}
    Banaś-Wędrychowicz, 12.49.
\end{integral}

% Banaś, Wędrychowicz: 12.50
\begin{integral}
    Banaś-Wędrychowicz, 12.50.
\end{integral}

% Banaś, Wędrychowicz: 12.51
\begin{integral}
    Banaś-Wędrychowicz, 12.51.
\end{integral}

% Banaś, Wędrychowicz: 12.52
\begin{integral}
    Banaś-Wędrychowicz, 12.52.
\end{integral}

% Banaś, Wędrychowicz: 12.53
\begin{integral}
    Banaś-Wędrychowicz, 12.53.
\end{integral}

% Banaś, Wędrychowicz: 12.54
\begin{integral}
    Banaś-Wędrychowicz, 12.54.
\end{integral}

% Banaś, Wędrychowicz: 12.55
\begin{integral}
    Banaś-Wędrychowicz, 12.55.
\end{integral}

% Banaś, Wędrychowicz: 12.56
\begin{integral}
    Banaś-Wędrychowicz, 12.56.
\end{integral}

% Banaś, Wędrychowicz: 12.57
\begin{integral}
    Banaś-Wędrychowicz, 12.57.
\end{integral}

% Banaś, Wędrychowicz: 12.58
\begin{integral}
    Banaś-Wędrychowicz, 12.58.
\end{integral}

\subsection{Podstawienia Eulera}

TODO: Banaś Wędrychowicz, 12.71 - 12.87

\begin{integral}
    Banaś-Wędrychowicz, 12.58.
\end{integral}

% https://math.stackexchange.com/questions/541751/how-prove-this-i-int-0-infty-frac1x-ln-left-frac1x1-x-right2/541861#541861
\begin{integral}[pytanie 541751 na math.stackexchange.com]
    \begin{equation}
        I = \int_0^\infty \frac{1}{x} \log \left(\frac{1+x}{1-x}\right)^2 \,\mathrm{d}x = \pi^2
    \end{equation}
\end{integral}

\begin{proof}
    Podstawiamy $y = (1+x) / (1-y)$:
    \begin{align}
        I & = 2 \int_{-1}^1 \frac{\log y^2}{1-y^2} \,\mathrm{d}y \\
          & = 8 \int_0^1 \frac{\log y}{1-y^2} \, \mathrm{d}{y} \\
          & = 8 \sum_{k=0}^\infty \int_0^1 y^{2k} \log y \,\mathrm{d} y \\
          & = 8 \sum_{k=0}^\infty \frac{1}{(2k+1)^2} \\
          & = 8 \cdot \frac{\pi^2}{8} = 8.
    \end{align}
\end{proof}
%

\section{Całkowanie przez części}

\begin{proposition}[wzór na całkowanie przez części]
\label{prp_int_by_parts}%
    Jeśli funkcje $f, g \colon I \to \R$ są różniczkowalne, to
    \begin{equation}
        \int f(x) g'(x) \,\mathrm{d}x = f(x) g(x) - \int f'(x) g(x) \,\mathrm{d} x.
    \end{equation}
\end{proposition}

\begin{proof}
    Całkujemy obie strony wzoru na pochodną iloczynu $(fg)' = fg' + f'g$, a następnie porządkujemy strony równości.
\end{proof}

\begin{integral}
    $\int x \sin x \,\mathrm{d} x = \ldots$
\end{integral}

\begin{proof}
    Całkujemy przez części, $f(x) = x$, $g'(x) = \sin x$.
    \begin{align}
        \int x \sin x \,\mathrm{d} x & = -x \cos x - \int - \cos x \, \mathrm{d}x \\
                                     & = -x \cos x + \sin x.
    \end{align}
\end{proof}

Analogicznie obliczamy całki:

\begin{multicols}{2}
\begin{integral}
    $\int x \cos x \,\mathrm{d} x = \ldots$
\end{integral}

\begin{integral}
    $\int x \exp x \,\mathrm{d} x = \ldots$
\end{integral}
\end{multicols}

\begin{integral}
    $\int x \arctan x \,\mathrm{d} x$.
\end{integral}

\begin{proof}
    Całkujemy przez części, $f(x) = \arctan x$, $g'(x) = x$.
    \begin{align}
        \int x \arctan x \, \mathrm{d} x & = \frac 12 x^2 \arctan x - \int \frac{x^2 \,\mathrm{d}x}{2(x^2+1)} \\
                                         & = \frac 12 x^2 \arctan x - \frac 12 \left(\int 1 \,\mathrm{d}x - \int \frac{\mathrm{d}x}{x^2+1} \right) \\
                                         & = \frac 12 x^2 \arctan x - \frac 12 \left(x - \arctan x \right) \\
                                         & = \frac 12 \left((x^2+1)\arctan x - x \right).
    \end{align}
\end{proof}

\begin{integral}
    $\int \arccos x \,\mathrm{d} x$.
\end{integral}

\begin{proof}
    Całkujemy najpierw przez części, $f(x) = \arccos x$, $g'(x) = 1$, a potem podstawiamy $u = 1 - x^2$, $\mathrm{d} u = -2x \mathrm{d}x$:
    \begin{align}
        \int \arccos x \, \mathrm{d} x & = x \arccos x - \int  \frac{-x \,\mathrm{d}x}{\sqrt{1-x^2}} \\
        & = x \arccos x - \frac 12 \int \frac {\mathrm{d}u}{\sqrt{u}} \\
        & = x \arccos x - \sqrt{1 - x^2}.
    \end{align}
\end{proof}

\begin{integral}
    $\int x^n \log x \,\mathrm{d} x = \ldots$, gdzie $n \in \N$.
\end{integral}

\begin{proof}
    Całkujemy przez części, $f(x) = \log x$, $g'(x) = x^n$.
    \begin{align}
        \int x^n \log x \, \mathrm{d} x & = \frac{x^{n+1} \log x}{n+1} - \int \frac{x^n \,\mathrm{d} x}{n+1} \\
                                        & = \frac{x^{n+1} \log x}{n+1} - \frac{x^{n+1}}{(n+1)^2}.
    \end{align}
\end{proof}

%
%

\section{Całkowanie funkcji wymiernych}
Całkowanie funkcji wymiernych

\begin{problem}
\begin{equation}
    \int_0^\infty \frac{x^n \,\mathrm{d}x}{(ax+b)^{m+1}}  = \frac{(-1)^{n+1} (-1-m)! \cdot n!}{a^{n+1} b^{m-n} (n-m)!}
\end{equation}
\end{problem}

\begin{solution}
    Rozdział 3 Borosa, strony 48-60.
\end{solution}

\begin{problem}
\begin{equation}
    \int_0^\infty \frac{\mathrm{d}x}{x^3 - 1} = - \frac{\pi}{3\sqrt{3}}
\end{equation}
\end{problem}

\begin{solution}
    Strona 22 w Inside Interesting Integrals.
\end{solution}


Całka Donalda Percy'ego Dalzella:
\begin{problem}
\begin{equation}
    \int_0^1 \frac{x^4(1-x)^4}{1 + x^4} \,\mathrm{d}x = 22/7 - \pi
\end{equation}
\end{problem}

\begin{solution}
    Strona 24 w Inside Interesting Integrals.
\end{solution}


\begin{problem}
\begin{equation}
    \int_0^\infty \frac{x^8-4x^6+9x^4-5x^2+1}{x^{12}-10x^{10}+37x^8-42x^6+26x^4-8x^2+1} \,\mathrm{d}x = \frac{\pi}{2}
\end{equation}
\end{problem}

\begin{solution}
    Strona 258 w: Bailey, D. H.; Borwein, J. M.; Calkin, N. J.; Girgensohn, R.; Luke, D. R.; and Moll, V. H. Experimental Mathematics in Action. Wellesley, MA: A K Peters, 2007.
\end{solution}

%

\begin{problem}
    \begin{equation}
        \int \sec x \,\mathrm{d} x = \log| \sec x + \tan x|.
    \end{equation}
\end{problem}

% https://en.wikipedia.org/wiki/Integral_of_the_secant_function#History
W 1599 roku Edward Wright znalazł wartość tej całki metodami numerycznymi (było mu to potrzebne by dokładnie skonstruować odwzorowanie walcowe wiernokątne -- Mercatora).
\index{odwzorowanie Mercatora}%
\index[persons]{Wright, Edward}%
W latach czterdziestych XVI wieku Henry Bond porównał wyniki Wrighta z tablicami logarytmicznymi i postawił hipotezę co do zwartej postaci całki.
\index[persons]{Bond, Henry}%
Pierwsze rozwiązanie podał szkocki matematyk i astronom James Gregory w pracy \emph{,,Exercitationes Geometricae''} z 1668 roku, ale było tak trudne w zrozumieniu (chociaż poprawne; podstawił $u = \sec \theta + \tan \theta$), że angielski teolog i matematyk Isaac Barrow zaproponował w \emph{,,Lectiones Geometricae''} z 1670 roku zupełnie inne podejście.
\index[persons]{Gregory, James}%
\index[persons]{Barrow, Isaac}%
Dowód Barrowa jest znany przede wszystkim dlatego, że zawiera najstarszy znany rozkład na ułamki proste podczas całkowania (!).
\index{rozkład na ułamki proste}%

\begin{problem}[sinus całkowy]
    \begin{equation}
        \int_0^\infty \frac {\sin x}{x} \,\mathrm{d} x = \frac \pi 2.
    \end{equation}
\end{problem}

% TODO: https://math.stackexchange.com/questions/13344/proof-of-int-0-infty-left-frac-sin-xx-right2-mathrm-dx-frac-pi2
\begin{problem}
    \begin{equation}
        \int_0^\infty \left(\frac {\sin x}{x}\right)^2 \,\mathrm{d} x = \frac \pi 2.
    \end{equation}
\end{problem}

% TODO: https://math.stackexchange.com/questions/9286/evaluation-of-gaussian-integral-int-0-infty-mathrme-x2-dx
\begin{problem}
    \begin{equation}
        I = \int_{-\infty}^\infty \exp \left( -x^2 \right) \,\mathrm{d} x = \sqrt{\pi}.
    \end{equation}
\end{problem}

% https://math.stackexchange.com/questions/34767/int-infty-infty-e-x2-dx-with-complex-analysis
Podstawiamy $u = x^2$, $\mathrm{d} u = 2x \,\mathrm{d}x$.
Wtedy
\begin{align}
	I = \int_0^\infty u^{-1/2} e^{-u}\,\mathrm{d}u = \Gamma \left(\frac 12\right) = \sqrt{\pi}
\end{align}
na mocy wzoru $\Gamma (z) \Gamma(1-z) = \pi/\sin \pi z$.

%

\section{Całkowanie różniczek dwumiennych} % https://encyclopediaofmath.org/wiki/Differential_binomial
Całkowanie różniczek dwumiennych

%
% \section{Całkowanie funkcji trygonometrycznych}
% Całkowanie funkcji trygonometrycznych
	%

\section{Sztuczka Feynmana: różniczkowanie pod znakiem całki}
% SOLUTION
\section{Sztuczka Feynmana: różniczkowanie pod znakiem całki}
% SOLUTION

% https://math.stackexchange.com/questions/942263/really-advanced-techniques-of-integration-definite-or-indefinite
\begin{problem}
    $\int_0^\infty \sin(x) / x \,\mathrm{d}x = \pi/2$.
\end{problem}

% TODO: przepisać całkę z s. 82, Nahin

\begin{problem_with_solution}
    \label{nahin_holzweg}%
    Niech $a, b > 0$, wtedy
    \begin{equation}
        \int_{-\infty}^\infty \frac{\cos ax}{b^2 - x^4} \,\mathrm{d} x = \frac{\pi}{b} \sin (ab)
    \end{equation}
\end{problem_with_solution}

% SOLUTION
\textbf{Problem \ref{nahin_holzweg}} -- \cite[s. 115, 375, 376]{nahin15}.
% SOLUTION

\begin{problem}
    \label{nahin_datenautobahn}%
    Niech $a > b$, wtedy
    \begin{equation}
        \int_{-\infty}^\infty \frac{\cos ax}{b^4 - x^4} \,\mathrm{d} x = \frac{\pi}{2b^3} [\sin (ab) + \exp (-ab)]
    \end{equation}
\end{problem}

% SOLUTION
\textbf{Problem \ref{nahin_datenautobahn}} -- \cite[s. 115, 376]{nahin15}.
% SOLUTION

% Nahin Inside interesting... page 83
\begin{problem}
    \begin{equation}
        \int_0^\infty \frac{\sin ax}{x e^{xy}} \,\mathrm{d}x = \pm \frac \pi 2 - \arctan \frac y a.
    \end{equation}
\end{problem}

\begin{problem}
    Niech $a > 0$.
    Wtedy
    \begin{equation}
        \int_0^\infty \frac{\sin ax}{x} \,\mathrm{d}x = \frac \pi 2.
    \end{equation}
\end{problem}




\begin{problem}[całka Frullaniego]
\index{całka Frullaniego}%
    Niech $f \colon [0, \infty) \to \R$ będzie funkcją ciągle różniczkowalną, której granica w nieskończoności istnieje.
    Wtedy dla ustalonych liczb rzeczywistych $a, b > 0$ mamy
    \begin{equation}
        \int_0^\infty \frac{f(ax) - f(bx)}{x} \,\mathrm{d} x = \left[\lim_{x \to \infty} f(x) - f(0) \right] \cdot \log \frac a b.
    \end{equation}
\end{problem}

Okazuje się, że problem rozwiązał Cauchy (około 1823 roku), ale też Giuliano Frullani, matematyk włoski (zapowiedź w 1821 roku, publikacja około 1829 roku).
\index[persons]{Frullani, Giuliano}%
Nahin \cite[s. 85]{nahin85} używa tej nazwy do konkretnego wcielenia całki Frullaniego, dla $f = \arctan$.
Boros, Moll \cite[s. 98]{boros94} piszą mgliście ,,under some mild conditions on the function $f$''...

% Nahin Inside interesting... page 8x
\begin{problem}
    Niech $a, b > 0$.
    Wtedy
    \begin{equation}
        \int_0^\infty \frac{e^{-ax} - e^{-bx}}{x} \,\mathrm{d}x = \log \frac b a.
    \end{equation}
\end{problem}

% Nahin Inside interesting... page 89
\begin{problem}
    \begin{equation}
        \int_0^\infty \frac{\cos (ax) - \cos (bx)}{x^2} \,\mathrm{d}x = \frac \pi 2 (b - a).
    \end{equation}
\end{problem}

% Nahin Inside interesting... page 89
\begin{problem}
    \begin{equation}
        \int_0^\infty \frac{\cos (ax) - \cos (bx)}{x} \,\mathrm{d}x = \log \frac b a.
    \end{equation}
\end{problem}

% Nahin Inside interesting... page 89
\begin{problem}
    \label{nahin_kriegsrecht}
    \begin{equation}
        \int_0^\infty \frac{\log (a^2 x^2 + 1)}{x^2 + b^2} \,\mathrm{d}x = \frac \pi b \log (1 + ab)
    \end{equation}
\end{problem}

% SOLUTION
\textbf{Problem \ref{nahin_kriegsrecht}} -- \cite[s. 67]{nahin15} w szczególnym przypadku $a = b = 1$; \cite[s. 114, 375]{nahin15} w ogólności.
% SOLUTION

% Nahin Inside interesting... page 91
\begin{problem}
    Niech $a \ge 0$, wtedy
    \begin{equation}
        \int_0^1 \frac{x^a - 1}{\log x} \,\mathrm{d}x = \log(1+a).
    \end{equation}
\end{problem}

% Nahin Inside interesting... page 92
\begin{problem}
    Niech $a \ge 0$, wtedy
    \begin{equation}
        \int_0^1 \frac{x^a - x^b}{\log x} \,\mathrm{d}x = \log \frac{1+a}{1+b}.
    \end{equation}
\end{problem}

% Nahin Inside interesting... page 96
\begin{problem}
    Niech $a > b$, wtedy
    \begin{equation}
        \int_0^\pi \frac{\mathrm{d}x} {a + b \cos x} = \frac{\pi}{\sqrt{a^2 - b^2}}.
    \end{equation}
\end{problem}

\begin{problem}
    \label{nahin_dini}%
    Niech $a \ge 0$ będzie dowolną liczbą rzeczywistą.
    Wtedy
    \begin{equation}
        \int_0^\pi \log (1 - 2 a \cos x + a^2) \,\mathrm{d} x = \begin{cases}
            0, & \text{gdy } a^2 \le 1, \\
            2 \pi \log a & \textrm{w przeciwnym razie}.
        \end{cases}
    \end{equation}
\end{problem}

Nahin pisze, że powyższą całkę wyznaczył jako pierwszy włoski matematyk Ulisse Dini w 1878 roku i że ma ważne zastosowania w fizyce i inżynierii.
\index[persons]{Dini, Ulisse}%

% SOLUTION
\textbf{Problem \ref{nahin_dini}} -- \cite[s. 109-112]{nahin15}
% SOLUTION

% TODO: https://math.stackexchange.com/questions/9402/calculating-the-integral-int-0-infty-frac-cos-x1x2-mathrmdx-with
\begin{equation}
    I = \int_0^\infty \frac {\cos x}{1+x^2} \,\mathrm{d}x
\end{equation}


%

\section{Zadania z turniejów całkowania}
	%

\subsection{MIT Integration BEE 2024} % SOLUTION

\begin{problem}[ćwierćfinał 2, problem 2]
    \begin{equation}
        I = \int_0^1 \frac 1 x \log (1 + x^2 + x^3 + x^4 + x^5 + x^6 + x^7 + x^9) \,\mathrm{d}x
    \end{equation}
\end{problem}

% https://math.stackexchange.com/questions/1617081/proving-an-integration-equality % SOLUTION
\begin{solution} % SOLUTION
    Łatwo widać, że szukana całka jest równa $I_2 + I_3 + I_4$, gdzie % SOLUTION
    \begin{align} % SOLUTION
        I_n & = \int_0^1 \frac {\log (1 + x^n)}{x} \,\mathrm{d}x \\ % SOLUTION
            & = \frac 1 n \int_0^1 \frac {\log (1 + x)}{x} \,\mathrm{d}x \\ % SOLUTION
            & = \frac 1 n \int_0^1 \frac 1 x \sum_{k=1}^\infty \frac{(-1)^{k-1}x^k}{k} \,\mathrm{d}x \\ % SOLUTION
            & = \frac 1 n \sum_{k=1}^\infty \frac{(-1)^{k-1}}{k^2} \\ % SOLUTION
            & = \frac {\pi^2}{12n}. % SOLUTION
    \end{align} % SOLUTION
\end{solution} % SOLUTION


%

%

\section{Teoretyczna teoria}
\section{Teoretyczna teoria} % SOLUTION

\begin{problem}[problem B4 na egzaminie Putnam 1968]
    \label{putnam_1968_b4}%
    Niech $f \colon \R \to \R$ będzie ciągłą funkcją taką, że całka $\int_\R f(x)\,\mathrm{d}x$ istnieje.
    Pokazać, że całka
    \begin{equation}
        \int_\R f\left(x - \frac 1 x\right)\,\mathrm{d}x
    \end{equation}
    też istnieje i przyjmuje tę samą wartość.
\end{problem}

% SOLUTION
\begin{solution}[do problemu \ref{putnam_1968_b4}]
    Będziemy całkować przez podstawienie, $x = \exp \theta$ (i potem $x = - \exp -\theta$):
    \begin{align}
        \int_{-\infty}^{\infty}f\left(x-x^{-1}\right)dx&=\int_{0}^{\infty}f\left(x-x^{-1}\right)dx+\int_{-\infty}^{0}f\left(x-x^{-1}\right)dx=\\
        &=\int_{-\infty}^{\infty}f(2\sinh\theta)\,e^{\theta}d\theta+\int_{-\infty}^{\infty}f(2\sinh\theta)\,e^{-\theta}d\theta=\\
        &=\int_{-\infty}^{\infty}f(2\sinh\theta)\,2\cosh\theta\,d\theta=\\
        &=\int_{-\infty}^{\infty}f(x)\,dx.
    \end{align}
\end{solution}
% SOLUTION

%

\section{Trudne całki}
	%

\begin{problem_with_solution}
    \label{reuleaux_tetrahedron}%
    Czworościan Reuleaux to bryła będąca częścią wspólną czterech kul, których środki leżą w wierzchołkach czworościanu foremnego, a promienie są tej samej długości, co krawędzie tego czworościanu.
    Znaleźć objętość tej bryły,
    \begin{equation}
        V = \int_0^1
        \frac{
            8\sqrt{3}
        }{
            1 + 3t^2
        } - \frac{
            16 \sqrt{2} (3t+1) (4t^2 +t+1)^{3/2}
        }{
            (3t^2+1)(11t^2 + 2t + 3)^2
        } - \frac{
            \sqrt{2} (249 t^2 + 54t + 65)
        }{
            (11t^2 + 2t +3)^2
        } \,\mathrm{d} t.
    \end{equation}
\end{problem_with_solution}

% SOLUTION
\textbf{Problem \ref{reuleaux_tetrahedron}} -- patrz \url{https://mathworld.wolfram.com/ReuleauxTetrahedron.html}.
% SOLUTION

% TODO: https://mathworld.wolfram.com/images/gifs/FoxTrotMathTest.jpg
% TODO https://mathworld.wolfram.com/DefiniteIntegral.html

%
	\subsection{Prawie niemożliwe całki}
% SOLUTION
\subsection{Prawie niemożliwe całki}
% SOLUTION
Wszystkie poniższe całki pojawiają się w książce Valeana \cite{valean19}.

\begin{problem_with_solution}
    \label{valean_grundpreis}%
    Niech $y \in (-1, 1)$.
    Wtedy
    \begin{equation}
        \int_0^1 \frac{\mathrm{d}x}{(1+yx) \sqrt{1-x^2}} = \frac{\arccos y}{\sqrt{1-y^2}}.
    \end{equation}
\end{problem_with_solution}

% SOLUTION
\textbf{Problem \ref{valean_grundpreis}} -- 
patrz \cite[s. 1]{valean19}.
% SOLUTION

\begin{problem_with_solution}
    \label{valean_zeugenstand}%
    Niech $m, n$ będą liczbami naturalnymi.
    Wtedy
    \begin{equation}
        \int_0^1 x^m \log^n x \,\mathrm{d} x = \frac{(-1)^n \cdot n!}{(m+1)^{n+1}}.
    \end{equation}
\end{problem_with_solution}

% SOLUTION
\begin{solution}[do problemu \ref{valean_zeugenstand}]
    Patrz \cite[s. 1]{valean19}.
\end{solution}
% SOLUTION


Niech $H_{n}^{(m)} = 1 + 1/2^m + \ldots + 1/n^m$ oznacza $n$-tą uogólnioną liczbę harmoniczną.

\begin{problem_with_solution}
    \label{valean_1_3}%
    Rozpatrujemy rodzinę całek
    \begin{equation}
        I_{k,n} := \int_0^1 x^{n-1} \log^k (1-x) \,\mathrm{d} x.
    \end{equation}
    Mamy:
    \begin{align}
        I_{1,n} & = - \frac{H_n}{n} \\
        I_{2,n} & = \frac{H_n^2 + H_n^{(2)}}{n} \\
        I_{3,n} & = - \frac{H_n^3 + 3H_nH_n^{(2)} + 2H_n^{(3)}}{n} \\
        I_{4,n} & = \frac{H_n^4 + 6H_n^2 H_n^{(2)} + 8H_nH_n^{(3)} + 3(H_n^{(2)})^2 + 6H_n^{(4)}}{n}.
    \end{align}
\end{problem_with_solution}

% (Valean nazywa to ,,four logarithmic integrals strongly connected with the league of harmonic series'').

% SOLUTION
\begin{solution}[do problemu \ref{valean_1_3}]
    Patrz \cite[s. 2]{valean19}.
\end{solution}
% SOLUTION

\begin{problem_with_solution}
    \label{valean_1_5}%
    Niech $s > 0$ będzie liczbą rzeczywistą, zaś $\psi$ oznacza funkcję digamma.
    Wtedy
    \begin{align}
        \int_0^1 \frac{x^{s-1}}{x+1} \,\mathrm{d} x & = \psi(s) - \psi\left(\frac s2\right) - \log 2 \\
        \int_0^\infty e^{-sx} \tanh x \,\mathrm{d} x & = \frac 1 2 \left[\psi\left(\frac{s+2}{4}\right) - \psi \left(\frac s4 \right) - \frac 2 s\right]. 
    \end{align}
\end{problem_with_solution}

% (Valean nazywa to ,,a couple of practical definite integrals expressed in terms of the digamma function'').

% SOLUTION
\begin{solution}[do problemu \ref{valean_1_5}]
    Patrz \cite[s. 3]{valean19}.
\end{solution}
% SOLUTION

\begin{problem_with_solution}
    \label{valean_1_7}%
    \begin{align}
        \int_0^1 \frac{1}{x} \log^2 (1+x) \,\mathrm{d}x & = \frac{1}{4} \zeta(3) \\
        \int_0^1 \frac{1}{x} \log (1+x) \log (1-x) \,\mathrm{d}x & = -\frac{5}{8} \zeta(3)
    \end{align}
\end{problem_with_solution}

% two little tricky classical logarithmic integrals

% SOLUTION
\begin{solution}[do problemu \ref{valean_1_7}]
    Patrz \cite[s. 4]{valean19}.
\end{solution}
% SOLUTION

\begin{problem_with_solution}[]
    \label{valean_1_8}%
    \begin{align}
        \int_0^1 [\log(1+x) \log(1-x)]^2 \,\mathrm{d} x & =
        24 - 8 \zeta(2)- 8 \zeta(3) - \zeta(4) \\
        & + 8 \log(2)\zeta(2) + 8 \log(2)\zeta(3) \\
        & - 4 \log^2(2)\zeta(2) \\
        & - 24 \log(2) + 12 \log^2(2)- 4 \log^3(2) + \log^4(2); 
    \end{align}
\end{problem_with_solution}

% a special trio of integrals

% SOLUTION
\begin{solution}[do problemu \ref{valean_1_8}]
    Patrz \cite[s. 4, 5]{valean19}.
\end{solution}
% SOLUTION

% TODO: https://math.stackexchange.com/questions/3413586/conjectural-closed-form-of-int-01-frac-logn-1-x-logn-1-1x1x-d

\begin{problem_with_solution}
    \label{valean_1_10}%
    Niech $n \ge 1$ będzie liczbą naturalną.
    Znaleźć
    \begin{equation}
        I_n = \int_0^1 \frac 1 x \log(1-x) \log^{2n} x \log (1+x) \,\mathrm{d}x.
    \end{equation}
    Jeśli jest to za trudne, pokazać, że
    \begin{align}
        I_1 & = \frac 3 4 \zeta (2) \zeta (3) - \frac {27}{16} \zeta(5), \\
        I_2 & = \frac 9 4 \zeta (3) \zeta (4) + \frac{45}{4} \zeta(2) \zeta(5) - \frac{363}{16} \zeta (7), \\
        I_3 & = \frac{2835}{8} \zeta(2) \zeta (7) + \frac {135}{8} \zeta (3) \zeta (6) + \frac {675}{8} \zeta (4) \zeta (5) - \frac {22635}{32} \zeta (9).
    \end{align} 
\end{problem_with_solution}

% the evaluation of a class of logarithmic integrals using a slightly modified result from ,,Table of Integrals, Series and Products'' by I. S. Gradshteyn and I. M. Ryzhik together with a series result elementarily proved by Guy Bastien

% SOLUTION
\begin{solution}[do problemu \ref{valean_1_10}]
    Patrz \cite[s. 6, 7]{valean19}.
\end{solution}
% SOLUTION

\begin{problem_with_solution}
    \label{valean_1_13}%
    \begin{equation}
        \int_0^1 \frac{x \log (1 \pm x)}{1 + x^2} \, \mathrm{d} x = \frac 1 8 \left(\log^2 (2) + \frac{\pm 3 - 2}{2} \zeta(2)\right).
    \end{equation} 
\end{problem_with_solution}

% A Special Pair of Logarithmic Integrals with Connections in the Area of the Alternating Harmonic Series

% SOLUTION
\begin{solution}[do problemu \ref{valean_1_13}]
    Patrz \cite[s. 8]{valean19}.
\end{solution}
% SOLUTION

% Another Special Pair of Logarithmic Integrals with Connections in the Area of the Alternating Harmonic Series

\begin{problem_with_solution}
    \label{valean_1_14}%
    \begin{align}
        16 \int_0^1 \frac{x}{1+ x^2} \log (1 - x) \log x \,\mathrm{d}x & = \frac{41}{4} \zeta(3) - 9 \log(2) \zeta(2) \\
        16 \int_0^1 \frac{x}{1+ x^2} \log (1 + x) \log x \,\mathrm{d}x & = -\frac{15}{4} \zeta(3) + 3 \log(2) \zeta(2)
    \end{align} 
\end{problem_with_solution}

% A Special Pair of Logarithmic Integrals with Connections in the Area of the Alternating Harmonic Series

% SOLUTION
\begin{solution}[do problemu \ref{valean_1_14}]
    Patrz \cite[s. 8]{valean19}.
\end{solution}
% SOLUTION

\begin{problem_with_solution}
    \label{valean_1_17}%
    \begin{align}
        \int_0^1 \int_0^1 \frac{\log y - \log x}{\log (- \log x) - \log(- \log y)} \,\mathrm{d}x \,\mathrm{d}y = \frac{7 \zeta(3)}{6 \zeta (2)}.
    \end{align} 
\end{problem_with_solution}

% Let’s Take Two Double Logarithmic Integrals with Beautiful Values Expressed in Terms of the Riemann Zeta Function

% SOLUTION
\begin{solution}[do problemu \ref{valean_1_17}]
    Patrz \cite[s. 10]{valean19}.
\end{solution}
% SOLUTION

\begin{problem_with_solution}
    \label{valean_1_18}%
    Niech $G$ oznacza stałą Catalana.
    \begin{align}
        8 \int_0^1 \log (1 - x) \arctan x \,\mathrm{d}x & = 4 \log (2) - \log^2 (2) + \frac 5 2 \zeta(2) - 2 \pi + \pi \log 2 - 8 G \\
        8 \int_0^1 \log (1 + x) \arctan x \,\mathrm{d}x & = 4 \log (2) - \log^2 (2) - \frac 1 2 \zeta(2) - 2 \pi + 3 \pi \log 2.
    \end{align} 
\end{problem_with_solution}

% Interesting Integrals Containing the Inverse Tangent Function and the Logarithmic Function

% SOLUTION
\begin{solution}[do problemu \ref{valean_1_18}]
    Patrz \cite[s. 10, 11]{valean19}.
\end{solution}
% SOLUTION

\begin{problem_with_solution}
    \label{valean_1_20}%
    Niech $G$ oznacza stałą Catalana.
    \begin{align}
        8 \int_0^1 \frac{\log (1 - x) \arctan x}{1+x^2} \,\mathrm{d}x & = \frac 3 4 \log (2) \zeta(2) - \frac 7 8 \zeta(3) - \pi G, \\
        8 \int_0^1 \frac{\log (1 + x) \arctan x}{1+x^2} \,\mathrm{d}x & = \frac 3 4 \log (2) \zeta(2) + \frac {21} 8 \zeta(3) - \pi G,
    \end{align} 
\end{problem_with_solution}

% More Interesting Integrals Involving the Inverse Tangent Function and the Logarithmic Function: The First Part

% SOLUTION
\begin{solution}[do problemu \ref{valean_1_20}]
    Patrz \cite[s. 12]{valean19}.
\end{solution}
% SOLUTION

\begin{problem_with_solution}
    \label{valean_1_21}%
    Niech $G$ oznacza stałą Catalana.
    \begin{align}
        \int_0^1 \frac{\arctan^2 x \log (1 + x)}{1 + x^2} \,\mathrm{d} x = \log 2 \frac {\pi^3}{384} + \frac {21}{256} \pi \zeta(3) - \frac{3}{16} \zeta (2) G.
    \end{align} 
\end{problem_with_solution}

% More Interesting Integrals Involving the Inverse Tangent Function and the Logarithmic Function: The Second Part

% SOLUTION
\begin{solution}[do problemu \ref{valean_1_21}]
    Patrz \cite[s. 12]{valean19}.
\end{solution}
% SOLUTION

\begin{problem_with_solution}
    \label{valean_1_22}%
    Niech $G$ oznacza stałą Catalana.
    \begin{align}
        I & = \int_0^1 \arctan x \log x \left(\log (1-x) - \frac {x}{1-x}\right) \,\mathrm{d} x \\
        & = G - \frac{41}{64} \zeta (3) + \frac{9 \log 2 - 5}{96} \pi^2 + \frac{2 - \log 2}{8} \pi - \frac {\log 2}{2} + \frac{\log^2 (2)}{8}.
    \end{align} 
\end{problem_with_solution}

% Challenging Integrals Involving arctan(x), log(x), log(1−x)

% SOLUTION
\begin{solution}[do problemu \ref{valean_1_22}]
    Patrz \cite[s. 13]{valean19}.
\end{solution}
% SOLUTION


\begin{problem_with_solution}
    \label{valean_1_23}%
    \begin{align}
        I & = \int_0^1 \arctan x \log x \log (1 + x) \,\mathrm{d}x \\
        & = \frac{\log 2}{2} G - \frac{\pi^3}{64} + \frac{15}{64} \zeta(3) - \frac{\pi^2}{96} (3 \log 2 + 1) + \frac{\pi} {8} (4-3 \log 2) + \frac{\log^2(2)}{8} - \log 2.
    \end{align} 
\end{problem_with_solution}

% Challenging Integrals Involving arctan(x), log(x), log(1−x)

% SOLUTION
\begin{solution}[do problemu \ref{valean_1_23}]
    Patrz \cite[s. 13, 14]{valean19}.
\end{solution}
% SOLUTION



\begin{problem_with_solution}
    \label{valean_1_24}%
    Niech $n \ge 1$ będzie naturalne.
    Znaleźć wartość
    \begin{align}
        I_{2n} & := \int_0^1 \frac {\arctan x \log^{2n} (x)}{1 + x } \,\mathrm{d}x
    \end{align} 
    lub, jeśli jest to za trudne, pokazać, że 
    \begin{align}
        I_{1} = \frac{\log 2}{2}G - \frac{\pi^3}{64}.
    \end{align} 
\end{problem_with_solution}

% Challenging Integrals Involving arctan(x), log(x), log(1−x)

% SOLUTION
\begin{solution}[do problemu \ref{valean_1_24}]
    Patrz \cite[s. 14, 15]{valean19}.
\end{solution}
% SOLUTION


\begin{problem_with_solution}
    \label{valean_1_26}%
    \begin{align}
        \int_0^1 \frac{\arctan x}{x} \log \frac{1+x^2}{(1-x)^2} \,\mathrm{d}x = \frac{\pi^3}{16}.
    \end{align} 
\end{problem_with_solution}

% Challenging Integrals Involving arctan(x), log(x), log(1−x)

% SOLUTION
\begin{solution}[do problemu \ref{valean_1_26}]
    Patrz \cite[s. 17]{valean19}.
\end{solution}
% SOLUTION

\begin{problem_with_solution}
    \label{valean_1_32}%
    Znaleźć rekurencję, jaką spełnia
    \begin{align}
        I_n = \int_0^1 \frac{x^n}{(1+x)(1+x^2)^n} \,\mathrm{d}x.
    \end{align} 
\end{problem_with_solution}

% Challenging Integrals Involving arctan(x), log(x), log(1−x)

% SOLUTION
\begin{solution}[do problemu \ref{valean_1_32}]
    Patrz \cite[s. 21, 22]{valean19}.
\end{solution}
% SOLUTION

\begin{problem_with_solution}
    \label{valean_1_37}%
    \begin{align}
        \int_0^\infty \int_0^\infty \frac {e^{-x}-e^{-y}}{x-y} \frac{1-e^{-x}}{x} \frac{1-e^{-y}}{y} \,\mathrm{d}x \,\mathrm{d}y.
    \end{align} 
\end{problem_with_solution}

% SOLUTION
\begin{solution}[do problemu \ref{valean_1_37}]
    Patrz \cite[s. ?????]{valean19}.
\end{solution}
% SOLUTION


\begin{problem_with_solution}
    \label{valean_1_38}%
    Znaleźć
    \begin{align}
        \lim_{n\to\infty} \left(\frac{1}{n!} \int_0^\infty \int_0^\infty \frac{x^n - y^n}{e^x - e^y} \,\mathrm{d}x \,\mathrm{d}y - 2n\right)
    \end{align} 
    po znalezieniu wewnętrznej całki (dla całkowitych liczb $n \ge 1$).
\end{problem_with_solution}

% SOLUTION
\begin{solution}[do problemu \ref{valean_1_38}]
    Patrz \cite[s. ?????]{valean19}.
\end{solution}
% SOLUTION

\begin{problem_with_solution}
    \label{valean_1_40}%
    \begin{align}
        I_4 & = \int_0^\infty \frac{\pi^2}{x^3} \tanh (\pi x)  + \frac{3}{x^5} \tanh(\pi x) - \frac{3\pi}{x^4} \,\mathrm{d}x = 93 \zeta(5) - 42 \zeta(2) \zeta(3), \\
        I_5 & = \int_0^\infty \frac{2\pi^2}{x^3} \tanh(\pi x) + \frac{12}{x^5} \tanh (\pi x) - \frac{6 \pi^2}{x^3} \csch (2 \pi x) - \frac{9 \pi}{x^4} \,\mathrm{d} x \\
        & = 372 \zeta(5) - 192 \zeta(2)\zeta(3).
    \end{align} 
\end{problem_with_solution}

% SOLUTION
\begin{solution}[do problemu \ref{valean_1_40}]
    Patrz \cite[s. ?????]{valean19}.
\end{solution}
% SOLUTION


\begin{problem_with_solution}
    \label{valean_1_41}%
    \begin{align}
        I_1 & = \int_0^\infty \frac{\sin (\sin x)}{x} e^{\cos x} \,\mathrm{d} x = \frac{\pi} 2 (e - 1), \\
        I_2 & = \int_0^\infty \frac{\sin x \cdot \sin (\sin x)}{x^2} e^{\cos x} \,\mathrm{d} x = \frac{\pi} 2 (e - 1).
    \end{align} 
\end{problem_with_solution}

% SOLUTION
\begin{solution}[do problemu \ref{valean_1_41}]
    Patrz \cite[s. ?????]{valean19}.
\end{solution}
% SOLUTION


%
	%

\subsection{Znalezione na math.stackexchange.com}
\subsection{Znalezione na math.stackexchange.com} % SOLUTION
Wszystkie poniższe całki pojawiają się na stronie na math.stackexchange.com.

% https://math.stackexchange.com/questions/541751/how-prove-this-i-int-0-infty-frac1x-ln-left-frac1x1-x-right2/541861#541861
\begin{problem}[pytanie 541751]
    \label{stack_541751}%
    \begin{equation}
        I = \int_0^\infty \frac{1}{x} \log \left(\frac{1+x}{1-x}\right)^2 \,\mathrm{d}x = \pi^2
    \end{equation}
\end{problem}

\textbf{Problem \ref{stack_562694}} -- podstawiamy $y = (1+x) / (1-y)$ i mamy % SOLUTION
\begin{align} % SOLUTION
    I & = 2 \int_{-1}^1 \frac{\log y^2}{1-y^2} \,\mathrm{d}y \\ % SOLUTION
        & = 8 \int_0^1 \frac{\log y}{1-y^2} \, \mathrm{d}{y} \\ % SOLUTION
        & = 8 \sum_{k=0}^\infty \int_0^1 y^{2k} \log y \,\mathrm{d} y \\ % SOLUTION
        & = 8 \sum_{k=0}^\infty \frac{1}{(2k+1)^2} \\ % SOLUTION
        & = 8 \cdot \frac{\pi^2}{8} = 8. % SOLUTION
\end{align} % SOLUTION

% https://math.stackexchange.com/q/562694
\begin{problem}[pytanie 562694]
    \label{stack_562694}%
    \begin{equation}
        \int_{-1}^1 \frac{1}{x} \sqrt{\frac{1+x}{1-x}} \log \frac{2x^2+2x+1}{2x^2-2x+1} \,\mathrm{d}x = 4 \pi \operatorname{arccot} \sqrt{\phi}.
    \end{equation}
\end{problem}

\textbf{Problem \ref{stack_562694}} -- patrz przypis\footnote{\url{https://math.stackexchange.com/questions/562694/}}. % SOLUTION

% TODO: https://math.stackexchange.com/a/942440

%

\chapter{Rozwiązania}
% %

\section{Całkowanie przez zgadnięcie pochodnej}

Czasami wystarczy zgadnąć wynik (albo znaleźć go w tablicy pochodnych) i sprawdzić, że pasuje przez zróżniczkowanie.

\begin{integral}
    Niech $n \neq -1$.
    Wtedy
    \begin{equation}
        \int x^n \,\mathrm{d}x = \frac{x^{n+1}}{n+1}.
    \end{equation}
\end{integral}

\begin{proof}
    Wprost z \ref{prp:derivative_power}.
\end{proof}

%
\section{Całkowanie przez podstawianie}

% Banaś, Wędrychowicz, 12.18.
\begin{integral}
    $\int (\arcsin x)^2 \,\mathrm{d}x$.
\end{integral}

\begin{solution}
    Podstawiamy $u = \arcsin x$ i dostajemy całkę z $u^2 \cos u$, którą rozwiązujemy przez części, tak jak w przykładzie \ref{banas_12_14}.
\end{solution}

% Banaś, Wędrychowicz, 12.19.
\begin{integral}
    $\int \sin(\log x) \, \mathrm{d}x$.
\end{integral}

\begin{solution}
    Podstawiamy $u = \log x$, $\mathrm{d} u = \mathrm{d} x / x$, $x = \exp u$ i dostajemy całkę z $e^u \sin u$, którą rozwiązujemy przez części, tak jak w przykładzie \ref{banas_12_19_auxilia}.
\end{solution}

% Banaś, Wędrychowicz, 12.20.
\begin{integral}
    $\int \cos(\log x) \, \mathrm{d}x$.
\end{integral}    

% Banaś, Wędrychowicz: 12.29
\begin{integral}
    Banaś-Wędrychowicz, 12.29.
\end{integral}

% Banaś, Wędrychowicz: 12.30
\begin{integral}
    Banaś-Wędrychowicz, 12.30.
\end{integral}

% Banaś, Wędrychowicz: 12.31
\begin{integral}
    Banaś-Wędrychowicz, 12.31.
\end{integral}

% Banaś, Wędrychowicz: 12.32
\begin{integral}
    Banaś-Wędrychowicz, 12.32.
\end{integral}

% Banaś, Wędrychowicz: 12.33
\begin{integral}
    Banaś-Wędrychowicz, 12.33.
\end{integral}

% Banaś, Wędrychowicz: 12.34
\begin{integral}
    Banaś-Wędrychowicz, 12.34.
\end{integral}

% Banaś, Wędrychowicz: 12.35
\begin{integral}
    Banaś-Wędrychowicz, 12.35.
\end{integral}

% Banaś, Wędrychowicz: 12.36
\begin{integral}
    Banaś-Wędrychowicz, 12.36.
\end{integral}

% Banaś, Wędrychowicz: 12.37
\begin{integral}
    Banaś-Wędrychowicz, 12.37.
\end{integral}

% Banaś, Wędrychowicz: 12.38
\begin{integral}
    Banaś-Wędrychowicz, 12.38.
\end{integral}

% Banaś, Wędrychowicz: 12.39
\begin{integral}
    Banaś-Wędrychowicz, 12.39.
\end{integral}

% Banaś, Wędrychowicz: 12.40
\begin{integral}
    Banaś-Wędrychowicz, 12.40.
\end{integral}

% Banaś, Wędrychowicz: 12.41
\begin{integral}
    Banaś-Wędrychowicz, 12.41.
\end{integral}

% Banaś, Wędrychowicz: 12.42
\begin{integral}
    Banaś-Wędrychowicz, 12.42.
\end{integral}

% Banaś, Wędrychowicz: 12.43
\begin{integral}
    Banaś-Wędrychowicz, 12.43.
\end{integral}

% Banaś, Wędrychowicz: 12.44
\begin{integral}
    Banaś-Wędrychowicz, 12.44.
\end{integral}

% Banaś, Wędrychowicz: 12.45
\begin{integral}
    Banaś-Wędrychowicz, 12.45.
\end{integral}

% Banaś, Wędrychowicz: 12.46
\begin{integral}
    Banaś-Wędrychowicz, 12.46.
\end{integral}

% Banaś, Wędrychowicz: 12.47
\begin{integral}
    Banaś-Wędrychowicz, 12.47.
\end{integral}

% Banaś, Wędrychowicz: 12.48
\begin{integral}
    Banaś-Wędrychowicz, 12.48.
\end{integral}

% Banaś, Wędrychowicz: 12.49
\begin{integral}
    Banaś-Wędrychowicz, 12.49.
\end{integral}

% Banaś, Wędrychowicz: 12.50
\begin{integral}
    Banaś-Wędrychowicz, 12.50.
\end{integral}

% Banaś, Wędrychowicz: 12.51
\begin{integral}
    Banaś-Wędrychowicz, 12.51.
\end{integral}

% Banaś, Wędrychowicz: 12.52
\begin{integral}
    Banaś-Wędrychowicz, 12.52.
\end{integral}

% Banaś, Wędrychowicz: 12.53
\begin{integral}
    Banaś-Wędrychowicz, 12.53.
\end{integral}

% Banaś, Wędrychowicz: 12.54
\begin{integral}
    Banaś-Wędrychowicz, 12.54.
\end{integral}

% Banaś, Wędrychowicz: 12.55
\begin{integral}
    Banaś-Wędrychowicz, 12.55.
\end{integral}

% Banaś, Wędrychowicz: 12.56
\begin{integral}
    Banaś-Wędrychowicz, 12.56.
\end{integral}

% Banaś, Wędrychowicz: 12.57
\begin{integral}
    Banaś-Wędrychowicz, 12.57.
\end{integral}

% Banaś, Wędrychowicz: 12.58
\begin{integral}
    Banaś-Wędrychowicz, 12.58.
\end{integral}

\subsection{Podstawienia Eulera}

TODO: Banaś Wędrychowicz, 12.71 - 12.87

\begin{integral}
    Banaś-Wędrychowicz, 12.58.
\end{integral}

% https://math.stackexchange.com/questions/541751/how-prove-this-i-int-0-infty-frac1x-ln-left-frac1x1-x-right2/541861#541861
\begin{integral}[pytanie 541751 na math.stackexchange.com]
    \begin{equation}
        I = \int_0^\infty \frac{1}{x} \log \left(\frac{1+x}{1-x}\right)^2 \,\mathrm{d}x = \pi^2
    \end{equation}
\end{integral}

\begin{proof}
    Podstawiamy $y = (1+x) / (1-y)$:
    \begin{align}
        I & = 2 \int_{-1}^1 \frac{\log y^2}{1-y^2} \,\mathrm{d}y \\
          & = 8 \int_0^1 \frac{\log y}{1-y^2} \, \mathrm{d}{y} \\
          & = 8 \sum_{k=0}^\infty \int_0^1 y^{2k} \log y \,\mathrm{d} y \\
          & = 8 \sum_{k=0}^\infty \frac{1}{(2k+1)^2} \\
          & = 8 \cdot \frac{\pi^2}{8} = 8.
    \end{align}
\end{proof}
%

\section{Całkowanie przez części}

\begin{proposition}[wzór na całkowanie przez części]
\label{prp_int_by_parts}%
    Jeśli funkcje $f, g \colon I \to \R$ są różniczkowalne, to
    \begin{equation}
        \int f(x) g'(x) \,\mathrm{d}x = f(x) g(x) - \int f'(x) g(x) \,\mathrm{d} x.
    \end{equation}
\end{proposition}

\begin{proof}
    Całkujemy obie strony wzoru na pochodną iloczynu $(fg)' = fg' + f'g$, a następnie porządkujemy strony równości.
\end{proof}

\begin{integral}
    $\int x \sin x \,\mathrm{d} x = \ldots$
\end{integral}

\begin{proof}
    Całkujemy przez części, $f(x) = x$, $g'(x) = \sin x$.
    \begin{align}
        \int x \sin x \,\mathrm{d} x & = -x \cos x - \int - \cos x \, \mathrm{d}x \\
                                     & = -x \cos x + \sin x.
    \end{align}
\end{proof}

Analogicznie obliczamy całki:

\begin{multicols}{2}
\begin{integral}
    $\int x \cos x \,\mathrm{d} x = \ldots$
\end{integral}

\begin{integral}
    $\int x \exp x \,\mathrm{d} x = \ldots$
\end{integral}
\end{multicols}

\begin{integral}
    $\int x \arctan x \,\mathrm{d} x$.
\end{integral}

\begin{proof}
    Całkujemy przez części, $f(x) = \arctan x$, $g'(x) = x$.
    \begin{align}
        \int x \arctan x \, \mathrm{d} x & = \frac 12 x^2 \arctan x - \int \frac{x^2 \,\mathrm{d}x}{2(x^2+1)} \\
                                         & = \frac 12 x^2 \arctan x - \frac 12 \left(\int 1 \,\mathrm{d}x - \int \frac{\mathrm{d}x}{x^2+1} \right) \\
                                         & = \frac 12 x^2 \arctan x - \frac 12 \left(x - \arctan x \right) \\
                                         & = \frac 12 \left((x^2+1)\arctan x - x \right).
    \end{align}
\end{proof}

\begin{integral}
    $\int \arccos x \,\mathrm{d} x$.
\end{integral}

\begin{proof}
    Całkujemy najpierw przez części, $f(x) = \arccos x$, $g'(x) = 1$, a potem podstawiamy $u = 1 - x^2$, $\mathrm{d} u = -2x \mathrm{d}x$:
    \begin{align}
        \int \arccos x \, \mathrm{d} x & = x \arccos x - \int  \frac{-x \,\mathrm{d}x}{\sqrt{1-x^2}} \\
        & = x \arccos x - \frac 12 \int \frac {\mathrm{d}u}{\sqrt{u}} \\
        & = x \arccos x - \sqrt{1 - x^2}.
    \end{align}
\end{proof}

\begin{integral}
    $\int x^n \log x \,\mathrm{d} x = \ldots$, gdzie $n \in \N$.
\end{integral}

\begin{proof}
    Całkujemy przez części, $f(x) = \log x$, $g'(x) = x^n$.
    \begin{align}
        \int x^n \log x \, \mathrm{d} x & = \frac{x^{n+1} \log x}{n+1} - \int \frac{x^n \,\mathrm{d} x}{n+1} \\
                                        & = \frac{x^{n+1} \log x}{n+1} - \frac{x^{n+1}}{(n+1)^2}.
    \end{align}
\end{proof}

%
% %

\section{Całkowanie funkcji wymiernych}
Całkowanie funkcji wymiernych

\begin{problem}
\begin{equation}
    \int_0^\infty \frac{x^n \,\mathrm{d}x}{(ax+b)^{m+1}}  = \frac{(-1)^{n+1} (-1-m)! \cdot n!}{a^{n+1} b^{m-n} (n-m)!}
\end{equation}
\end{problem}

\begin{solution}
    Rozdział 3 Borosa, strony 48-60.
\end{solution}

\begin{problem}
\begin{equation}
    \int_0^\infty \frac{\mathrm{d}x}{x^3 - 1} = - \frac{\pi}{3\sqrt{3}}
\end{equation}
\end{problem}

\begin{solution}
    Strona 22 w Inside Interesting Integrals.
\end{solution}


Całka Donalda Percy'ego Dalzella:
\begin{problem}
\begin{equation}
    \int_0^1 \frac{x^4(1-x)^4}{1 + x^4} \,\mathrm{d}x = 22/7 - \pi
\end{equation}
\end{problem}

\begin{solution}
    Strona 24 w Inside Interesting Integrals.
\end{solution}


\begin{problem}
\begin{equation}
    \int_0^\infty \frac{x^8-4x^6+9x^4-5x^2+1}{x^{12}-10x^{10}+37x^8-42x^6+26x^4-8x^2+1} \,\mathrm{d}x = \frac{\pi}{2}
\end{equation}
\end{problem}

\begin{solution}
    Strona 258 w: Bailey, D. H.; Borwein, J. M.; Calkin, N. J.; Girgensohn, R.; Luke, D. R.; and Moll, V. H. Experimental Mathematics in Action. Wellesley, MA: A K Peters, 2007.
\end{solution}

%
% %

\section{Całkowanie różniczek dwumiennych} % https://encyclopediaofmath.org/wiki/Differential_binomial
Całkowanie różniczek dwumiennych

%
% \section{Całkowanie funkcji trygonometrycznych}
% Całkowanie funkcji trygonometrycznych
%

\section{Sztuczka Feynmana: różniczkowanie pod znakiem całki}
% SOLUTION
\section{Sztuczka Feynmana: różniczkowanie pod znakiem całki}
% SOLUTION

% https://math.stackexchange.com/questions/942263/really-advanced-techniques-of-integration-definite-or-indefinite
\begin{problem}
    $\int_0^\infty \sin(x) / x \,\mathrm{d}x = \pi/2$.
\end{problem}

% TODO: przepisać całkę z s. 82, Nahin

\begin{problem_with_solution}
    \label{nahin_holzweg}%
    Niech $a, b > 0$, wtedy
    \begin{equation}
        \int_{-\infty}^\infty \frac{\cos ax}{b^2 - x^4} \,\mathrm{d} x = \frac{\pi}{b} \sin (ab)
    \end{equation}
\end{problem_with_solution}

% SOLUTION
\textbf{Problem \ref{nahin_holzweg}} -- \cite[s. 115, 375, 376]{nahin15}.
% SOLUTION

\begin{problem}
    \label{nahin_datenautobahn}%
    Niech $a > b$, wtedy
    \begin{equation}
        \int_{-\infty}^\infty \frac{\cos ax}{b^4 - x^4} \,\mathrm{d} x = \frac{\pi}{2b^3} [\sin (ab) + \exp (-ab)]
    \end{equation}
\end{problem}

% SOLUTION
\textbf{Problem \ref{nahin_datenautobahn}} -- \cite[s. 115, 376]{nahin15}.
% SOLUTION

% Nahin Inside interesting... page 83
\begin{problem}
    \begin{equation}
        \int_0^\infty \frac{\sin ax}{x e^{xy}} \,\mathrm{d}x = \pm \frac \pi 2 - \arctan \frac y a.
    \end{equation}
\end{problem}

\begin{problem}
    Niech $a > 0$.
    Wtedy
    \begin{equation}
        \int_0^\infty \frac{\sin ax}{x} \,\mathrm{d}x = \frac \pi 2.
    \end{equation}
\end{problem}




\begin{problem}[całka Frullaniego]
\index{całka Frullaniego}%
    Niech $f \colon [0, \infty) \to \R$ będzie funkcją ciągle różniczkowalną, której granica w nieskończoności istnieje.
    Wtedy dla ustalonych liczb rzeczywistych $a, b > 0$ mamy
    \begin{equation}
        \int_0^\infty \frac{f(ax) - f(bx)}{x} \,\mathrm{d} x = \left[\lim_{x \to \infty} f(x) - f(0) \right] \cdot \log \frac a b.
    \end{equation}
\end{problem}

Okazuje się, że problem rozwiązał Cauchy (około 1823 roku), ale też Giuliano Frullani, matematyk włoski (zapowiedź w 1821 roku, publikacja około 1829 roku).
\index[persons]{Frullani, Giuliano}%
Nahin \cite[s. 85]{nahin85} używa tej nazwy do konkretnego wcielenia całki Frullaniego, dla $f = \arctan$.
Boros, Moll \cite[s. 98]{boros94} piszą mgliście ,,under some mild conditions on the function $f$''...

% Nahin Inside interesting... page 8x
\begin{problem}
    Niech $a, b > 0$.
    Wtedy
    \begin{equation}
        \int_0^\infty \frac{e^{-ax} - e^{-bx}}{x} \,\mathrm{d}x = \log \frac b a.
    \end{equation}
\end{problem}

% Nahin Inside interesting... page 89
\begin{problem}
    \begin{equation}
        \int_0^\infty \frac{\cos (ax) - \cos (bx)}{x^2} \,\mathrm{d}x = \frac \pi 2 (b - a).
    \end{equation}
\end{problem}

% Nahin Inside interesting... page 89
\begin{problem}
    \begin{equation}
        \int_0^\infty \frac{\cos (ax) - \cos (bx)}{x} \,\mathrm{d}x = \log \frac b a.
    \end{equation}
\end{problem}

% Nahin Inside interesting... page 89
\begin{problem}
    \label{nahin_kriegsrecht}
    \begin{equation}
        \int_0^\infty \frac{\log (a^2 x^2 + 1)}{x^2 + b^2} \,\mathrm{d}x = \frac \pi b \log (1 + ab)
    \end{equation}
\end{problem}

% SOLUTION
\textbf{Problem \ref{nahin_kriegsrecht}} -- \cite[s. 67]{nahin15} w szczególnym przypadku $a = b = 1$; \cite[s. 114, 375]{nahin15} w ogólności.
% SOLUTION

% Nahin Inside interesting... page 91
\begin{problem}
    Niech $a \ge 0$, wtedy
    \begin{equation}
        \int_0^1 \frac{x^a - 1}{\log x} \,\mathrm{d}x = \log(1+a).
    \end{equation}
\end{problem}

% Nahin Inside interesting... page 92
\begin{problem}
    Niech $a \ge 0$, wtedy
    \begin{equation}
        \int_0^1 \frac{x^a - x^b}{\log x} \,\mathrm{d}x = \log \frac{1+a}{1+b}.
    \end{equation}
\end{problem}

% Nahin Inside interesting... page 96
\begin{problem}
    Niech $a > b$, wtedy
    \begin{equation}
        \int_0^\pi \frac{\mathrm{d}x} {a + b \cos x} = \frac{\pi}{\sqrt{a^2 - b^2}}.
    \end{equation}
\end{problem}

\begin{problem}
    \label{nahin_dini}%
    Niech $a \ge 0$ będzie dowolną liczbą rzeczywistą.
    Wtedy
    \begin{equation}
        \int_0^\pi \log (1 - 2 a \cos x + a^2) \,\mathrm{d} x = \begin{cases}
            0, & \text{gdy } a^2 \le 1, \\
            2 \pi \log a & \textrm{w przeciwnym razie}.
        \end{cases}
    \end{equation}
\end{problem}

Nahin pisze, że powyższą całkę wyznaczył jako pierwszy włoski matematyk Ulisse Dini w 1878 roku i że ma ważne zastosowania w fizyce i inżynierii.
\index[persons]{Dini, Ulisse}%

% SOLUTION
\textbf{Problem \ref{nahin_dini}} -- \cite[s. 109-112]{nahin15}
% SOLUTION

% TODO: https://math.stackexchange.com/questions/9402/calculating-the-integral-int-0-infty-frac-cos-x1x2-mathrmdx-with
\begin{equation}
    I = \int_0^\infty \frac {\cos x}{1+x^2} \,\mathrm{d}x
\end{equation}


%

\section{Zadania z turniejów całkowania}
	%

\subsection{MIT Integration BEE 2024} % SOLUTION

\begin{problem}[ćwierćfinał 2, problem 2]
    \begin{equation}
        I = \int_0^1 \frac 1 x \log (1 + x^2 + x^3 + x^4 + x^5 + x^6 + x^7 + x^9) \,\mathrm{d}x
    \end{equation}
\end{problem}

% https://math.stackexchange.com/questions/1617081/proving-an-integration-equality % SOLUTION
\begin{solution} % SOLUTION
    Łatwo widać, że szukana całka jest równa $I_2 + I_3 + I_4$, gdzie % SOLUTION
    \begin{align} % SOLUTION
        I_n & = \int_0^1 \frac {\log (1 + x^n)}{x} \,\mathrm{d}x \\ % SOLUTION
            & = \frac 1 n \int_0^1 \frac {\log (1 + x)}{x} \,\mathrm{d}x \\ % SOLUTION
            & = \frac 1 n \int_0^1 \frac 1 x \sum_{k=1}^\infty \frac{(-1)^{k-1}x^k}{k} \,\mathrm{d}x \\ % SOLUTION
            & = \frac 1 n \sum_{k=1}^\infty \frac{(-1)^{k-1}}{k^2} \\ % SOLUTION
            & = \frac {\pi^2}{12n}. % SOLUTION
    \end{align} % SOLUTION
\end{solution} % SOLUTION


%

%

\section{Teoretyczna teoria}
\section{Teoretyczna teoria} % SOLUTION

\begin{problem}[problem B4 na egzaminie Putnam 1968]
    \label{putnam_1968_b4}%
    Niech $f \colon \R \to \R$ będzie ciągłą funkcją taką, że całka $\int_\R f(x)\,\mathrm{d}x$ istnieje.
    Pokazać, że całka
    \begin{equation}
        \int_\R f\left(x - \frac 1 x\right)\,\mathrm{d}x
    \end{equation}
    też istnieje i przyjmuje tę samą wartość.
\end{problem}

% SOLUTION
\begin{solution}[do problemu \ref{putnam_1968_b4}]
    Będziemy całkować przez podstawienie, $x = \exp \theta$ (i potem $x = - \exp -\theta$):
    \begin{align}
        \int_{-\infty}^{\infty}f\left(x-x^{-1}\right)dx&=\int_{0}^{\infty}f\left(x-x^{-1}\right)dx+\int_{-\infty}^{0}f\left(x-x^{-1}\right)dx=\\
        &=\int_{-\infty}^{\infty}f(2\sinh\theta)\,e^{\theta}d\theta+\int_{-\infty}^{\infty}f(2\sinh\theta)\,e^{-\theta}d\theta=\\
        &=\int_{-\infty}^{\infty}f(2\sinh\theta)\,2\cosh\theta\,d\theta=\\
        &=\int_{-\infty}^{\infty}f(x)\,dx.
    \end{align}
\end{solution}
% SOLUTION

%

\section{Trudne całki}
	%

\begin{problem_with_solution}
    \label{reuleaux_tetrahedron}%
    Czworościan Reuleaux to bryła będąca częścią wspólną czterech kul, których środki leżą w wierzchołkach czworościanu foremnego, a promienie są tej samej długości, co krawędzie tego czworościanu.
    Znaleźć objętość tej bryły,
    \begin{equation}
        V = \int_0^1
        \frac{
            8\sqrt{3}
        }{
            1 + 3t^2
        } - \frac{
            16 \sqrt{2} (3t+1) (4t^2 +t+1)^{3/2}
        }{
            (3t^2+1)(11t^2 + 2t + 3)^2
        } - \frac{
            \sqrt{2} (249 t^2 + 54t + 65)
        }{
            (11t^2 + 2t +3)^2
        } \,\mathrm{d} t.
    \end{equation}
\end{problem_with_solution}

% SOLUTION
\textbf{Problem \ref{reuleaux_tetrahedron}} -- patrz \url{https://mathworld.wolfram.com/ReuleauxTetrahedron.html}.
% SOLUTION

% TODO: https://mathworld.wolfram.com/images/gifs/FoxTrotMathTest.jpg
% TODO https://mathworld.wolfram.com/DefiniteIntegral.html

%
	\subsection{Prawie niemożliwe całki}
% SOLUTION
\subsection{Prawie niemożliwe całki}
% SOLUTION
Wszystkie poniższe całki pojawiają się w książce Valeana \cite{valean19}.

\begin{problem_with_solution}
    \label{valean_grundpreis}%
    Niech $y \in (-1, 1)$.
    Wtedy
    \begin{equation}
        \int_0^1 \frac{\mathrm{d}x}{(1+yx) \sqrt{1-x^2}} = \frac{\arccos y}{\sqrt{1-y^2}}.
    \end{equation}
\end{problem_with_solution}

% SOLUTION
\textbf{Problem \ref{valean_grundpreis}} -- 
patrz \cite[s. 1]{valean19}.
% SOLUTION

\begin{problem_with_solution}
    \label{valean_zeugenstand}%
    Niech $m, n$ będą liczbami naturalnymi.
    Wtedy
    \begin{equation}
        \int_0^1 x^m \log^n x \,\mathrm{d} x = \frac{(-1)^n \cdot n!}{(m+1)^{n+1}}.
    \end{equation}
\end{problem_with_solution}

% SOLUTION
\begin{solution}[do problemu \ref{valean_zeugenstand}]
    Patrz \cite[s. 1]{valean19}.
\end{solution}
% SOLUTION


Niech $H_{n}^{(m)} = 1 + 1/2^m + \ldots + 1/n^m$ oznacza $n$-tą uogólnioną liczbę harmoniczną.

\begin{problem_with_solution}
    \label{valean_1_3}%
    Rozpatrujemy rodzinę całek
    \begin{equation}
        I_{k,n} := \int_0^1 x^{n-1} \log^k (1-x) \,\mathrm{d} x.
    \end{equation}
    Mamy:
    \begin{align}
        I_{1,n} & = - \frac{H_n}{n} \\
        I_{2,n} & = \frac{H_n^2 + H_n^{(2)}}{n} \\
        I_{3,n} & = - \frac{H_n^3 + 3H_nH_n^{(2)} + 2H_n^{(3)}}{n} \\
        I_{4,n} & = \frac{H_n^4 + 6H_n^2 H_n^{(2)} + 8H_nH_n^{(3)} + 3(H_n^{(2)})^2 + 6H_n^{(4)}}{n}.
    \end{align}
\end{problem_with_solution}

% (Valean nazywa to ,,four logarithmic integrals strongly connected with the league of harmonic series'').

% SOLUTION
\begin{solution}[do problemu \ref{valean_1_3}]
    Patrz \cite[s. 2]{valean19}.
\end{solution}
% SOLUTION

\begin{problem_with_solution}
    \label{valean_1_5}%
    Niech $s > 0$ będzie liczbą rzeczywistą, zaś $\psi$ oznacza funkcję digamma.
    Wtedy
    \begin{align}
        \int_0^1 \frac{x^{s-1}}{x+1} \,\mathrm{d} x & = \psi(s) - \psi\left(\frac s2\right) - \log 2 \\
        \int_0^\infty e^{-sx} \tanh x \,\mathrm{d} x & = \frac 1 2 \left[\psi\left(\frac{s+2}{4}\right) - \psi \left(\frac s4 \right) - \frac 2 s\right]. 
    \end{align}
\end{problem_with_solution}

% (Valean nazywa to ,,a couple of practical definite integrals expressed in terms of the digamma function'').

% SOLUTION
\begin{solution}[do problemu \ref{valean_1_5}]
    Patrz \cite[s. 3]{valean19}.
\end{solution}
% SOLUTION

\begin{problem_with_solution}
    \label{valean_1_7}%
    \begin{align}
        \int_0^1 \frac{1}{x} \log^2 (1+x) \,\mathrm{d}x & = \frac{1}{4} \zeta(3) \\
        \int_0^1 \frac{1}{x} \log (1+x) \log (1-x) \,\mathrm{d}x & = -\frac{5}{8} \zeta(3)
    \end{align}
\end{problem_with_solution}

% two little tricky classical logarithmic integrals

% SOLUTION
\begin{solution}[do problemu \ref{valean_1_7}]
    Patrz \cite[s. 4]{valean19}.
\end{solution}
% SOLUTION

\begin{problem_with_solution}[]
    \label{valean_1_8}%
    \begin{align}
        \int_0^1 [\log(1+x) \log(1-x)]^2 \,\mathrm{d} x & =
        24 - 8 \zeta(2)- 8 \zeta(3) - \zeta(4) \\
        & + 8 \log(2)\zeta(2) + 8 \log(2)\zeta(3) \\
        & - 4 \log^2(2)\zeta(2) \\
        & - 24 \log(2) + 12 \log^2(2)- 4 \log^3(2) + \log^4(2); 
    \end{align}
\end{problem_with_solution}

% a special trio of integrals

% SOLUTION
\begin{solution}[do problemu \ref{valean_1_8}]
    Patrz \cite[s. 4, 5]{valean19}.
\end{solution}
% SOLUTION

% TODO: https://math.stackexchange.com/questions/3413586/conjectural-closed-form-of-int-01-frac-logn-1-x-logn-1-1x1x-d

\begin{problem_with_solution}
    \label{valean_1_10}%
    Niech $n \ge 1$ będzie liczbą naturalną.
    Znaleźć
    \begin{equation}
        I_n = \int_0^1 \frac 1 x \log(1-x) \log^{2n} x \log (1+x) \,\mathrm{d}x.
    \end{equation}
    Jeśli jest to za trudne, pokazać, że
    \begin{align}
        I_1 & = \frac 3 4 \zeta (2) \zeta (3) - \frac {27}{16} \zeta(5), \\
        I_2 & = \frac 9 4 \zeta (3) \zeta (4) + \frac{45}{4} \zeta(2) \zeta(5) - \frac{363}{16} \zeta (7), \\
        I_3 & = \frac{2835}{8} \zeta(2) \zeta (7) + \frac {135}{8} \zeta (3) \zeta (6) + \frac {675}{8} \zeta (4) \zeta (5) - \frac {22635}{32} \zeta (9).
    \end{align} 
\end{problem_with_solution}

% the evaluation of a class of logarithmic integrals using a slightly modified result from ,,Table of Integrals, Series and Products'' by I. S. Gradshteyn and I. M. Ryzhik together with a series result elementarily proved by Guy Bastien

% SOLUTION
\begin{solution}[do problemu \ref{valean_1_10}]
    Patrz \cite[s. 6, 7]{valean19}.
\end{solution}
% SOLUTION

\begin{problem_with_solution}
    \label{valean_1_13}%
    \begin{equation}
        \int_0^1 \frac{x \log (1 \pm x)}{1 + x^2} \, \mathrm{d} x = \frac 1 8 \left(\log^2 (2) + \frac{\pm 3 - 2}{2} \zeta(2)\right).
    \end{equation} 
\end{problem_with_solution}

% A Special Pair of Logarithmic Integrals with Connections in the Area of the Alternating Harmonic Series

% SOLUTION
\begin{solution}[do problemu \ref{valean_1_13}]
    Patrz \cite[s. 8]{valean19}.
\end{solution}
% SOLUTION

% Another Special Pair of Logarithmic Integrals with Connections in the Area of the Alternating Harmonic Series

\begin{problem_with_solution}
    \label{valean_1_14}%
    \begin{align}
        16 \int_0^1 \frac{x}{1+ x^2} \log (1 - x) \log x \,\mathrm{d}x & = \frac{41}{4} \zeta(3) - 9 \log(2) \zeta(2) \\
        16 \int_0^1 \frac{x}{1+ x^2} \log (1 + x) \log x \,\mathrm{d}x & = -\frac{15}{4} \zeta(3) + 3 \log(2) \zeta(2)
    \end{align} 
\end{problem_with_solution}

% A Special Pair of Logarithmic Integrals with Connections in the Area of the Alternating Harmonic Series

% SOLUTION
\begin{solution}[do problemu \ref{valean_1_14}]
    Patrz \cite[s. 8]{valean19}.
\end{solution}
% SOLUTION

\begin{problem_with_solution}
    \label{valean_1_17}%
    \begin{align}
        \int_0^1 \int_0^1 \frac{\log y - \log x}{\log (- \log x) - \log(- \log y)} \,\mathrm{d}x \,\mathrm{d}y = \frac{7 \zeta(3)}{6 \zeta (2)}.
    \end{align} 
\end{problem_with_solution}

% Let’s Take Two Double Logarithmic Integrals with Beautiful Values Expressed in Terms of the Riemann Zeta Function

% SOLUTION
\begin{solution}[do problemu \ref{valean_1_17}]
    Patrz \cite[s. 10]{valean19}.
\end{solution}
% SOLUTION

\begin{problem_with_solution}
    \label{valean_1_18}%
    Niech $G$ oznacza stałą Catalana.
    \begin{align}
        8 \int_0^1 \log (1 - x) \arctan x \,\mathrm{d}x & = 4 \log (2) - \log^2 (2) + \frac 5 2 \zeta(2) - 2 \pi + \pi \log 2 - 8 G \\
        8 \int_0^1 \log (1 + x) \arctan x \,\mathrm{d}x & = 4 \log (2) - \log^2 (2) - \frac 1 2 \zeta(2) - 2 \pi + 3 \pi \log 2.
    \end{align} 
\end{problem_with_solution}

% Interesting Integrals Containing the Inverse Tangent Function and the Logarithmic Function

% SOLUTION
\begin{solution}[do problemu \ref{valean_1_18}]
    Patrz \cite[s. 10, 11]{valean19}.
\end{solution}
% SOLUTION

\begin{problem_with_solution}
    \label{valean_1_20}%
    Niech $G$ oznacza stałą Catalana.
    \begin{align}
        8 \int_0^1 \frac{\log (1 - x) \arctan x}{1+x^2} \,\mathrm{d}x & = \frac 3 4 \log (2) \zeta(2) - \frac 7 8 \zeta(3) - \pi G, \\
        8 \int_0^1 \frac{\log (1 + x) \arctan x}{1+x^2} \,\mathrm{d}x & = \frac 3 4 \log (2) \zeta(2) + \frac {21} 8 \zeta(3) - \pi G,
    \end{align} 
\end{problem_with_solution}

% More Interesting Integrals Involving the Inverse Tangent Function and the Logarithmic Function: The First Part

% SOLUTION
\begin{solution}[do problemu \ref{valean_1_20}]
    Patrz \cite[s. 12]{valean19}.
\end{solution}
% SOLUTION

\begin{problem_with_solution}
    \label{valean_1_21}%
    Niech $G$ oznacza stałą Catalana.
    \begin{align}
        \int_0^1 \frac{\arctan^2 x \log (1 + x)}{1 + x^2} \,\mathrm{d} x = \log 2 \frac {\pi^3}{384} + \frac {21}{256} \pi \zeta(3) - \frac{3}{16} \zeta (2) G.
    \end{align} 
\end{problem_with_solution}

% More Interesting Integrals Involving the Inverse Tangent Function and the Logarithmic Function: The Second Part

% SOLUTION
\begin{solution}[do problemu \ref{valean_1_21}]
    Patrz \cite[s. 12]{valean19}.
\end{solution}
% SOLUTION

\begin{problem_with_solution}
    \label{valean_1_22}%
    Niech $G$ oznacza stałą Catalana.
    \begin{align}
        I & = \int_0^1 \arctan x \log x \left(\log (1-x) - \frac {x}{1-x}\right) \,\mathrm{d} x \\
        & = G - \frac{41}{64} \zeta (3) + \frac{9 \log 2 - 5}{96} \pi^2 + \frac{2 - \log 2}{8} \pi - \frac {\log 2}{2} + \frac{\log^2 (2)}{8}.
    \end{align} 
\end{problem_with_solution}

% Challenging Integrals Involving arctan(x), log(x), log(1−x)

% SOLUTION
\begin{solution}[do problemu \ref{valean_1_22}]
    Patrz \cite[s. 13]{valean19}.
\end{solution}
% SOLUTION


\begin{problem_with_solution}
    \label{valean_1_23}%
    \begin{align}
        I & = \int_0^1 \arctan x \log x \log (1 + x) \,\mathrm{d}x \\
        & = \frac{\log 2}{2} G - \frac{\pi^3}{64} + \frac{15}{64} \zeta(3) - \frac{\pi^2}{96} (3 \log 2 + 1) + \frac{\pi} {8} (4-3 \log 2) + \frac{\log^2(2)}{8} - \log 2.
    \end{align} 
\end{problem_with_solution}

% Challenging Integrals Involving arctan(x), log(x), log(1−x)

% SOLUTION
\begin{solution}[do problemu \ref{valean_1_23}]
    Patrz \cite[s. 13, 14]{valean19}.
\end{solution}
% SOLUTION



\begin{problem_with_solution}
    \label{valean_1_24}%
    Niech $n \ge 1$ będzie naturalne.
    Znaleźć wartość
    \begin{align}
        I_{2n} & := \int_0^1 \frac {\arctan x \log^{2n} (x)}{1 + x } \,\mathrm{d}x
    \end{align} 
    lub, jeśli jest to za trudne, pokazać, że 
    \begin{align}
        I_{1} = \frac{\log 2}{2}G - \frac{\pi^3}{64}.
    \end{align} 
\end{problem_with_solution}

% Challenging Integrals Involving arctan(x), log(x), log(1−x)

% SOLUTION
\begin{solution}[do problemu \ref{valean_1_24}]
    Patrz \cite[s. 14, 15]{valean19}.
\end{solution}
% SOLUTION


\begin{problem_with_solution}
    \label{valean_1_26}%
    \begin{align}
        \int_0^1 \frac{\arctan x}{x} \log \frac{1+x^2}{(1-x)^2} \,\mathrm{d}x = \frac{\pi^3}{16}.
    \end{align} 
\end{problem_with_solution}

% Challenging Integrals Involving arctan(x), log(x), log(1−x)

% SOLUTION
\begin{solution}[do problemu \ref{valean_1_26}]
    Patrz \cite[s. 17]{valean19}.
\end{solution}
% SOLUTION

\begin{problem_with_solution}
    \label{valean_1_32}%
    Znaleźć rekurencję, jaką spełnia
    \begin{align}
        I_n = \int_0^1 \frac{x^n}{(1+x)(1+x^2)^n} \,\mathrm{d}x.
    \end{align} 
\end{problem_with_solution}

% Challenging Integrals Involving arctan(x), log(x), log(1−x)

% SOLUTION
\begin{solution}[do problemu \ref{valean_1_32}]
    Patrz \cite[s. 21, 22]{valean19}.
\end{solution}
% SOLUTION

\begin{problem_with_solution}
    \label{valean_1_37}%
    \begin{align}
        \int_0^\infty \int_0^\infty \frac {e^{-x}-e^{-y}}{x-y} \frac{1-e^{-x}}{x} \frac{1-e^{-y}}{y} \,\mathrm{d}x \,\mathrm{d}y.
    \end{align} 
\end{problem_with_solution}

% SOLUTION
\begin{solution}[do problemu \ref{valean_1_37}]
    Patrz \cite[s. ?????]{valean19}.
\end{solution}
% SOLUTION


\begin{problem_with_solution}
    \label{valean_1_38}%
    Znaleźć
    \begin{align}
        \lim_{n\to\infty} \left(\frac{1}{n!} \int_0^\infty \int_0^\infty \frac{x^n - y^n}{e^x - e^y} \,\mathrm{d}x \,\mathrm{d}y - 2n\right)
    \end{align} 
    po znalezieniu wewnętrznej całki (dla całkowitych liczb $n \ge 1$).
\end{problem_with_solution}

% SOLUTION
\begin{solution}[do problemu \ref{valean_1_38}]
    Patrz \cite[s. ?????]{valean19}.
\end{solution}
% SOLUTION

\begin{problem_with_solution}
    \label{valean_1_40}%
    \begin{align}
        I_4 & = \int_0^\infty \frac{\pi^2}{x^3} \tanh (\pi x)  + \frac{3}{x^5} \tanh(\pi x) - \frac{3\pi}{x^4} \,\mathrm{d}x = 93 \zeta(5) - 42 \zeta(2) \zeta(3), \\
        I_5 & = \int_0^\infty \frac{2\pi^2}{x^3} \tanh(\pi x) + \frac{12}{x^5} \tanh (\pi x) - \frac{6 \pi^2}{x^3} \csch (2 \pi x) - \frac{9 \pi}{x^4} \,\mathrm{d} x \\
        & = 372 \zeta(5) - 192 \zeta(2)\zeta(3).
    \end{align} 
\end{problem_with_solution}

% SOLUTION
\begin{solution}[do problemu \ref{valean_1_40}]
    Patrz \cite[s. ?????]{valean19}.
\end{solution}
% SOLUTION


\begin{problem_with_solution}
    \label{valean_1_41}%
    \begin{align}
        I_1 & = \int_0^\infty \frac{\sin (\sin x)}{x} e^{\cos x} \,\mathrm{d} x = \frac{\pi} 2 (e - 1), \\
        I_2 & = \int_0^\infty \frac{\sin x \cdot \sin (\sin x)}{x^2} e^{\cos x} \,\mathrm{d} x = \frac{\pi} 2 (e - 1).
    \end{align} 
\end{problem_with_solution}

% SOLUTION
\begin{solution}[do problemu \ref{valean_1_41}]
    Patrz \cite[s. ?????]{valean19}.
\end{solution}
% SOLUTION


%
	%

\subsection{Znalezione na math.stackexchange.com}
\subsection{Znalezione na math.stackexchange.com} % SOLUTION
Wszystkie poniższe całki pojawiają się na stronie na math.stackexchange.com.

% https://math.stackexchange.com/questions/541751/how-prove-this-i-int-0-infty-frac1x-ln-left-frac1x1-x-right2/541861#541861
\begin{problem}[pytanie 541751]
    \label{stack_541751}%
    \begin{equation}
        I = \int_0^\infty \frac{1}{x} \log \left(\frac{1+x}{1-x}\right)^2 \,\mathrm{d}x = \pi^2
    \end{equation}
\end{problem}

\textbf{Problem \ref{stack_562694}} -- podstawiamy $y = (1+x) / (1-y)$ i mamy % SOLUTION
\begin{align} % SOLUTION
    I & = 2 \int_{-1}^1 \frac{\log y^2}{1-y^2} \,\mathrm{d}y \\ % SOLUTION
        & = 8 \int_0^1 \frac{\log y}{1-y^2} \, \mathrm{d}{y} \\ % SOLUTION
        & = 8 \sum_{k=0}^\infty \int_0^1 y^{2k} \log y \,\mathrm{d} y \\ % SOLUTION
        & = 8 \sum_{k=0}^\infty \frac{1}{(2k+1)^2} \\ % SOLUTION
        & = 8 \cdot \frac{\pi^2}{8} = 8. % SOLUTION
\end{align} % SOLUTION

% https://math.stackexchange.com/q/562694
\begin{problem}[pytanie 562694]
    \label{stack_562694}%
    \begin{equation}
        \int_{-1}^1 \frac{1}{x} \sqrt{\frac{1+x}{1-x}} \log \frac{2x^2+2x+1}{2x^2-2x+1} \,\mathrm{d}x = 4 \pi \operatorname{arccot} \sqrt{\phi}.
    \end{equation}
\end{problem}

\textbf{Problem \ref{stack_562694}} -- patrz przypis\footnote{\url{https://math.stackexchange.com/questions/562694/}}. % SOLUTION

% TODO: https://math.stackexchange.com/a/942440

%

\chapter{Tłumaczenie numeracji}
Po lewej stronie numer zadania w \cite{banas_wedrychowicz}, po prawej numer tego samego zadaina u nas.

\begin{multicols}{3}
\begin{itemize}
    \item 12.1 -- \ref{banas_12_1}
    \item 12.2 -- \ref{banas_12_2}
    \item 12.3 -- \ref{banas_12_3}
    \item 12.4 -- patrz 12.3
    \item 12.5 -- patrz 12.3
    \item 12.6 -- \ref{banas_12_6}
    \item 12.7 -- \ref{banas_12_7}
    \item 12.8 -- \ref{banas_12_8}
    \item 12.9 -- \ref{banas_12_9}
    \item 12.10 -- \ref{banas_12_10}
    \item 12.11 -- \ref{banas_12_11}
    \item 12.12 -- \ref{banas_12_12}
    \item 12.13 -- \ref{banas_12_13}
    \item 12.14 -- \ref{banas_12_14}
    \item 12.15 -- patrz 12.14
    \item 12.16 -- patrz 12.12
    \item 12.17 -- patrz 12.12
    \item 12.18 -- \ref{banas_12_18}
    \item 12.19 -- \ref{banas_12_19}
    \item 12.20 -- \ref{banas_12_20}
    \item 12.21 -- \ref{banas_12_21}
    \item 12.22 -- pominięta
    \item 12.23 -- pominięta
    \item 12.24 -- pominięta
    \item 12.25 -- patrz 12.14
    \item 12.26 -- \ref{banas_12_26}
    % \item 12.27 -- \ref{banas_12_27}
    % \item 12.28 -- \ref{banas_12_28}
    % \item 12.29 -- \ref{banas_12_29}
    % \item 12.30 -- \ref{banas_12_30}
    % \item 12.31 -- \ref{banas_12_31}
    % \item 12.32 -- \ref{banas_12_32}
    % \item 12.33 -- \ref{banas_12_33}
    % \item 12.34 -- \ref{banas_12_34}
    % \item 12.35 -- \ref{banas_12_35}
    % \item 12.36 -- \ref{banas_12_36}
    % \item 12.37 -- \ref{banas_12_37}
    % \item 12.38 -- \ref{banas_12_38}
    % \item 12.39 -- \ref{banas_12_39}
    % \item 12.40 -- \ref{banas_12_40}
    % \item 12.41 -- \ref{banas_12_41}
    % \item 12.42 -- \ref{banas_12_42}
    % \item 12.43 -- \ref{banas_12_43}
    % \item 12.44 -- \ref{banas_12_44}
    % \item 12.45 -- \ref{banas_12_45}
    % \item 12.46 -- \ref{banas_12_46}
    % \item 12.47 -- \ref{banas_12_47}
    % \item 12.48 -- \ref{banas_12_48}
    % \item 12.49 -- \ref{banas_12_49}
    % \item 12.50 -- \ref{banas_12_50}
    % \item 12.51 -- \ref{banas_12_51}
    % \item 12.52 -- \ref{banas_12_52}
    % \item 12.53 -- \ref{banas_12_53}
    % \item 12.54 -- \ref{banas_12_54}
    % \item 12.55 -- \ref{banas_12_55}
    % \item 12.56 -- \ref{banas_12_56}
    % \item 12.57 -- \ref{banas_12_57}
    % \item 12.58 -- \ref{banas_12_58}
    % \item 12.59 -- \ref{banas_12_59}
    % \item 12.60 -- \ref{banas_12_60}
    % \item 12.61 -- \ref{banas_12_61}
    % \item 12.62 -- \ref{banas_12_62}
    % \item 12.63 -- \ref{banas_12_63}
    % \item 12.64 -- \ref{banas_12_64}
    % \item 12.65 -- \ref{banas_12_65}
    % \item 12.66 -- \ref{banas_12_66}
    % \item 12.67 -- \ref{banas_12_67}
    % \item 12.68 -- \ref{banas_12_68}
    % \item 12.69 -- \ref{banas_12_69}
    % \item 12.70 -- \ref{banas_12_70}
    % \item 12.71 -- \ref{banas_12_71}
    % \item 12.72 -- \ref{banas_12_72}
    % \item 12.73 -- \ref{banas_12_73}
    % \item 12.74 -- \ref{banas_12_74}
    % \item 12.75 -- \ref{banas_12_75}
    % \item 12.76 -- \ref{banas_12_76}
    % \item 12.77 -- \ref{banas_12_77}
    % \item 12.78 -- \ref{banas_12_78}
    % \item 12.79 -- \ref{banas_12_79}
    % \item 12.80 -- \ref{banas_12_80}
    % \item 12.81 -- \ref{banas_12_81}
    % \item 12.82 -- \ref{banas_12_82}
    % \item 12.83 -- \ref{banas_12_83}
    % \item 12.84 -- \ref{banas_12_84}
    % \item 12.85 -- \ref{banas_12_85}
    % \item 12.86 -- \ref{banas_12_86}
    % \item 12.87 -- \ref{banas_12_87}
    % \item 12.88 -- \ref{banas_12_88}
    % \item 12.89 -- \ref{banas_12_89}
    % \item 12.90 -- \ref{banas_12_90}
    % \item 12.91 -- \ref{banas_12_91}
    % \item 12.92 -- \ref{banas_12_92}
    % \item 12.93 -- \ref{banas_12_93}
    % \item 12.94 -- \ref{banas_12_94}
    % \item 12.95 -- \ref{banas_12_95}
    % \item 12.96 -- \ref{banas_12_96}
    % \item 12.97 -- \ref{banas_12_97}
    % \item 12.98 -- \ref{banas_12_98}
    % \item 12.99 -- \ref{banas_12_99}
    % \item 12.100 -- \ref{banas_12_100}
    % \item 12.101 -- \ref{banas_12_101}
    % \item 12.102 -- \ref{banas_12_102}
    % \item 12.103 -- \ref{banas_12_103}
    % \item 12.104 -- \ref{banas_12_104}
    % \item 12.105 -- \ref{banas_12_105}
    % \item 12.106 -- \ref{banas_12_106}
    % \item 12.107 -- \ref{banas_12_107}
    % \item 12.108 -- \ref{banas_12_108}
    % \item 12.109 -- \ref{banas_12_109}
    % \item 12.110 -- \ref{banas_12_110}
    % \item 12.111 -- \ref{banas_12_111}
    % \item 12.112 -- \ref{banas_12_112}
    % \item 12.113 -- \ref{banas_12_113}
    % \item 12.114 -- \ref{banas_12_114}
    % \item 12.115 -- \ref{banas_12_115}
    % \item 12.116 -- \ref{banas_12_116}
    % \item 12.117 -- \ref{banas_12_117}
    % \item 12.118 -- \ref{banas_12_118}
    % \item 12.119 -- \ref{banas_12_119}
    % \item 12.120 -- \ref{banas_12_120}
    % \item 12.121 -- \ref{banas_12_121}
    % \item 12.122 -- \ref{banas_12_122}
    % \item 12.123 -- \ref{banas_12_123}
    % \item 12.124 -- \ref{banas_12_124}
    % \item 12.125 -- \ref{banas_12_125}
    % \item 12.126 -- \ref{banas_12_126}
    % \item 12.127 -- \ref{banas_12_127}
    % \item 12.128 -- \ref{banas_12_128}
    % \item 12.129 -- \ref{banas_12_129}
    % \item 12.130 -- \ref{banas_12_130}
    % \item 12.131 -- \ref{banas_12_131}
    % \item 12.132 -- \ref{banas_12_132}
    % \item 12.133 -- \ref{banas_12_133}
    % \item 12.134 -- \ref{banas_12_134}
    % \item 12.135 -- \ref{banas_12_135}
    % \item 12.136 -- \ref{banas_12_136}
    % \item 12.137 -- \ref{banas_12_137}
    % \item 12.138 -- \ref{banas_12_138}
    % \item 12.139 -- \ref{banas_12_139}
    % \item 12.140 -- \ref{banas_12_140}
    % \item 12.141 -- \ref{banas_12_141}
    % \item 12.142 -- \ref{banas_12_142}
    % \item 12.143 -- \ref{banas_12_143}
    % \item 12.144 -- \ref{banas_12_144}
    % \item 12.145 -- \ref{banas_12_145}
    % \item 12.146 -- \ref{banas_12_146}
    % \item 12.147 -- \ref{banas_12_147}
    % \item 12.148 -- \ref{banas_12_148}
    % \item 12.149 -- \ref{banas_12_149}
    % \item 12.150 -- \ref{banas_12_150}
    % \item 12.151 -- \ref{banas_12_151}
    % \item 12.152 -- \ref{banas_12_152}
    % \item 12.153 -- \ref{banas_12_153}
    % \item 12.154 -- \ref{banas_12_154}
    % \item 12.155 -- \ref{banas_12_155}
    % \item 12.156 -- \ref{banas_12_156}
    % \item 12.157 -- \ref{banas_12_157}
    % \item 12.158 -- \ref{banas_12_158}
    % \item 12.159 -- \ref{banas_12_159}
    % \item 12.160 -- \ref{banas_12_160}
    % \item 12.161 -- \ref{banas_12_161}
    % \item 12.162 -- \ref{banas_12_162}
    % \item 12.163 -- \ref{banas_12_163}
    % \item 12.164 -- \ref{banas_12_164}
    % \item 12.165 -- \ref{banas_12_165}
    % \item 12.166 -- \ref{banas_12_166}
    % \item 12.167 -- \ref{banas_12_167}
    % \item 12.168 -- \ref{banas_12_168}
    % \item 12.169 -- \ref{banas_12_169}
    % \item 12.170 -- \ref{banas_12_170}
    % \item 12.171 -- \ref{banas_12_171}
    % \item 12.172 -- \ref{banas_12_172}
    % \item 12.173 -- \ref{banas_12_173}
    % \item 12.174 -- \ref{banas_12_174}
    % \item 12.175 -- \ref{banas_12_175}
    % \item 12.176 -- \ref{banas_12_176}
    % \item 12.177 -- \ref{banas_12_177}
    % \item 12.178 -- \ref{banas_12_178}
    % \item 12.179 -- \ref{banas_12_179}
    % \item 12.180 -- \ref{banas_12_180}
    % \item 12.181 -- \ref{banas_12_181}
    % \item 12.182 -- \ref{banas_12_182}
    % \item 12.183 -- \ref{banas_12_183}
    % \item 12.184 -- \ref{banas_12_184}
    % \item 12.185 -- \ref{banas_12_185}
    % \item 12.186 -- \ref{banas_12_186}
    % \item 12.187 -- \ref{banas_12_187}
    % \item 12.188 -- \ref{banas_12_188}
    % \item 12.189 -- \ref{banas_12_189}
    % \item 12.190 -- \ref{banas_12_190}
    % \item 12.191 -- \ref{banas_12_191}
    % \item 12.192 -- \ref{banas_12_192}
    % \item 12.193 -- \ref{banas_12_193}
    % \item 12.194 -- \ref{banas_12_194}
    % \item 12.195 -- \ref{banas_12_195}
    % \item 12.196 -- \ref{banas_12_196}
    % \item 12.197 -- \ref{banas_12_197}
    % \item 12.198 -- \ref{banas_12_198}
    % \item 12.199 -- \ref{banas_12_199}
\end{itemize}
\end{multicols}

%

\indexprologue{\small Tekst prologu I.}
\printindex

\indexprologue{\small Tekst prologu II.}
\printindex[persons]

\raggedright
\bibliography{integrals}{}
\bibliographystyle{plain}

\end{document}

% TODO: https://www.amazon.com/dp/0867202939
% TODO: $$\int_0^\infty \log(x) / (1+x^2) dx = 0$$ (Euler?)
% TODO: $$\int_0^\infty 1/(x^3 - 1) dx = -pi sqrt 3 / 9$$

% TODO: https://math.stackexchange.com/questions/tagged/integration?tab=votes&page=2&pagesize=15
